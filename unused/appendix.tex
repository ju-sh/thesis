\renewcommand{\thechapter}{A}
\section{eDSL for encoding regex} \label{code:edsl-combinators}
The \gls{rgx} that serves as the input to our flow is a Coq term of
type $re\ (\Sigma \to \mathbb{B})$.
These terms are verbose and not easy to get right without a certain
level of familiarity with Coq and the types itself.
In order to make it more convenient to enter the input \gls{rgx}, we
have made an \gls{eDSL} by taking advantage of the notations feature
that allows us to customize the Coq parser.
A few of the constructs of this \gls{eDSL} was presented in
Table~\ref{tab:edsl-constructors}.
The remaining constructs are shown in Table~\ref{tab:edsl}.

\begin{figure}[h]
  \begin{minipage}{0.5\textwidth}%
    \begin{subfigure}{\linewidth}
      \begin{mathpar}
        \begin{array}{ll}
          \toprule
          \text{eDSL} & \text{Meaning}               \\
          \midrule
          T           & \text{Boolean value $true$}  \\
          F           & \text{Boolean value $false$} \\
          \bottomrule
        \end{array}
      \end{mathpar}
      \label{tab:edsl-bool}
      \caption{Boolean-specific}
    \end{subfigure}%
  \end{minipage}%
  \hfill%
  \begin{subfigure}[c]{0.5\linewidth}
    \centering
    \begin{mathpar}
      \begin{array}{ll}
        \toprule
        \text{eDSL}                 & \text{Meaning}                        \\
        \midrule
        @c                          & \text{Character $c$}                  \\
        \cdot                       & \text{Any character}                  \\
        \text{$\textbackslash x$ n} & \text{Character with ASCII value $n$} \\
        \textbackslash w            & \text{Word character}                 \\
        \textbackslash W            & \text{Non-word character}             \\
        \textbackslash s            & \text{Whitespace character}           \\
        \textbackslash S            & \text{Non-whitespace character}       \\
        \textbackslash d            & \text{Digit character}                \\
        \textbackslash D            & \text{Non-digit character}            \\
        \bottomrule
      \end{array}
    \end{mathpar}
    \caption{Character-specific}
    \label{tab:edsl-char}
  \end{subfigure}
  \caption{Syntax of eDSL for input regex}
  \label{tab:edsl}
\end{figure}


\section{Classic regex}
\label{fig:re}

Syntax of classic regex:
% \label{gram:re-syntax}

\begin{mathpar}%\scriptsize
\begin{array}{rcll}
r        & :=    & \emptyset   & \text{(Empty language)}     \\
         & \Big| & \varepsilon & \text{(Empty string)}       \\
         & \Big| & c           & \text{(Character as atom)}  \\ 
         & \Big| & r_1;r_2     & \text{(Concatenation)}      \\ 
         & \Big| & r_1|r_2     & \text{(Choice)}             \\ 
         & \Big| & r^*         & \text{(Zero or more match)} \\
\end{array}
\end{mathpar}

Language associated with classic regex: 
% \label{gram:re-lang}

\begin{mathpar}
\begin{array}{rcl}
L(∅)     & =     & \{\}                                      \\
L(ε)     & =     & \{""\}                                    \\
L(a)     & =     & \{a\}                                     \\
L(r₁;r₂) & =     & \{w₁ w₂ \mid w₁ ∈ L(r₁), w₂ ∈ L(r₂)\}     \\
L(r₁|r₂) & =     & \{w \mid w ∈ L(r₁) ∨ w ∈ L(r₂)\}          \\
L(r*)    & =     & \{w \mid w ∈ L(r;r*) ∨ w ∈ L(ε)\}         \\
\end{array}
\end{mathpar}

Semantics of classic regex:

% \label{gram:re-sem}

\begin{mathpar}
  \inferrule*[right=SEps]
    { }
    {[] \models \varepsilon}

  \inferrule*[right=SAtom]
    {a: T \\ c: T \\ c = a}
    {[c] \models Atom \ a}

  \inferrule*[right=SCat]
    {w_1 \models r_1 \\ w_2 \models r_2}
    {w_1 w_2 \models r_1;r_2}

  \inferrule*[right=SAltL]
    {w \models r_1}
    {w \models r_1|r_2}

  \inferrule*[right=SAltR]
    {w \models r_2}
    {w \models r_1|r_2}

  \inferrule*[right=SStarD]
    {w \models \varepsilon}
    {w \models r*}

  \inferrule*[right=SStarM]
    {w_1 \models r \\ w_2 \models r*}
    {w_1 w_2 \models r*}
\end{mathpar}

\section{A complete Clash design}
The Coq program, extracted Haskell code, Clash design built using this haskell
code and the resultant Verilog code for the \gls{rgx} $\lambda x.\ x==True$ is
shown in this section.

\subsection{Coq program}
Coq file used to generate matcher for the input \gls{rgx}.

{\scriptsize
\begin{Verbatim}[commandchars=\\\{\}]
\PY{k+kn}{From} \PY{n}{courant} \PY{k+kn}{Require} \PY{k+kn}{Import} \PY{n}{prelude}\PY{o}{.}

\PY{k+kn}{Example} \PY{n}{rf} \PY{o}{:=} \PY{n}{re.Atom} \PY{o}{(}\PY{n}{Bool.eqb} \PY{n+nb+bp}{true}\PY{o}{)}\PY{o}{.}
\PY{k+kn}{Example} \PY{n}{rA} \PY{o}{:=} \PY{n}{reΣ2A} \PY{n}{rf}\PY{o}{.}
\PY{k+kn}{Example} \PY{n}{n} \PY{o}{:=} \PY{n}{of\PYZus{}re} \PY{n}{rA}\PY{o}{.}

\PY{k+kn}{Example} \PY{n}{ft} \PY{o}{:=} \PY{k+kn}{Eval} \PY{k}{compute} \PY{k}{in} \PY{n}{driver.f2v1} \PY{o}{(}\PY{n}{reΣ2ft} \PY{n}{rf}\PY{o}{)}\PY{o}{.}
\PY{k+kn}{Example} \PY{n}{strt} \PY{o}{:=} \PY{k+kn}{Eval} \PY{k}{compute} \PY{k}{in} \PY{n}{driver.f2v1} \PY{o}{(}\PY{n}{start} \PY{n}{n}\PY{o}{)}\PY{o}{.}
\PY{k+kn}{Example} \PY{n}{finl} \PY{o}{:=} \PY{k+kn}{Eval} \PY{k}{compute} \PY{k}{in} \PY{n}{driver.f2v1} \PY{o}{(}\PY{n}{final} \PY{n}{n}\PY{o}{)}\PY{o}{.}
\PY{k+kn}{Example} \PY{n}{tf} \PY{o}{:=} \PY{k+kn}{Eval} \PY{k}{compute} \PY{k}{in} \PY{n}{driver.f2v3} \PY{o}{(}\PY{n}{trans} \PY{n}{n}\PY{o}{)}\PY{o}{.}

\PY{n}{Extraction} \PY{n}{strt}\PY{o}{.}
\PY{n}{Extraction} \PY{n}{finl}\PY{o}{.}
\PY{n}{Extraction} \PY{n}{tf}\PY{o}{.}
\PY{n}{Extraction} \PY{n}{ft}\PY{o}{.}
\end{Verbatim}
}


\subsection{Extracted Haskell}
\gls{NFA} information extracted out of Coq as Haskell.

{\scriptsize
\begin{Verbatim}[commandchars=\\\{\}]
\PY{n+nf}{strt}\PY{+w}{ }\PY{o+ow}{::}\PY{+w}{ }\PY{k+kt}{Vec}\PY{+w}{ }\PY{k+kr}{\PYZus{}}\PY{+w}{ }\PY{k+kt}{Bool}
\PY{n+nf}{strt}\PY{+w}{ }\PY{o+ow}{=}
\PY{+w}{  }\PY{p}{(}\PY{k+kt}{:\PYZgt{}}\PY{p}{)}\PY{+w}{ }\PY{k+kt}{True}\PY{+w}{ }\PY{p}{(}\PY{p}{(}\PY{k+kt}{:\PYZgt{}}\PY{p}{)}\PY{+w}{ }\PY{k+kt}{False}\PY{+w}{ }\PY{k+kt}{Nil}\PY{p}{)}


\PY{n+nf}{finl}\PY{+w}{ }\PY{o+ow}{::}\PY{+w}{ }\PY{k+kt}{Vec}\PY{+w}{ }\PY{k+kr}{\PYZus{}}\PY{+w}{ }\PY{k+kt}{Bool}
\PY{n+nf}{finl}\PY{+w}{ }\PY{o+ow}{=}
\PY{+w}{  }\PY{p}{(}\PY{k+kt}{:\PYZgt{}}\PY{p}{)}\PY{+w}{ }\PY{k+kt}{False}\PY{+w}{ }\PY{p}{(}\PY{p}{(}\PY{k+kt}{:\PYZgt{}}\PY{p}{)}\PY{+w}{ }\PY{k+kt}{True}\PY{+w}{ }\PY{k+kt}{Nil}\PY{p}{)}


\PY{n+nf}{tf}\PY{+w}{ }\PY{o+ow}{::}\PY{+w}{ }\PY{k+kt}{Vec}\PY{+w}{ }\PY{k+kr}{\PYZus{}}\PY{+w}{ }\PY{p}{(}\PY{k+kt}{Vec}\PY{+w}{ }\PY{k+kr}{\PYZus{}}\PY{+w}{ }\PY{p}{(}\PY{k+kt}{Vec}\PY{+w}{ }\PY{k+kr}{\PYZus{}}\PY{+w}{ }\PY{k+kt}{Bool}\PY{p}{)}\PY{p}{)}
\PY{n+nf}{tf}\PY{+w}{ }\PY{o+ow}{=}
\PY{+w}{  }\PY{p}{(}\PY{k+kt}{:\PYZgt{}}\PY{p}{)}\PY{+w}{ }\PY{p}{(}\PY{p}{(}\PY{k+kt}{:\PYZgt{}}\PY{p}{)}\PY{+w}{ }\PY{p}{(}\PY{p}{(}\PY{k+kt}{:\PYZgt{}}\PY{p}{)}\PY{+w}{ }\PY{k+kt}{False}\PY{+w}{ }\PY{p}{(}\PY{p}{(}\PY{k+kt}{:\PYZgt{}}\PY{p}{)}\PY{+w}{ }\PY{k+kt}{True}\PY{+w}{ }\PY{k+kt}{Nil}\PY{p}{)}\PY{p}{)}\PY{+w}{ }\PY{p}{(}\PY{p}{(}\PY{k+kt}{:\PYZgt{}}\PY{p}{)}\PY{+w}{ }\PY{p}{(}\PY{p}{(}\PY{k+kt}{:\PYZgt{}}\PY{p}{)}\PY{+w}{ }\PY{k+kt}{False}\PY{+w}{ }\PY{p}{(}\PY{p}{(}\PY{k+kt}{:\PYZgt{}}\PY{p}{)}\PY{+w}{ }\PY{k+kt}{False}\PY{+w}{ }\PY{k+kt}{Nil}\PY{p}{)}\PY{p}{)}
\PY{+w}{    }\PY{k+kt}{Nil}\PY{p}{)}\PY{p}{)}\PY{+w}{ }\PY{k+kt}{Nil}


\PY{n+nf}{ft}\PY{+w}{ }\PY{o+ow}{::}\PY{+w}{ }\PY{k+kt}{Vec}\PY{+w}{ }\PY{k+kr}{\PYZus{}}\PY{+w}{ }\PY{p}{(}\PY{k+kt}{Bool}\PY{+w}{ }\PY{o+ow}{\PYZhy{}\PYZgt{}}\PY{+w}{ }\PY{k+kt}{Bool}\PY{p}{)}
\PY{n+nf}{ft}\PY{+w}{ }\PY{o+ow}{=}
\PY{+w}{  }\PY{p}{(}\PY{k+kt}{:\PYZgt{}}\PY{p}{)}\PY{+w}{ }\PY{p}{(}\PY{n+nf}{\PYZbs{}}\PY{n}{b2}\PY{+w}{ }\PY{o+ow}{\PYZhy{}\PYZgt{}}\PY{+w}{ }\PY{n}{b2}\PY{p}{)}\PY{+w}{ }\PY{k+kt}{Nil}
\end{Verbatim}
}


\subsection{Clash design}
Clash design built from the Haskell code extracted out of Coq.
% by our flow for the regex \code{re.Atom
% (Bool.eqb true)}.

{\scriptsize
\begin{Verbatim}[commandchars=\\\{\}]
\PY{c+cm}{\PYZob{}\PYZhy{}}\PY{c+cm}{\PYZsh{} LANGUAGE PartialTypeSignatures \PYZsh{}}\PY{c+cm}{\PYZhy{}\PYZcb{}}
\PY{c+cm}{\PYZob{}\PYZhy{}}\PY{c+cm}{\PYZsh{} LANGUAGE ScopedTypeVariables \PYZsh{}}\PY{c+cm}{\PYZhy{}\PYZcb{}}
\PY{c+cm}{\PYZob{}\PYZhy{}}\PY{c+cm}{\PYZsh{} LANGUAGE NoImplicitPrelude \PYZsh{}}\PY{c+cm}{\PYZhy{}\PYZcb{}}
\PY{c+cm}{\PYZob{}\PYZhy{}}\PY{c+cm}{\PYZsh{} OPTIONS\PYZus{}GHC }\PY{c+cm}{\PYZhy{}}\PY{c+cm}{fno}\PY{c+cm}{\PYZhy{}}\PY{c+cm}{warn}\PY{c+cm}{\PYZhy{}}\PY{c+cm}{partial}\PY{c+cm}{\PYZhy{}}\PY{c+cm}{type}\PY{c+cm}{\PYZhy{}}\PY{c+cm}{signatures \PYZsh{}}\PY{c+cm}{\PYZhy{}\PYZcb{}}

\PY{k+kr}{module}\PY{+w}{ }\PY{n+nn}{Mon}\PY{+w}{ }\PY{k+kr}{where}

\PY{k+kr}{import}\PY{+w}{ }\PY{n+nn}{Clash.Prelude}
\PY{k+kr}{import}\PY{+w}{ }\PY{n+nn}{Nfadata}

\PY{c+cm}{\PYZob{}\PYZhy{}}\PY{c+cm}{\PYZsh{} ANN topEntity}
\PY{c+cm}{  (Synthesize}
\PY{c+cm}{    }\PY{c+cm}{\PYZob{}}\PY{c+cm}{ t\PYZus{}name   = \PYZdq{}topEntity\PYZdq{}}
\PY{c+cm}{    , t\PYZus{}inputs = [ PortName \PYZdq{}clk\PYZdq{}}
\PY{c+cm}{                 , PortName \PYZdq{}rst\PYZdq{}}
\PY{c+cm}{                 , PortName \PYZdq{}en\PYZdq{}}
\PY{c+cm}{                 , PortName \PYZdq{}inp\PYZdq{} ]}
\PY{c+cm}{    , t\PYZus{}output = PortName \PYZdq{}res\PYZdq{}}
\PY{c+cm}{    }\PY{c+cm}{\PYZcb{}}\PY{c+cm}{) \PYZsh{}}\PY{c+cm}{\PYZhy{}\PYZcb{}}

\PY{n+nf}{dotProduct}
\PY{+w}{  }\PY{o+ow}{::}\PY{+w}{ }\PY{k+kt}{Vec}\PY{+w}{ }\PY{n}{n}\PY{+w}{ }\PY{k+kt}{Bool}
\PY{+w}{  }\PY{o+ow}{\PYZhy{}\PYZgt{}}\PY{+w}{ }\PY{k+kt}{Vec}\PY{+w}{ }\PY{n}{n}\PY{+w}{ }\PY{k+kt}{Bool}
\PY{+w}{  }\PY{o+ow}{\PYZhy{}\PYZgt{}}\PY{+w}{ }\PY{k+kt}{Bool}
\PY{n+nf}{dotProduct}\PY{+w}{ }\PY{n}{v1}\PY{+w}{ }\PY{n}{v2}\PY{+w}{ }\PY{o+ow}{=}
\PY{+w}{  }\PY{n}{foldr}\PY{+w}{ }\PY{p}{(}\PY{o}{||}\PY{p}{)}\PY{+w}{ }\PY{k+kt}{False}\PY{+w}{ }\PY{p}{(}\PY{n}{zipWith}\PY{+w}{ }\PY{p}{(}\PY{o}{\PYZam{}\PYZam{}}\PY{p}{)}\PY{+w}{ }\PY{n}{v1}\PY{+w}{ }\PY{n}{v2}\PY{p}{)}

\PY{n+nf}{mulMxVec}
\PY{+w}{  }\PY{o+ow}{::}\PY{+w}{ }\PY{k+kt}{Vec}\PY{+w}{ }\PY{n}{n}\PY{+w}{ }\PY{p}{(}\PY{k+kt}{Vec}\PY{+w}{ }\PY{n}{m}\PY{+w}{ }\PY{k+kt}{Bool}\PY{p}{)}
\PY{+w}{  }\PY{o+ow}{\PYZhy{}\PYZgt{}}\PY{+w}{ }\PY{k+kt}{Vec}\PY{+w}{ }\PY{n}{m}\PY{+w}{ }\PY{k+kt}{Bool}
\PY{+w}{  }\PY{o+ow}{\PYZhy{}\PYZgt{}}\PY{+w}{ }\PY{k+kt}{Vec}\PY{+w}{ }\PY{n}{n}\PY{+w}{ }\PY{k+kt}{Bool}
\PY{n+nf}{mulMxVec}\PY{+w}{ }\PY{n}{mx}\PY{+w}{ }\PY{n}{v}\PY{+w}{ }\PY{o+ow}{=}
\PY{+w}{  }\PY{n}{map}\PY{+w}{ }\PY{p}{(}\PY{n}{dotProduct}\PY{+w}{ }\PY{n}{v}\PY{p}{)}\PY{+w}{ }\PY{n}{mx}

\PY{n+nf}{plus}
\PY{+w}{  }\PY{o+ow}{::}\PY{+w}{ }\PY{k+kt}{Vec}\PY{+w}{ }\PY{k+kr}{\PYZus{}}\PY{+w}{ }\PY{p}{(}\PY{k+kt}{Vec}\PY{+w}{ }\PY{k+kr}{\PYZus{}}\PY{+w}{ }\PY{k+kt}{Bool}\PY{p}{)}
\PY{+w}{  }\PY{o+ow}{\PYZhy{}\PYZgt{}}\PY{+w}{ }\PY{k+kt}{Vec}\PY{+w}{ }\PY{k+kr}{\PYZus{}}\PY{+w}{ }\PY{p}{(}\PY{k+kt}{Vec}\PY{+w}{ }\PY{k+kr}{\PYZus{}}\PY{+w}{ }\PY{k+kt}{Bool}\PY{p}{)}
\PY{+w}{  }\PY{o+ow}{\PYZhy{}\PYZgt{}}\PY{+w}{ }\PY{k+kt}{Vec}\PY{+w}{ }\PY{k+kr}{\PYZus{}}\PY{+w}{ }\PY{p}{(}\PY{k+kt}{Vec}\PY{+w}{ }\PY{k+kr}{\PYZus{}}\PY{+w}{ }\PY{k+kt}{Bool}\PY{p}{)}
\PY{n+nf}{plus}\PY{+w}{ }\PY{n}{a}\PY{+w}{ }\PY{n}{b}\PY{+w}{ }\PY{o+ow}{=}\PY{+w}{ }\PY{n}{zipWith}\PY{+w}{ }\PY{p}{(}\PY{n}{zipWith}\PY{+w}{ }\PY{p}{(}\PY{o}{||}\PY{p}{)}\PY{p}{)}\PY{+w}{ }\PY{n}{a}\PY{+w}{ }\PY{n}{b}


\PY{n+nf}{mult}
\PY{+w}{  }\PY{o+ow}{::}\PY{+w}{ }\PY{k+kt}{Vec}\PY{+w}{ }\PY{k+kr}{\PYZus{}}\PY{+w}{ }\PY{p}{(}\PY{k+kt}{Vec}\PY{+w}{ }\PY{k+kr}{\PYZus{}}\PY{+w}{ }\PY{p}{(}\PY{k+kt}{Vec}\PY{+w}{ }\PY{k+kr}{\PYZus{}}\PY{+w}{ }\PY{k+kt}{Bool}\PY{p}{)}\PY{p}{)}
\PY{+w}{  }\PY{o+ow}{\PYZhy{}\PYZgt{}}\PY{+w}{ }\PY{k+kt}{Vec}\PY{+w}{ }\PY{k+kr}{\PYZus{}}\PY{+w}{ }\PY{k+kt}{Bool}
\PY{+w}{  }\PY{o+ow}{\PYZhy{}\PYZgt{}}\PY{+w}{ }\PY{k+kt}{Vec}\PY{+w}{ }\PY{k+kr}{\PYZus{}}\PY{+w}{ }\PY{p}{(}\PY{k+kt}{Vec}\PY{+w}{ }\PY{k+kr}{\PYZus{}}\PY{+w}{ }\PY{k+kt}{Bool}\PY{p}{)}
\PY{n+nf}{mult}\PY{+w}{ }\PY{n}{a}\PY{+w}{ }\PY{n}{b}\PY{+w}{ }\PY{o+ow}{=}
\PY{+w}{  }\PY{k+kr}{let}\PY{+w}{ }\PY{n}{tmp}\PY{+w}{ }\PY{o+ow}{=}\PY{+w}{ }\PY{n}{zipWith}\PY{+w}{ }\PY{p}{(}\PY{n+nf}{\PYZbs{}}\PY{n}{tf}\PY{+w}{ }\PY{n}{b}\PY{+w}{ }\PY{o+ow}{\PYZhy{}\PYZgt{}}\PY{+w}{ }\PY{k+kr}{if}\PY{+w}{ }\PY{n}{b}\PY{+w}{ }\PY{k+kr}{then}\PY{+w}{ }\PY{n}{tf}\PY{+w}{ }\PY{k+kr}{else}\PY{+w}{ }\PY{n}{bottom2D}\PY{+w}{ }\PY{n}{tf}\PY{p}{)}\PY{+w}{ }\PY{n}{a}\PY{+w}{ }\PY{n}{b}\PY{+w}{ }\PY{k+kr}{in}
\PY{+w}{  }\PY{n}{foldr}\PY{+w}{ }\PY{n}{plus}\PY{+w}{ }\PY{p}{(}\PY{n}{bottomXD}\PY{+w}{ }\PY{n}{strt}\PY{p}{)}\PY{+w}{ }\PY{n}{tmp}

\PY{n+nf}{run1}
\PY{+w}{  }\PY{o+ow}{::}\PY{+w}{ }\PY{k+kt}{Vec}\PY{+w}{ }\PY{k+kr}{\PYZus{}}\PY{+w}{ }\PY{p}{(}\PY{k+kt}{Vec}\PY{+w}{ }\PY{k+kr}{\PYZus{}}\PY{+w}{ }\PY{p}{(}\PY{k+kt}{Vec}\PY{+w}{ }\PY{k+kr}{\PYZus{}}\PY{+w}{ }\PY{k+kt}{Bool}\PY{p}{)}\PY{p}{)}\PY{+w}{ }\PY{c+c1}{\PYZhy{}\PYZhy{} \PYZca{} tf}
\PY{+w}{  }\PY{o+ow}{\PYZhy{}\PYZgt{}}\PY{+w}{ }\PY{k+kt}{Vec}\PY{+w}{ }\PY{k+kr}{\PYZus{}}\PY{+w}{ }\PY{p}{(}\PY{k+kr}{\PYZus{}}\PY{+w}{ }\PY{o+ow}{\PYZhy{}\PYZgt{}}\PY{+w}{ }\PY{k+kt}{Bool}\PY{p}{)}\PY{+w}{          }\PY{c+c1}{\PYZhy{}\PYZhy{} \PYZca{} ft}
\PY{+w}{  }\PY{o+ow}{\PYZhy{}\PYZgt{}}\PY{+w}{ }\PY{k+kt}{Vec}\PY{+w}{ }\PY{k+kr}{\PYZus{}}\PY{+w}{ }\PY{k+kt}{Bool}\PY{+w}{                 }\PY{c+c1}{\PYZhy{}\PYZhy{} \PYZca{} curst}
\PY{+w}{  }\PY{o+ow}{\PYZhy{}\PYZgt{}}\PY{+w}{ }\PY{k+kt}{Bool}\PY{+w}{                       }\PY{c+c1}{\PYZhy{}\PYZhy{} \PYZca{} inpchar}
\PY{+w}{  }\PY{o+ow}{\PYZhy{}\PYZgt{}}\PY{+w}{ }\PY{k+kt}{Vec}\PY{+w}{ }\PY{k+kr}{\PYZus{}}\PY{+w}{ }\PY{k+kt}{Bool}\PY{+w}{                 }\PY{c+c1}{\PYZhy{}\PYZhy{} \PYZca{} nxtst}
\PY{n+nf}{run1}\PY{+w}{ }\PY{n}{tf}\PY{+w}{ }\PY{n}{ft}\PY{+w}{ }\PY{n}{st}\PY{+w}{ }\PY{n}{i}\PY{+w}{ }\PY{o+ow}{=}
\PY{+w}{  }\PY{k+kr}{let}\PY{+w}{ }\PY{n}{ft\PYZsq{}}\PY{+w}{ }\PY{o+ow}{::}\PY{+w}{ }\PY{k+kt}{Vec}\PY{+w}{ }\PY{k+kr}{\PYZus{}}\PY{+w}{ }\PY{k+kt}{Bool}\PY{+w}{ }\PY{o+ow}{=}\PY{+w}{ }\PY{n}{map}\PY{+w}{ }\PY{p}{(}\PY{n+nf}{\PYZbs{}}\PY{n}{f}\PY{+w}{ }\PY{o+ow}{\PYZhy{}\PYZgt{}}\PY{+w}{ }\PY{n}{f}\PY{+w}{ }\PY{n}{i}\PY{p}{)}\PY{+w}{ }\PY{n}{ft}\PY{+w}{ }\PY{k+kr}{in}
\PY{+w}{  }\PY{k+kr}{let}\PY{+w}{ }\PY{n}{tf\PYZsq{}}\PY{+w}{ }\PY{o+ow}{::}\PY{+w}{ }\PY{k+kt}{Vec}\PY{+w}{ }\PY{k+kr}{\PYZus{}}\PY{+w}{ }\PY{p}{(}\PY{k+kt}{Vec}\PY{+w}{ }\PY{k+kr}{\PYZus{}}\PY{+w}{ }\PY{k+kt}{Bool}\PY{p}{)}\PY{+w}{ }\PY{o+ow}{=}\PY{+w}{ }\PY{n}{transpose}\PY{+w}{ }\PY{o}{\PYZdl{}}\PY{+w}{ }\PY{n}{mult}\PY{+w}{ }\PY{n}{tf}\PY{+w}{ }\PY{n}{ft\PYZsq{}}\PY{+w}{ }\PY{k+kr}{in}
\PY{+w}{  }\PY{n}{mulMxVec}\PY{+w}{ }\PY{n}{tf\PYZsq{}}\PY{+w}{ }\PY{n}{st}

\PY{n+nf}{bottomXD}\PY{+w}{ }\PY{o+ow}{::}\PY{+w}{ }\PY{k+kt}{Vec}\PY{+w}{ }\PY{k+kr}{\PYZus{}}\PY{+w}{ }\PY{k+kt}{Bool}\PY{+w}{ }\PY{o+ow}{\PYZhy{}\PYZgt{}}\PY{+w}{ }\PY{k+kt}{Vec}\PY{+w}{ }\PY{k+kr}{\PYZus{}}\PY{+w}{ }\PY{p}{(}\PY{k+kt}{Vec}\PY{+w}{ }\PY{k+kr}{\PYZus{}}\PY{+w}{ }\PY{k+kt}{Bool}\PY{p}{)}
\PY{n+nf}{bottomXD}\PY{+w}{ }\PY{n}{v}\PY{+w}{ }\PY{o+ow}{=}
\PY{+w}{  }\PY{k+kr}{let}\PY{+w}{ }\PY{n}{x}\PY{+w}{ }\PY{o+ow}{=}\PY{+w}{ }\PY{n}{const}\PY{+w}{ }\PY{k+kt}{False}\PY{+w}{ }\PY{o}{\PYZlt{}\PYZdl{}\PYZgt{}}\PY{+w}{ }\PY{n}{v}\PY{+w}{ }\PY{k+kr}{in}
\PY{+w}{  }\PY{p}{(}\PY{n}{const}\PY{+w}{ }\PY{n}{x}\PY{p}{)}\PY{+w}{ }\PY{o}{\PYZlt{}\PYZdl{}\PYZgt{}}\PY{+w}{ }\PY{n}{x}

\PY{n+nf}{bottom2D}\PY{+w}{ }\PY{o+ow}{::}\PY{+w}{ }\PY{k+kt}{Vec}\PY{+w}{ }\PY{k+kr}{\PYZus{}}\PY{+w}{ }\PY{p}{(}\PY{k+kt}{Vec}\PY{+w}{ }\PY{k+kr}{\PYZus{}}\PY{+w}{ }\PY{k+kt}{Bool}\PY{p}{)}\PY{+w}{ }\PY{o+ow}{\PYZhy{}\PYZgt{}}\PY{+w}{ }\PY{k+kt}{Vec}\PY{+w}{ }\PY{k+kr}{\PYZus{}}\PY{+w}{ }\PY{p}{(}\PY{k+kt}{Vec}\PY{+w}{ }\PY{k+kr}{\PYZus{}}\PY{+w}{ }\PY{k+kt}{Bool}\PY{p}{)}
\PY{n+nf}{bottom2D}\PY{+w}{ }\PY{o+ow}{=}\PY{+w}{ }\PY{p}{(}\PY{p}{(}\PY{n}{const}\PY{+w}{ }\PY{k+kt}{False}\PY{+w}{ }\PY{o}{\PYZlt{}\PYZdl{}\PYZgt{}}\PY{p}{)}\PY{+w}{ }\PY{o}{\PYZlt{}\PYZdl{}\PYZgt{}}\PY{p}{)}

\PY{c+c1}{\PYZhy{}\PYZhy{} At least one True}
\PY{n+nf}{anyTru}\PY{+w}{ }\PY{o+ow}{::}\PY{+w}{ }\PY{k+kt}{Vec}\PY{+w}{ }\PY{k+kr}{\PYZus{}}\PY{+w}{ }\PY{k+kt}{Bool}\PY{+w}{ }\PY{o+ow}{\PYZhy{}\PYZgt{}}\PY{+w}{ }\PY{k+kt}{Bool}
\PY{n+nf}{anyTru}\PY{+w}{ }\PY{n}{v}\PY{+w}{ }\PY{o+ow}{=}\PY{+w}{ }\PY{n}{foldr}\PY{+w}{ }\PY{p}{(}\PY{o}{||}\PY{p}{)}\PY{+w}{ }\PY{k+kt}{False}\PY{+w}{ }\PY{n}{v}

\PY{n+nf}{isFinal}
\PY{+w}{  }\PY{o+ow}{::}\PY{+w}{ }\PY{k+kt}{Vec}\PY{+w}{ }\PY{k+kr}{\PYZus{}}\PY{+w}{ }\PY{k+kt}{Bool}\PY{+w}{  }\PY{c+c1}{\PYZhy{}\PYZhy{} State}
\PY{+w}{  }\PY{o+ow}{\PYZhy{}\PYZgt{}}\PY{+w}{ }\PY{k+kt}{Vec}\PY{+w}{ }\PY{k+kr}{\PYZus{}}\PY{+w}{ }\PY{k+kt}{Bool}\PY{+w}{  }\PY{c+c1}{\PYZhy{}\PYZhy{} Finals}
\PY{+w}{  }\PY{o+ow}{\PYZhy{}\PYZgt{}}\PY{+w}{ }\PY{k+kt}{Bool}\PY{+w}{        }\PY{c+c1}{\PYZhy{}\PYZhy{} Result}
\PY{n+nf}{isFinal}\PY{+w}{ }\PY{n}{st}\PY{+w}{ }\PY{n}{fin}\PY{+w}{ }\PY{o+ow}{=}\PY{+w}{ }\PY{n}{anyTru}\PY{+w}{ }\PY{o}{\PYZdl{}}\PY{+w}{ }\PY{n}{zipWith}\PY{+w}{ }\PY{p}{(}\PY{o}{\PYZam{}\PYZam{}}\PY{p}{)}\PY{+w}{ }\PY{n}{st}\PY{+w}{ }\PY{n}{fin}

\PY{c+c1}{\PYZhy{}\PYZhy{} All are False}
\PY{n+nf}{isNull}
\PY{+w}{  }\PY{o+ow}{::}\PY{+w}{ }\PY{k+kt}{Vec}\PY{+w}{ }\PY{k+kr}{\PYZus{}}\PY{+w}{ }\PY{k+kt}{Bool}\PY{+w}{  }\PY{c+c1}{\PYZhy{}\PYZhy{} State}
\PY{+w}{  }\PY{o+ow}{\PYZhy{}\PYZgt{}}\PY{+w}{ }\PY{k+kt}{Bool}\PY{+w}{        }\PY{c+c1}{\PYZhy{}\PYZhy{} Result}
\PY{n+nf}{isNull}\PY{+w}{ }\PY{n}{v}\PY{+w}{ }\PY{o+ow}{=}\PY{+w}{ }\PY{n}{not}\PY{+w}{ }\PY{o}{\PYZdl{}}\PY{+w}{ }\PY{n}{anyTru}\PY{+w}{ }\PY{n}{v}

\PY{n+nf}{wr}
\PY{+w}{  }\PY{o+ow}{::}\PY{+w}{ }\PY{k+kt}{Vec}\PY{+w}{ }\PY{k+kr}{\PYZus{}}\PY{+w}{ }\PY{k+kt}{Bool}
\PY{+w}{  }\PY{o+ow}{\PYZhy{}\PYZgt{}}\PY{+w}{ }\PY{k+kr}{\PYZus{}}
\PY{+w}{  }\PY{o+ow}{\PYZhy{}\PYZgt{}}\PY{+w}{ }\PY{p}{(}\PY{k+kt}{Vec}\PY{+w}{ }\PY{k+kr}{\PYZus{}}\PY{+w}{ }\PY{k+kt}{Bool}\PY{p}{,}\PY{+w}{ }\PY{k+kt}{Bool}\PY{p}{)}
\PY{n+nf}{wr}\PY{+w}{ }\PY{n}{st}\PY{+w}{ }\PY{n}{i}\PY{+w}{ }\PY{o+ow}{=}
\PY{+w}{  }\PY{k+kr}{let}\PY{+w}{ }\PY{n}{st\PYZsq{}}\PY{+w}{ }\PY{o+ow}{=}\PY{+w}{ }\PY{n}{run1}\PY{+w}{ }\PY{n}{tf}\PY{+w}{ }\PY{n}{ft}\PY{+w}{ }\PY{n}{st}\PY{+w}{ }\PY{n}{i}\PY{+w}{ }\PY{k+kr}{in}
\PY{+w}{  }\PY{k+kr}{let}\PY{+w}{ }\PY{n}{ot\PYZsq{}}\PY{+w}{ }\PY{o+ow}{=}\PY{+w}{ }\PY{n}{isFinal}\PY{+w}{ }\PY{n}{st\PYZsq{}}\PY{+w}{ }\PY{n}{finl}\PY{+w}{ }\PY{k+kr}{in}
\PY{+w}{  }\PY{p}{(}\PY{n}{st\PYZsq{}}\PY{p}{,}\PY{+w}{ }\PY{n}{ot\PYZsq{}}\PY{p}{)}

\PY{n+nf}{mon}
\PY{+w}{  }\PY{o+ow}{::}\PY{+w}{ }\PY{k+kt}{SystemClockResetEnable}
\PY{+w}{  }\PY{o+ow}{=\PYZgt{}}\PY{+w}{ }\PY{k+kt}{Signal}\PY{+w}{ }\PY{k+kt}{System}\PY{+w}{ }\PY{k+kr}{\PYZus{}}
\PY{+w}{  }\PY{o+ow}{\PYZhy{}\PYZgt{}}\PY{+w}{ }\PY{k+kt}{Signal}\PY{+w}{ }\PY{k+kt}{System}\PY{+w}{ }\PY{k+kt}{Bool}
\PY{n+nf}{mon}\PY{+w}{ }\PY{o+ow}{=}\PY{+w}{ }\PY{n}{mealy}\PY{+w}{ }\PY{n}{wr}\PY{+w}{ }\PY{n}{strt}

\PY{n+nf}{topEntity}
\PY{+w}{  }\PY{o+ow}{::}\PY{+w}{ }\PY{k+kt}{Clock}\PY{+w}{ }\PY{k+kt}{System}
\PY{+w}{  }\PY{o+ow}{\PYZhy{}\PYZgt{}}\PY{+w}{ }\PY{k+kt}{Reset}\PY{+w}{ }\PY{k+kt}{System}
\PY{+w}{  }\PY{o+ow}{\PYZhy{}\PYZgt{}}\PY{+w}{ }\PY{k+kt}{Enable}\PY{+w}{ }\PY{k+kt}{System}
\PY{+w}{  }\PY{o+ow}{\PYZhy{}\PYZgt{}}\PY{+w}{ }\PY{k+kt}{Signal}\PY{+w}{ }\PY{k+kt}{System}\PY{+w}{ }\PY{k+kr}{\PYZus{}}
\PY{+w}{  }\PY{o+ow}{\PYZhy{}\PYZgt{}}\PY{+w}{ }\PY{k+kt}{Signal}\PY{+w}{ }\PY{k+kt}{System}\PY{+w}{ }\PY{k+kt}{Bool}
\PY{n+nf}{topEntity}\PY{+w}{ }\PY{o+ow}{=}\PY{+w}{ }\PY{n}{exposeClockResetEnable}\PY{+w}{ }\PY{n}{mon}
\end{Verbatim}
}

\subsection{Resultant Verilog}
Verilog code generated by the Clash compiler using our Clash design.

{\scriptsize
\begin{Verbatim}[commandchars=\\\{\}]
\PY{c+cm}{/* AUTOMATICALLY GENERATED VERILOG\PYZhy{}2001 SOURCE CODE.}
\PY{c+cm}{** GENERATED BY CLASH 1.6.3. DO NOT MODIFY.}
\PY{c+cm}{*/}
\PY{n+no}{`timescale}\PY{+w}{ }\PY{l+m+mf}{100f}\PY{n}{s}\PY{o}{/}\PY{l+m+mf}{100f}\PY{n}{s}
\PY{k}{module}\PY{+w}{ }\PY{n}{topEntity}
\PY{+w}{    }\PY{p}{(}\PY{+w}{ }\PY{c+c1}{// Inputs}
\PY{+w}{      }\PY{k}{input}\PY{+w}{  }\PY{n}{clk}\PY{+w}{ }\PY{c+c1}{// clock}
\PY{+w}{    }\PY{p}{,}\PY{+w}{ }\PY{k}{input}\PY{+w}{  }\PY{n}{rst}\PY{+w}{ }\PY{c+c1}{// reset}
\PY{+w}{    }\PY{p}{,}\PY{+w}{ }\PY{k}{input}\PY{+w}{  }\PY{n}{en}\PY{+w}{ }\PY{c+c1}{// enable}
\PY{+w}{    }\PY{p}{,}\PY{+w}{ }\PY{k}{input}\PY{+w}{  }\PY{n}{inp}

\PY{+w}{      }\PY{c+c1}{// Outputs}
\PY{+w}{    }\PY{p}{,}\PY{+w}{ }\PY{k}{output}\PY{+w}{ }\PY{k+kt}{wire}\PY{+w}{  }\PY{n}{res}
\PY{+w}{    }\PY{p}{)}\PY{p}{;}
\PY{+w}{  }\PY{c+c1}{// src/Mon.hs:98:1\PYZhy{}3}
\PY{+w}{  }\PY{k+kt}{reg}\PY{+w}{ }\PY{p}{[}\PY{l+m+mh}{1}\PY{o}{:}\PY{l+m+mh}{0}\PY{p}{]}\PY{+w}{ }\PY{n}{c}\PY{n}{\PYZdl{}ds\PYZus{}app\PYZus{}arg}\PY{+w}{ }\PY{o}{=}\PY{+w}{ }\PY{p}{\PYZob{}}\PY{l+m+mh}{1}\PY{l+m+mb}{\PYZsq{}b1}\PY{p}{,}\PY{+w}{   }\PY{l+m+mh}{1}\PY{l+m+mb}{\PYZsq{}b0}\PY{p}{\PYZcb{}}\PY{p}{;}
\PY{+w}{  }\PY{k+kt}{wire}\PY{+w}{ }\PY{p}{[}\PY{l+m+mh}{3}\PY{o}{:}\PY{l+m+mh}{0}\PY{p}{]}\PY{+w}{ }\PY{n}{c}\PY{n}{\PYZdl{}app\PYZus{}arg}\PY{p}{;}
\PY{+w}{  }\PY{k+kt}{wire}\PY{+w}{ }\PY{p}{[}\PY{l+m+mh}{3}\PY{o}{:}\PY{l+m+mh}{0}\PY{p}{]}\PY{+w}{ }\PY{n}{c}\PY{n}{\PYZdl{}app\PYZus{}arg\PYZus{}0}\PY{p}{;}
\PY{+w}{  }\PY{k+kt}{wire}\PY{+w}{ }\PY{p}{[}\PY{l+m+mh}{3}\PY{o}{:}\PY{l+m+mh}{0}\PY{p}{]}\PY{+w}{ }\PY{n}{result}\PY{p}{;}
\PY{+w}{  }\PY{k+kt}{wire}\PY{+w}{ }\PY{p}{[}\PY{l+m+mh}{1}\PY{o}{:}\PY{l+m+mh}{0}\PY{p}{]}\PY{+w}{ }\PY{n}{result\PYZus{}0}\PY{p}{;}
\PY{+w}{  }\PY{k+kt}{wire}\PY{+w}{ }\PY{p}{[}\PY{l+m+mh}{2}\PY{o}{:}\PY{l+m+mh}{0}\PY{p}{]}\PY{+w}{ }\PY{n}{result\PYZus{}1}\PY{p}{;}
\PY{+w}{  }\PY{k+kt}{wire}\PY{+w}{  }\PY{n}{c}\PY{n}{\PYZdl{}app\PYZus{}arg\PYZus{}1}\PY{p}{;}
\PY{+w}{  }\PY{k+kt}{wire}\PY{+w}{ }\PY{p}{[}\PY{l+m+mh}{1}\PY{o}{:}\PY{l+m+mh}{0}\PY{p}{]}\PY{+w}{ }\PY{n}{c}\PY{n}{\PYZdl{}app\PYZus{}arg\PYZus{}2}\PY{p}{;}
\PY{+w}{  }\PY{k+kt}{wire}\PY{+w}{ }\PY{p}{[}\PY{l+m+mh}{3}\PY{o}{:}\PY{l+m+mh}{0}\PY{p}{]}\PY{+w}{ }\PY{n}{c}\PY{n}{\PYZdl{}app\PYZus{}arg\PYZus{}sel\PYZus{}alt\PYZus{}f\PYZus{}res}\PY{p}{;}
\PY{+w}{  }\PY{k+kt}{wire}\PY{+w}{ }\PY{p}{[}\PY{l+m+mh}{3}\PY{o}{:}\PY{l+m+mh}{0}\PY{p}{]}\PY{+w}{ }\PY{n}{c}\PY{n}{\PYZdl{}vec}\PY{p}{;}
\PY{+w}{  }\PY{k+kt}{wire}\PY{+w}{ }\PY{p}{[}\PY{l+m+mh}{3}\PY{o}{:}\PY{l+m+mh}{0}\PY{p}{]}\PY{+w}{ }\PY{n}{result\PYZus{}res}\PY{p}{;}
\PY{+w}{  }\PY{k+kt}{wire}\PY{+w}{ }\PY{p}{[}\PY{l+m+mh}{1}\PY{o}{:}\PY{l+m+mh}{0}\PY{p}{]}\PY{+w}{ }\PY{n}{result\PYZus{}res\PYZus{}res}\PY{p}{;}
\PY{+w}{  }\PY{k+kt}{wire}\PY{+w}{ }\PY{p}{[}\PY{l+m+mh}{1}\PY{o}{:}\PY{l+m+mh}{0}\PY{p}{]}\PY{+w}{ }\PY{n}{c}\PY{n}{\PYZdl{}vec\PYZus{}0}\PY{p}{;}
\PY{+w}{  }\PY{k+kt}{wire}\PY{+w}{ }\PY{p}{[}\PY{l+m+mh}{3}\PY{o}{:}\PY{l+m+mh}{0}\PY{p}{]}\PY{+w}{ }\PY{n}{c}\PY{n}{\PYZdl{}xs}\PY{p}{;}
\PY{+w}{  }\PY{k+kt}{wire}\PY{+w}{ }\PY{p}{[}\PY{l+m+mh}{1}\PY{o}{:}\PY{l+m+mh}{0}\PY{p}{]}\PY{+w}{ }\PY{n}{c}\PY{n}{\PYZdl{}vec2}\PY{p}{;}

\PY{+w}{  }\PY{c+c1}{// register begin}
\PY{+w}{  }\PY{k}{always}\PY{+w}{ }\PY{p}{@}\PY{p}{(}\PY{k}{posedge}\PY{+w}{ }\PY{n}{clk}\PY{+w}{ }\PY{k}{or}\PY{+w}{  }\PY{k}{posedge}\PY{+w}{  }\PY{n}{rst}\PY{p}{)}\PY{+w}{ }\PY{k}{begin}\PY{+w}{ }\PY{o}{:}\PY{+w}{ }\PY{n}{c}\PY{n}{\PYZdl{}ds\PYZus{}app\PYZus{}arg\PYZus{}register}
\PY{+w}{    }\PY{k}{if}\PY{+w}{ }\PY{p}{(}\PY{+w}{ }\PY{n}{rst}\PY{p}{)}\PY{+w}{ }\PY{k}{begin}
\PY{+w}{      }\PY{n}{c}\PY{n}{\PYZdl{}ds\PYZus{}app\PYZus{}arg}\PY{+w}{ }\PY{o}{\PYZlt{}}\PY{o}{=}\PY{+w}{ }\PY{p}{\PYZob{}}\PY{l+m+mh}{1}\PY{l+m+mb}{\PYZsq{}b1}\PY{p}{,}\PY{+w}{   }\PY{l+m+mh}{1}\PY{l+m+mb}{\PYZsq{}b0}\PY{p}{\PYZcb{}}\PY{p}{;}
\PY{+w}{    }\PY{k}{end}\PY{+w}{ }\PY{k}{else}\PY{+w}{ }\PY{k}{if}\PY{+w}{ }\PY{p}{(}\PY{n}{en}\PY{p}{)}\PY{+w}{ }\PY{k}{begin}
\PY{+w}{      }\PY{n}{c}\PY{n}{\PYZdl{}ds\PYZus{}app\PYZus{}arg}\PY{+w}{ }\PY{o}{\PYZlt{}}\PY{o}{=}\PY{+w}{ }\PY{n}{result\PYZus{}1}\PY{p}{[}\PY{l+m+mh}{2}\PY{o}{:}\PY{l+m+mh}{1}\PY{p}{]}\PY{p}{;}
\PY{+w}{    }\PY{k}{end}
\PY{+w}{  }\PY{k}{end}
\PY{+w}{  }\PY{c+c1}{// register end}

\PY{+w}{  }\PY{k}{assign}\PY{+w}{ }\PY{n}{res}\PY{+w}{ }\PY{o}{=}\PY{+w}{ }\PY{n}{result\PYZus{}1}\PY{p}{[}\PY{l+m+mh}{0}\PY{o}{:}\PY{l+m+mh}{0}\PY{p}{]}\PY{p}{;}

\PY{+w}{  }\PY{c+c1}{// transpose begin}
\PY{+w}{  }\PY{k}{genvar}\PY{+w}{ }\PY{n}{row\PYZus{}index}\PY{p}{;}
\PY{+w}{  }\PY{k}{genvar}\PY{+w}{ }\PY{n}{col\PYZus{}index}\PY{p}{;}
\PY{+w}{  }\PY{k}{generate}
\PY{+w}{  }\PY{k}{for}\PY{+w}{ }\PY{p}{(}\PY{n}{row\PYZus{}index}\PY{+w}{ }\PY{o}{=}\PY{+w}{ }\PY{l+m+mh}{0}\PY{p}{;}\PY{+w}{ }\PY{n}{row\PYZus{}index}\PY{+w}{ }\PY{o}{\PYZlt{}}\PY{+w}{ }\PY{l+m+mh}{2}\PY{p}{;}\PY{+w}{ }\PY{n}{row\PYZus{}index}\PY{+w}{ }\PY{o}{=}\PY{+w}{ }\PY{n}{row\PYZus{}index}\PY{+w}{ }\PY{o}{+}\PY{+w}{ }\PY{l+m+mh}{1}\PY{p}{)}\PY{+w}{ }\PY{k}{begin}\PY{+w}{ }\PY{o}{:}\PY{+w}{ }\PY{n}{transpose\PYZus{}outer}
\PY{+w}{    }\PY{k}{for}\PY{+w}{ }\PY{p}{(}\PY{n}{col\PYZus{}index}\PY{+w}{ }\PY{o}{=}\PY{+w}{ }\PY{l+m+mh}{0}\PY{p}{;}\PY{+w}{ }\PY{n}{col\PYZus{}index}\PY{+w}{ }\PY{o}{\PYZlt{}}\PY{+w}{ }\PY{l+m+mh}{2}\PY{p}{;}\PY{+w}{ }\PY{n}{col\PYZus{}index}\PY{+w}{ }\PY{o}{=}\PY{+w}{ }\PY{n}{col\PYZus{}index}\PY{+w}{ }\PY{o}{+}\PY{+w}{ }\PY{l+m+mh}{1}\PY{p}{)}\PY{+w}{ }\PY{k}{begin}\PY{+w}{ }\PY{o}{:}\PY{+w}{ }\PY{n}{transpose\PYZus{}inner}
\PY{+w}{      }\PY{k}{assign}\PY{+w}{ }\PY{n}{c}\PY{n}{\PYZdl{}app\PYZus{}arg}\PY{p}{[}\PY{p}{(}\PY{p}{(}\PY{n}{col\PYZus{}index}\PY{o}{*}\PY{l+m+mh}{2}\PY{p}{)}\PY{o}{+}\PY{p}{(}\PY{n}{row\PYZus{}index}\PY{o}{*}\PY{l+m+mh}{1}\PY{p}{)}\PY{p}{)}\PY{o}{+}\PY{o}{:}\PY{l+m+mh}{1}\PY{p}{]}\PY{+w}{ }\PY{o}{=}\PY{+w}{ }\PY{n}{result}\PY{p}{[}\PY{p}{(}\PY{p}{(}\PY{n}{row\PYZus{}index}\PY{o}{*}\PY{l+m+mh}{2}\PY{p}{)}\PY{o}{+}\PY{p}{(}\PY{n}{col\PYZus{}index}\PY{o}{*}\PY{l+m+mh}{1}\PY{p}{)}\PY{p}{)}\PY{o}{+}\PY{o}{:}\PY{l+m+mh}{1}\PY{p}{]}\PY{p}{;}
\PY{+w}{    }\PY{k}{end}
\PY{+w}{  }\PY{k}{end}
\PY{+w}{  }\PY{k}{endgenerate}
\PY{+w}{  }\PY{c+c1}{// transpose end}

\PY{+w}{  }\PY{k}{assign}\PY{+w}{ }\PY{n}{c}\PY{n}{\PYZdl{}vec}\PY{+w}{ }\PY{o}{=}\PY{+w}{ }\PY{p}{\PYZob{}}\PY{p}{\PYZob{}}\PY{l+m+mh}{1}\PY{l+m+mb}{\PYZsq{}b0}\PY{p}{,}\PY{+w}{   }\PY{l+m+mh}{1}\PY{l+m+mb}{\PYZsq{}b1}\PY{p}{\PYZcb{}}\PY{p}{,}\PY{+w}{   }\PY{p}{\PYZob{}}\PY{l+m+mh}{1}\PY{l+m+mb}{\PYZsq{}b0}\PY{p}{,}
\PY{+w}{                                     }\PY{l+m+mh}{1}\PY{l+m+mb}{\PYZsq{}b0}\PY{p}{\PYZcb{}}\PY{p}{\PYZcb{}}\PY{p}{;}

\PY{+w}{  }\PY{c+c1}{// map begin}
\PY{+w}{  }\PY{k}{genvar}\PY{+w}{ }\PY{n}{i\PYZus{}0}\PY{p}{;}
\PY{+w}{  }\PY{k}{generate}
\PY{+w}{  }\PY{k}{for}\PY{+w}{ }\PY{p}{(}\PY{n}{i\PYZus{}0}\PY{o}{=}\PY{l+m+mh}{0}\PY{p}{;}\PY{+w}{ }\PY{n}{i\PYZus{}0}\PY{+w}{ }\PY{o}{\PYZlt{}}\PY{+w}{ }\PY{l+m+mh}{2}\PY{p}{;}\PY{+w}{ }\PY{n}{i\PYZus{}0}\PY{+w}{ }\PY{o}{=}\PY{+w}{ }\PY{n}{i\PYZus{}0}\PY{+w}{ }\PY{o}{+}\PY{+w}{ }\PY{l+m+mh}{1}\PY{p}{)}\PY{+w}{ }\PY{k}{begin}\PY{+w}{ }\PY{o}{:}\PY{+w}{ }\PY{n}{map\PYZus{}0}
\PY{+w}{    }\PY{k+kt}{wire}\PY{+w}{ }\PY{p}{[}\PY{l+m+mh}{1}\PY{o}{:}\PY{l+m+mh}{0}\PY{p}{]}\PY{+w}{ }\PY{n}{map\PYZus{}in\PYZus{}0}\PY{p}{;}
\PY{+w}{    }\PY{k}{assign}\PY{+w}{ }\PY{n}{map\PYZus{}in\PYZus{}0}\PY{+w}{ }\PY{o}{=}\PY{+w}{ }\PY{n}{c}\PY{n}{\PYZdl{}vec}\PY{p}{[}\PY{n}{i\PYZus{}0}\PY{o}{*}\PY{l+m+mh}{2}\PY{o}{+}\PY{o}{:}\PY{l+m+mh}{2}\PY{p}{]}\PY{p}{;}
\PY{+w}{    }\PY{k+kt}{wire}\PY{+w}{ }\PY{p}{[}\PY{l+m+mh}{1}\PY{o}{:}\PY{l+m+mh}{0}\PY{p}{]}\PY{+w}{ }\PY{n}{map\PYZus{}out\PYZus{}0}\PY{p}{;}
\PY{+w}{    }\PY{c+c1}{// map begin}
\PY{+w}{    }\PY{k}{genvar}\PY{+w}{ }\PY{n}{i\PYZus{}12}\PY{p}{;}

\PY{+w}{    }\PY{k}{for}\PY{+w}{ }\PY{p}{(}\PY{n}{i\PYZus{}12}\PY{o}{=}\PY{l+m+mh}{0}\PY{p}{;}\PY{+w}{ }\PY{n}{i\PYZus{}12}\PY{+w}{ }\PY{o}{\PYZlt{}}\PY{+w}{ }\PY{l+m+mh}{2}\PY{p}{;}\PY{+w}{ }\PY{n}{i\PYZus{}12}\PY{+w}{ }\PY{o}{=}\PY{+w}{ }\PY{n}{i\PYZus{}12}\PY{+w}{ }\PY{o}{+}\PY{+w}{ }\PY{l+m+mh}{1}\PY{p}{)}\PY{+w}{ }\PY{k}{begin}\PY{+w}{ }\PY{o}{:}\PY{+w}{ }\PY{n}{map\PYZus{}5}
\PY{+w}{      }\PY{k+kt}{wire}\PY{+w}{  }\PY{n}{map\PYZus{}in\PYZus{}5}\PY{p}{;}
\PY{+w}{      }\PY{k}{assign}\PY{+w}{ }\PY{n}{map\PYZus{}in\PYZus{}5}\PY{+w}{ }\PY{o}{=}\PY{+w}{ }\PY{n}{map\PYZus{}in\PYZus{}0}\PY{p}{[}\PY{n}{i\PYZus{}12}\PY{o}{*}\PY{l+m+mh}{1}\PY{o}{+}\PY{o}{:}\PY{l+m+mh}{1}\PY{p}{]}\PY{p}{;}
\PY{+w}{      }\PY{k+kt}{wire}\PY{+w}{  }\PY{n}{map\PYZus{}out\PYZus{}5}\PY{p}{;}
\PY{+w}{      }\PY{k}{assign}\PY{+w}{ }\PY{n}{map\PYZus{}out\PYZus{}5}\PY{+w}{ }\PY{o}{=}\PY{+w}{ }\PY{l+m+mh}{1}\PY{l+m+mb}{\PYZsq{}b0}\PY{p}{;}


\PY{+w}{      }\PY{k}{assign}\PY{+w}{ }\PY{n}{map\PYZus{}out\PYZus{}0}\PY{p}{[}\PY{n}{i\PYZus{}12}\PY{o}{*}\PY{l+m+mh}{1}\PY{o}{+}\PY{o}{:}\PY{l+m+mh}{1}\PY{p}{]}\PY{+w}{ }\PY{o}{=}\PY{+w}{ }\PY{n}{map\PYZus{}out\PYZus{}5}\PY{p}{;}
\PY{+w}{    }\PY{k}{end}

\PY{+w}{    }\PY{c+c1}{// map end}
\PY{+w}{    }\PY{k}{assign}\PY{+w}{ }\PY{n}{c}\PY{n}{\PYZdl{}app\PYZus{}arg\PYZus{}sel\PYZus{}alt\PYZus{}f\PYZus{}res}\PY{p}{[}\PY{n}{i\PYZus{}0}\PY{o}{*}\PY{l+m+mh}{2}\PY{o}{+}\PY{o}{:}\PY{l+m+mh}{2}\PY{p}{]}\PY{+w}{ }\PY{o}{=}\PY{+w}{ }\PY{n}{map\PYZus{}out\PYZus{}0}\PY{p}{;}
\PY{+w}{  }\PY{k}{end}
\PY{+w}{  }\PY{k}{endgenerate}
\PY{+w}{  }\PY{c+c1}{// map end}

\PY{+w}{  }\PY{k}{assign}\PY{+w}{ }\PY{n}{c}\PY{n}{\PYZdl{}app\PYZus{}arg\PYZus{}0}\PY{+w}{ }\PY{o}{=}\PY{+w}{ }\PY{n}{inp}\PY{+w}{ }\PY{o}{?}\PY{+w}{ }\PY{p}{\PYZob{}}\PY{p}{\PYZob{}}\PY{l+m+mh}{1}\PY{l+m+mb}{\PYZsq{}b0}\PY{p}{,}\PY{+w}{   }\PY{l+m+mh}{1}\PY{l+m+mb}{\PYZsq{}b1}\PY{p}{\PYZcb{}}\PY{p}{,}
\PY{+w}{                              }\PY{p}{\PYZob{}}\PY{l+m+mh}{1}\PY{l+m+mb}{\PYZsq{}b0}\PY{p}{,}\PY{+w}{   }\PY{l+m+mh}{1}\PY{l+m+mb}{\PYZsq{}b0}\PY{p}{\PYZcb{}}\PY{p}{\PYZcb{}}\PY{+w}{ }\PY{o}{:}\PY{+w}{ }\PY{n}{c}\PY{n}{\PYZdl{}app\PYZus{}arg\PYZus{}sel\PYZus{}alt\PYZus{}f\PYZus{}res}\PY{p}{;}

\PY{+w}{  }\PY{k}{assign}\PY{+w}{ }\PY{n}{c}\PY{n}{\PYZdl{}vec\PYZus{}0}\PY{+w}{ }\PY{o}{=}\PY{+w}{ }\PY{p}{\PYZob{}}\PY{l+m+mh}{1}\PY{l+m+mb}{\PYZsq{}b1}\PY{p}{,}\PY{+w}{   }\PY{l+m+mh}{1}\PY{l+m+mb}{\PYZsq{}b0}\PY{p}{\PYZcb{}}\PY{p}{;}

\PY{+w}{  }\PY{c+c1}{// map begin}
\PY{+w}{  }\PY{k}{genvar}\PY{+w}{ }\PY{n}{i\PYZus{}1}\PY{p}{;}
\PY{+w}{  }\PY{k}{generate}
\PY{+w}{  }\PY{k}{for}\PY{+w}{ }\PY{p}{(}\PY{n}{i\PYZus{}1}\PY{o}{=}\PY{l+m+mh}{0}\PY{p}{;}\PY{+w}{ }\PY{n}{i\PYZus{}1}\PY{+w}{ }\PY{o}{\PYZlt{}}\PY{+w}{ }\PY{l+m+mh}{2}\PY{p}{;}\PY{+w}{ }\PY{n}{i\PYZus{}1}\PY{+w}{ }\PY{o}{=}\PY{+w}{ }\PY{n}{i\PYZus{}1}\PY{+w}{ }\PY{o}{+}\PY{+w}{ }\PY{l+m+mh}{1}\PY{p}{)}\PY{+w}{ }\PY{k}{begin}\PY{+w}{ }\PY{o}{:}\PY{+w}{ }\PY{n}{map\PYZus{}1}
\PY{+w}{    }\PY{k+kt}{wire}\PY{+w}{  }\PY{n}{map\PYZus{}in\PYZus{}1}\PY{p}{;}
\PY{+w}{    }\PY{k}{assign}\PY{+w}{ }\PY{n}{map\PYZus{}in\PYZus{}1}\PY{+w}{ }\PY{o}{=}\PY{+w}{ }\PY{n}{c}\PY{n}{\PYZdl{}vec\PYZus{}0}\PY{p}{[}\PY{n}{i\PYZus{}1}\PY{o}{*}\PY{l+m+mh}{1}\PY{o}{+}\PY{o}{:}\PY{l+m+mh}{1}\PY{p}{]}\PY{p}{;}
\PY{+w}{    }\PY{k+kt}{wire}\PY{+w}{  }\PY{n}{map\PYZus{}out\PYZus{}1}\PY{p}{;}
\PY{+w}{    }\PY{k}{assign}\PY{+w}{ }\PY{n}{map\PYZus{}out\PYZus{}1}\PY{+w}{ }\PY{o}{=}\PY{+w}{ }\PY{l+m+mh}{1}\PY{l+m+mb}{\PYZsq{}b0}\PY{p}{;}


\PY{+w}{    }\PY{k}{assign}\PY{+w}{ }\PY{n}{result\PYZus{}res\PYZus{}res}\PY{p}{[}\PY{n}{i\PYZus{}1}\PY{o}{*}\PY{l+m+mh}{1}\PY{o}{+}\PY{o}{:}\PY{l+m+mh}{1}\PY{p}{]}\PY{+w}{ }\PY{o}{=}\PY{+w}{ }\PY{n}{map\PYZus{}out\PYZus{}1}\PY{p}{;}
\PY{+w}{  }\PY{k}{end}
\PY{+w}{  }\PY{k}{endgenerate}
\PY{+w}{  }\PY{c+c1}{// map end}

\PY{+w}{  }\PY{c+c1}{// map begin}
\PY{+w}{  }\PY{k}{genvar}\PY{+w}{ }\PY{n}{i\PYZus{}3}\PY{p}{;}
\PY{+w}{  }\PY{k}{generate}
\PY{+w}{  }\PY{k}{for}\PY{+w}{ }\PY{p}{(}\PY{n}{i\PYZus{}3}\PY{o}{=}\PY{l+m+mh}{0}\PY{p}{;}\PY{+w}{ }\PY{n}{i\PYZus{}3}\PY{+w}{ }\PY{o}{\PYZlt{}}\PY{+w}{ }\PY{l+m+mh}{2}\PY{p}{;}\PY{+w}{ }\PY{n}{i\PYZus{}3}\PY{+w}{ }\PY{o}{=}\PY{+w}{ }\PY{n}{i\PYZus{}3}\PY{+w}{ }\PY{o}{+}\PY{+w}{ }\PY{l+m+mh}{1}\PY{p}{)}\PY{+w}{ }\PY{k}{begin}\PY{+w}{ }\PY{o}{:}\PY{+w}{ }\PY{n}{map\PYZus{}3}
\PY{+w}{    }\PY{k+kt}{wire}\PY{+w}{  }\PY{n}{map\PYZus{}in\PYZus{}3}\PY{p}{;}
\PY{+w}{    }\PY{k}{assign}\PY{+w}{ }\PY{n}{map\PYZus{}in\PYZus{}3}\PY{+w}{ }\PY{o}{=}\PY{+w}{ }\PY{n}{result\PYZus{}res\PYZus{}res}\PY{p}{[}\PY{n}{i\PYZus{}3}\PY{o}{*}\PY{l+m+mh}{1}\PY{o}{+}\PY{o}{:}\PY{l+m+mh}{1}\PY{p}{]}\PY{p}{;}
\PY{+w}{    }\PY{k+kt}{wire}\PY{+w}{ }\PY{p}{[}\PY{l+m+mh}{1}\PY{o}{:}\PY{l+m+mh}{0}\PY{p}{]}\PY{+w}{ }\PY{n}{map\PYZus{}out\PYZus{}3}\PY{p}{;}
\PY{+w}{    }\PY{k+kt}{wire}\PY{+w}{ }\PY{p}{[}\PY{l+m+mh}{1}\PY{o}{:}\PY{l+m+mh}{0}\PY{p}{]}\PY{+w}{ }\PY{n}{c}\PY{n}{\PYZdl{}bb\PYZus{}res\PYZus{}res}\PY{p}{;}
\PY{+w}{    }\PY{k+kt}{wire}\PY{+w}{ }\PY{p}{[}\PY{l+m+mh}{1}\PY{o}{:}\PY{l+m+mh}{0}\PY{p}{]}\PY{+w}{ }\PY{n}{c}\PY{n}{\PYZdl{}vec\PYZus{}1}\PY{p}{;}
\PY{+w}{    }\PY{k}{assign}\PY{+w}{ }\PY{n}{c}\PY{n}{\PYZdl{}vec\PYZus{}1}\PY{+w}{ }\PY{o}{=}\PY{+w}{ }\PY{p}{\PYZob{}}\PY{l+m+mh}{1}\PY{l+m+mb}{\PYZsq{}b1}\PY{p}{,}\PY{+w}{   }\PY{l+m+mh}{1}\PY{l+m+mb}{\PYZsq{}b0}\PY{p}{\PYZcb{}}\PY{p}{;}

\PY{+w}{    }\PY{c+c1}{// map begin}
\PY{+w}{    }\PY{k}{genvar}\PY{+w}{ }\PY{n}{i\PYZus{}2}\PY{p}{;}

\PY{+w}{    }\PY{k}{for}\PY{+w}{ }\PY{p}{(}\PY{n}{i\PYZus{}2}\PY{o}{=}\PY{l+m+mh}{0}\PY{p}{;}\PY{+w}{ }\PY{n}{i\PYZus{}2}\PY{+w}{ }\PY{o}{\PYZlt{}}\PY{+w}{ }\PY{l+m+mh}{2}\PY{p}{;}\PY{+w}{ }\PY{n}{i\PYZus{}2}\PY{+w}{ }\PY{o}{=}\PY{+w}{ }\PY{n}{i\PYZus{}2}\PY{+w}{ }\PY{o}{+}\PY{+w}{ }\PY{l+m+mh}{1}\PY{p}{)}\PY{+w}{ }\PY{k}{begin}\PY{+w}{ }\PY{o}{:}\PY{+w}{ }\PY{n}{map\PYZus{}2}
\PY{+w}{      }\PY{k+kt}{wire}\PY{+w}{  }\PY{n}{map\PYZus{}in\PYZus{}2}\PY{p}{;}
\PY{+w}{      }\PY{k}{assign}\PY{+w}{ }\PY{n}{map\PYZus{}in\PYZus{}2}\PY{+w}{ }\PY{o}{=}\PY{+w}{ }\PY{n}{c}\PY{n}{\PYZdl{}vec\PYZus{}1}\PY{p}{[}\PY{n}{i\PYZus{}2}\PY{o}{*}\PY{l+m+mh}{1}\PY{o}{+}\PY{o}{:}\PY{l+m+mh}{1}\PY{p}{]}\PY{p}{;}
\PY{+w}{      }\PY{k+kt}{wire}\PY{+w}{  }\PY{n}{map\PYZus{}out\PYZus{}2}\PY{p}{;}
\PY{+w}{      }\PY{k}{assign}\PY{+w}{ }\PY{n}{map\PYZus{}out\PYZus{}2}\PY{+w}{ }\PY{o}{=}\PY{+w}{ }\PY{l+m+mh}{1}\PY{l+m+mb}{\PYZsq{}b0}\PY{p}{;}


\PY{+w}{      }\PY{k}{assign}\PY{+w}{ }\PY{n}{c}\PY{n}{\PYZdl{}bb\PYZus{}res\PYZus{}res}\PY{p}{[}\PY{n}{i\PYZus{}2}\PY{o}{*}\PY{l+m+mh}{1}\PY{o}{+}\PY{o}{:}\PY{l+m+mh}{1}\PY{p}{]}\PY{+w}{ }\PY{o}{=}\PY{+w}{ }\PY{n}{map\PYZus{}out\PYZus{}2}\PY{p}{;}
\PY{+w}{    }\PY{k}{end}

\PY{+w}{    }\PY{c+c1}{// map end}

\PY{+w}{    }\PY{k}{assign}\PY{+w}{ }\PY{n}{map\PYZus{}out\PYZus{}3}\PY{+w}{ }\PY{o}{=}\PY{+w}{ }\PY{n}{c}\PY{n}{\PYZdl{}bb\PYZus{}res\PYZus{}res}\PY{p}{;}


\PY{+w}{    }\PY{k}{assign}\PY{+w}{ }\PY{n}{result\PYZus{}res}\PY{p}{[}\PY{n}{i\PYZus{}3}\PY{o}{*}\PY{l+m+mh}{2}\PY{o}{+}\PY{o}{:}\PY{l+m+mh}{2}\PY{p}{]}\PY{+w}{ }\PY{o}{=}\PY{+w}{ }\PY{n}{map\PYZus{}out\PYZus{}3}\PY{p}{;}
\PY{+w}{  }\PY{k}{end}
\PY{+w}{  }\PY{k}{endgenerate}
\PY{+w}{  }\PY{c+c1}{// map end}

\PY{+w}{  }\PY{k}{assign}\PY{+w}{ }\PY{n}{c}\PY{n}{\PYZdl{}xs}\PY{+w}{ }\PY{o}{=}\PY{+w}{ }\PY{n}{c}\PY{n}{\PYZdl{}app\PYZus{}arg\PYZus{}0}\PY{p}{;}

\PY{+w}{  }\PY{c+c1}{// foldr start}
\PY{+w}{  }\PY{k+kt}{wire}\PY{+w}{ }\PY{p}{[}\PY{l+m+mh}{3}\PY{o}{:}\PY{l+m+mh}{0}\PY{p}{]}\PY{+w}{ }\PY{n}{intermediate}\PY{+w}{ }\PY{p}{[}\PY{l+m+mh}{0}\PY{o}{:}\PY{l+m+mh}{1}\PY{p}{]}\PY{p}{;}
\PY{+w}{  }\PY{k}{assign}\PY{+w}{ }\PY{n}{intermediate}\PY{p}{[}\PY{l+m+mh}{1}\PY{p}{]}\PY{+w}{ }\PY{o}{=}\PY{+w}{ }\PY{n}{result\PYZus{}res}\PY{p}{;}

\PY{+w}{  }\PY{k}{genvar}\PY{+w}{ }\PY{n}{i\PYZus{}6}\PY{p}{;}
\PY{+w}{  }\PY{k}{generate}
\PY{+w}{  }\PY{k}{for}\PY{+w}{ }\PY{p}{(}\PY{n}{i\PYZus{}6}\PY{o}{=}\PY{l+m+mh}{0}\PY{p}{;}\PY{+w}{ }\PY{n}{i\PYZus{}6}\PY{+w}{ }\PY{o}{\PYZlt{}}\PY{+w}{ }\PY{l+m+mh}{1}\PY{p}{;}\PY{+w}{ }\PY{n}{i\PYZus{}6}\PY{o}{=}\PY{n}{i\PYZus{}6}\PY{o}{+}\PY{l+m+mh}{1}\PY{p}{)}\PY{+w}{ }\PY{k}{begin}\PY{+w}{ }\PY{o}{:}\PY{+w}{ }\PY{n}{foldr}
\PY{+w}{    }\PY{k+kt}{wire}\PY{+w}{ }\PY{p}{[}\PY{l+m+mh}{3}\PY{o}{:}\PY{l+m+mh}{0}\PY{p}{]}\PY{+w}{ }\PY{n}{foldr\PYZus{}in1}\PY{p}{;}
\PY{+w}{    }\PY{k}{assign}\PY{+w}{ }\PY{n}{foldr\PYZus{}in1}\PY{+w}{ }\PY{o}{=}\PY{+w}{ }\PY{n}{c}\PY{n}{\PYZdl{}xs}\PY{p}{[}\PY{p}{(}\PY{l+m+mh}{1}\PY{o}{\PYZhy{}}\PY{l+m+mh}{1}\PY{o}{\PYZhy{}}\PY{n}{i\PYZus{}6}\PY{p}{)}\PY{o}{*}\PY{l+m+mh}{4}\PY{o}{+}\PY{o}{:}\PY{l+m+mh}{4}\PY{p}{]}\PY{p}{;}
\PY{+w}{    }\PY{k+kt}{wire}\PY{+w}{ }\PY{p}{[}\PY{l+m+mh}{3}\PY{o}{:}\PY{l+m+mh}{0}\PY{p}{]}\PY{+w}{ }\PY{n}{foldr\PYZus{}in2}\PY{p}{;}
\PY{+w}{    }\PY{k+kt}{wire}\PY{+w}{ }\PY{p}{[}\PY{l+m+mh}{3}\PY{o}{:}\PY{l+m+mh}{0}\PY{p}{]}\PY{+w}{ }\PY{n}{foldr\PYZus{}out}\PY{p}{;}

\PY{+w}{    }\PY{k}{assign}\PY{+w}{ }\PY{n}{foldr\PYZus{}in2}\PY{+w}{ }\PY{o}{=}\PY{+w}{ }\PY{n}{intermediate}\PY{p}{[}\PY{n}{i\PYZus{}6}\PY{o}{+}\PY{l+m+mh}{1}\PY{p}{]}\PY{p}{;}
\PY{+w}{    }\PY{c+c1}{// zipWith start}
\PY{+w}{    }\PY{k}{genvar}\PY{+w}{ }\PY{n}{i\PYZus{}5\PYZus{}0}\PY{p}{;}

\PY{+w}{    }\PY{k}{for}\PY{+w}{ }\PY{p}{(}\PY{n}{i\PYZus{}5\PYZus{}0}\PY{+w}{ }\PY{o}{=}\PY{+w}{ }\PY{l+m+mh}{0}\PY{p}{;}\PY{+w}{ }\PY{n}{i\PYZus{}5\PYZus{}0}\PY{+w}{ }\PY{o}{\PYZlt{}}\PY{+w}{ }\PY{l+m+mh}{2}\PY{p}{;}\PY{+w}{ }\PY{n}{i\PYZus{}5\PYZus{}0}\PY{+w}{ }\PY{o}{=}\PY{+w}{ }\PY{n}{i\PYZus{}5\PYZus{}0}\PY{+w}{ }\PY{o}{+}\PY{+w}{ }\PY{l+m+mh}{1}\PY{p}{)}\PY{+w}{ }\PY{k}{begin}\PY{+w}{ }\PY{o}{:}\PY{+w}{ }\PY{n}{zipWith\PYZus{}0\PYZus{}0}
\PY{+w}{      }\PY{k+kt}{wire}\PY{+w}{ }\PY{p}{[}\PY{l+m+mh}{1}\PY{o}{:}\PY{l+m+mh}{0}\PY{p}{]}\PY{+w}{ }\PY{n}{zipWith\PYZus{}in1\PYZus{}0\PYZus{}0}\PY{p}{;}
\PY{+w}{      }\PY{k}{assign}\PY{+w}{ }\PY{n}{zipWith\PYZus{}in1\PYZus{}0\PYZus{}0}\PY{+w}{ }\PY{o}{=}\PY{+w}{ }\PY{n}{foldr\PYZus{}in1}\PY{p}{[}\PY{n}{i\PYZus{}5\PYZus{}0}\PY{o}{*}\PY{l+m+mh}{2}\PY{o}{+}\PY{o}{:}\PY{l+m+mh}{2}\PY{p}{]}\PY{p}{;}
\PY{+w}{      }\PY{k+kt}{wire}\PY{+w}{ }\PY{p}{[}\PY{l+m+mh}{1}\PY{o}{:}\PY{l+m+mh}{0}\PY{p}{]}\PY{+w}{ }\PY{n}{zipWith\PYZus{}in2\PYZus{}0\PYZus{}0}\PY{p}{;}
\PY{+w}{      }\PY{k}{assign}\PY{+w}{ }\PY{n}{zipWith\PYZus{}in2\PYZus{}0\PYZus{}0}\PY{+w}{ }\PY{o}{=}\PY{+w}{ }\PY{n}{foldr\PYZus{}in2}\PY{p}{[}\PY{n}{i\PYZus{}5\PYZus{}0}\PY{o}{*}\PY{l+m+mh}{2}\PY{o}{+}\PY{o}{:}\PY{l+m+mh}{2}\PY{p}{]}\PY{p}{;}
\PY{+w}{      }\PY{k+kt}{wire}\PY{+w}{ }\PY{p}{[}\PY{l+m+mh}{1}\PY{o}{:}\PY{l+m+mh}{0}\PY{p}{]}\PY{+w}{ }\PY{n}{c}\PY{n}{\PYZdl{}n\PYZus{}3}\PY{p}{;}
\PY{+w}{      }\PY{c+c1}{// zipWith start}
\PY{+w}{        }\PY{k}{genvar}\PY{+w}{ }\PY{n}{i\PYZus{}4\PYZus{}1}\PY{p}{;}

\PY{+w}{        }\PY{k}{for}\PY{+w}{ }\PY{p}{(}\PY{n}{i\PYZus{}4\PYZus{}1}\PY{+w}{ }\PY{o}{=}\PY{+w}{ }\PY{l+m+mh}{0}\PY{p}{;}\PY{+w}{ }\PY{n}{i\PYZus{}4\PYZus{}1}\PY{+w}{ }\PY{o}{\PYZlt{}}\PY{+w}{ }\PY{l+m+mh}{2}\PY{p}{;}\PY{+w}{ }\PY{n}{i\PYZus{}4\PYZus{}1}\PY{+w}{ }\PY{o}{=}\PY{+w}{ }\PY{n}{i\PYZus{}4\PYZus{}1}\PY{+w}{ }\PY{o}{+}\PY{+w}{ }\PY{l+m+mh}{1}\PY{p}{)}\PY{+w}{ }\PY{k}{begin}\PY{+w}{ }\PY{o}{:}\PY{+w}{ }\PY{n}{zipWith\PYZus{}3}
\PY{+w}{          }\PY{k+kt}{wire}\PY{+w}{  }\PY{n}{zipWith\PYZus{}in1\PYZus{}3}\PY{p}{;}
\PY{+w}{          }\PY{k}{assign}\PY{+w}{ }\PY{n}{zipWith\PYZus{}in1\PYZus{}3}\PY{+w}{ }\PY{o}{=}\PY{+w}{ }\PY{n}{zipWith\PYZus{}in1\PYZus{}0\PYZus{}0}\PY{p}{[}\PY{n}{i\PYZus{}4\PYZus{}1}\PY{o}{*}\PY{l+m+mh}{1}\PY{o}{+}\PY{o}{:}\PY{l+m+mh}{1}\PY{p}{]}\PY{p}{;}
\PY{+w}{          }\PY{k+kt}{wire}\PY{+w}{  }\PY{n}{zipWith\PYZus{}in2\PYZus{}3}\PY{p}{;}
\PY{+w}{          }\PY{k}{assign}\PY{+w}{ }\PY{n}{zipWith\PYZus{}in2\PYZus{}3}\PY{+w}{ }\PY{o}{=}\PY{+w}{ }\PY{n}{zipWith\PYZus{}in2\PYZus{}0\PYZus{}0}\PY{p}{[}\PY{n}{i\PYZus{}4\PYZus{}1}\PY{o}{*}\PY{l+m+mh}{1}\PY{o}{+}\PY{o}{:}\PY{l+m+mh}{1}\PY{p}{]}\PY{p}{;}
\PY{+w}{          }\PY{k+kt}{wire}\PY{+w}{  }\PY{n}{c}\PY{n}{\PYZdl{}n\PYZus{}4}\PY{p}{;}
\PY{+w}{          }\PY{k}{assign}\PY{+w}{ }\PY{n}{c}\PY{n}{\PYZdl{}n\PYZus{}4}\PY{+w}{ }\PY{o}{=}\PY{+w}{ }\PY{n}{zipWith\PYZus{}in1\PYZus{}3}\PY{+w}{ }\PY{o}{|}\PY{+w}{ }\PY{n}{zipWith\PYZus{}in2\PYZus{}3}\PY{p}{;}


\PY{+w}{          }\PY{k}{assign}\PY{+w}{ }\PY{n}{c}\PY{n}{\PYZdl{}n\PYZus{}3}\PY{p}{[}\PY{n}{i\PYZus{}4\PYZus{}1}\PY{o}{*}\PY{l+m+mh}{1}\PY{o}{+}\PY{o}{:}\PY{l+m+mh}{1}\PY{p}{]}\PY{+w}{ }\PY{o}{=}\PY{+w}{ }\PY{n}{c}\PY{n}{\PYZdl{}n\PYZus{}4}\PY{p}{;}
\PY{+w}{        }\PY{k}{end}

\PY{+w}{        }\PY{c+c1}{// zipWith end}
\PY{+w}{      }\PY{k}{assign}\PY{+w}{ }\PY{n}{foldr\PYZus{}out}\PY{p}{[}\PY{n}{i\PYZus{}5\PYZus{}0}\PY{o}{*}\PY{l+m+mh}{2}\PY{o}{+}\PY{o}{:}\PY{l+m+mh}{2}\PY{p}{]}\PY{+w}{ }\PY{o}{=}\PY{+w}{ }\PY{n}{c}\PY{n}{\PYZdl{}n\PYZus{}3}\PY{p}{;}
\PY{+w}{    }\PY{k}{end}

\PY{+w}{    }\PY{c+c1}{// zipWith end}
\PY{+w}{    }\PY{k}{assign}\PY{+w}{ }\PY{n}{intermediate}\PY{p}{[}\PY{n}{i\PYZus{}6}\PY{p}{]}\PY{+w}{ }\PY{o}{=}\PY{+w}{ }\PY{n}{foldr\PYZus{}out}\PY{p}{;}
\PY{+w}{  }\PY{k}{end}
\PY{+w}{  }\PY{k}{endgenerate}

\PY{+w}{  }\PY{k}{assign}\PY{+w}{ }\PY{n}{result}\PY{+w}{ }\PY{o}{=}\PY{+w}{ }\PY{n}{intermediate}\PY{p}{[}\PY{l+m+mh}{0}\PY{p}{]}\PY{p}{;}
\PY{+w}{  }\PY{c+c1}{// foldr end}

\PY{+w}{  }\PY{c+c1}{// map begin}
\PY{+w}{  }\PY{k}{genvar}\PY{+w}{ }\PY{n}{i\PYZus{}9}\PY{p}{;}
\PY{+w}{  }\PY{k}{generate}
\PY{+w}{  }\PY{k}{for}\PY{+w}{ }\PY{p}{(}\PY{n}{i\PYZus{}9}\PY{o}{=}\PY{l+m+mh}{0}\PY{p}{;}\PY{+w}{ }\PY{n}{i\PYZus{}9}\PY{+w}{ }\PY{o}{\PYZlt{}}\PY{+w}{ }\PY{l+m+mh}{2}\PY{p}{;}\PY{+w}{ }\PY{n}{i\PYZus{}9}\PY{+w}{ }\PY{o}{=}\PY{+w}{ }\PY{n}{i\PYZus{}9}\PY{+w}{ }\PY{o}{+}\PY{+w}{ }\PY{l+m+mh}{1}\PY{p}{)}\PY{+w}{ }\PY{k}{begin}\PY{+w}{ }\PY{o}{:}\PY{+w}{ }\PY{n}{map\PYZus{}4}
\PY{+w}{    }\PY{k+kt}{wire}\PY{+w}{ }\PY{p}{[}\PY{l+m+mh}{1}\PY{o}{:}\PY{l+m+mh}{0}\PY{p}{]}\PY{+w}{ }\PY{n}{map\PYZus{}in\PYZus{}4}\PY{p}{;}
\PY{+w}{    }\PY{k}{assign}\PY{+w}{ }\PY{n}{map\PYZus{}in\PYZus{}4}\PY{+w}{ }\PY{o}{=}\PY{+w}{ }\PY{n}{c}\PY{n}{\PYZdl{}app\PYZus{}arg}\PY{p}{[}\PY{n}{i\PYZus{}9}\PY{o}{*}\PY{l+m+mh}{2}\PY{o}{+}\PY{o}{:}\PY{l+m+mh}{2}\PY{p}{]}\PY{p}{;}
\PY{+w}{    }\PY{k+kt}{wire}\PY{+w}{  }\PY{n}{map\PYZus{}out\PYZus{}4}\PY{p}{;}
\PY{+w}{    }\PY{k+kt}{wire}\PY{+w}{ }\PY{p}{[}\PY{l+m+mh}{1}\PY{o}{:}\PY{l+m+mh}{0}\PY{p}{]}\PY{+w}{ }\PY{n}{c}\PY{n}{\PYZdl{}app\PYZus{}arg\PYZus{}3}\PY{p}{;}
\PY{+w}{    }\PY{k+kt}{wire}\PY{+w}{  }\PY{n}{result\PYZus{}3}\PY{p}{;}
\PY{+w}{    }\PY{k}{assign}\PY{+w}{ }\PY{n}{map\PYZus{}out\PYZus{}4}\PY{+w}{ }\PY{o}{=}\PY{+w}{ }\PY{n}{result\PYZus{}3}\PY{p}{;}

\PY{+w}{    }\PY{c+c1}{// zipWith start}
\PY{+w}{    }\PY{k}{genvar}\PY{+w}{ }\PY{n}{i\PYZus{}7}\PY{p}{;}

\PY{+w}{    }\PY{k}{for}\PY{+w}{ }\PY{p}{(}\PY{n}{i\PYZus{}7}\PY{+w}{ }\PY{o}{=}\PY{+w}{ }\PY{l+m+mh}{0}\PY{p}{;}\PY{+w}{ }\PY{n}{i\PYZus{}7}\PY{+w}{ }\PY{o}{\PYZlt{}}\PY{+w}{ }\PY{l+m+mh}{2}\PY{p}{;}\PY{+w}{ }\PY{n}{i\PYZus{}7}\PY{+w}{ }\PY{o}{=}\PY{+w}{ }\PY{n}{i\PYZus{}7}\PY{+w}{ }\PY{o}{+}\PY{+w}{ }\PY{l+m+mh}{1}\PY{p}{)}\PY{+w}{ }\PY{k}{begin}\PY{+w}{ }\PY{o}{:}\PY{+w}{ }\PY{n}{zipWith\PYZus{}1}
\PY{+w}{      }\PY{k+kt}{wire}\PY{+w}{  }\PY{n}{zipWith\PYZus{}in1\PYZus{}1}\PY{p}{;}
\PY{+w}{      }\PY{k}{assign}\PY{+w}{ }\PY{n}{zipWith\PYZus{}in1\PYZus{}1}\PY{+w}{ }\PY{o}{=}\PY{+w}{ }\PY{n}{c}\PY{n}{\PYZdl{}ds\PYZus{}app\PYZus{}arg}\PY{p}{[}\PY{n}{i\PYZus{}7}\PY{o}{*}\PY{l+m+mh}{1}\PY{o}{+}\PY{o}{:}\PY{l+m+mh}{1}\PY{p}{]}\PY{p}{;}
\PY{+w}{      }\PY{k+kt}{wire}\PY{+w}{  }\PY{n}{zipWith\PYZus{}in2\PYZus{}1}\PY{p}{;}
\PY{+w}{      }\PY{k}{assign}\PY{+w}{ }\PY{n}{zipWith\PYZus{}in2\PYZus{}1}\PY{+w}{ }\PY{o}{=}\PY{+w}{ }\PY{n}{map\PYZus{}in\PYZus{}4}\PY{p}{[}\PY{n}{i\PYZus{}7}\PY{o}{*}\PY{l+m+mh}{1}\PY{o}{+}\PY{o}{:}\PY{l+m+mh}{1}\PY{p}{]}\PY{p}{;}
\PY{+w}{      }\PY{k+kt}{wire}\PY{+w}{  }\PY{n}{c}\PY{n}{\PYZdl{}n\PYZus{}1}\PY{p}{;}
\PY{+w}{      }\PY{k}{assign}\PY{+w}{ }\PY{n}{c}\PY{n}{\PYZdl{}n\PYZus{}1}\PY{+w}{ }\PY{o}{=}\PY{+w}{ }\PY{n}{zipWith\PYZus{}in1\PYZus{}1}\PY{+w}{ }\PY{o}{\PYZam{}}\PY{+w}{ }\PY{n}{zipWith\PYZus{}in2\PYZus{}1}\PY{p}{;}


\PY{+w}{      }\PY{k}{assign}\PY{+w}{ }\PY{n}{c}\PY{n}{\PYZdl{}app\PYZus{}arg\PYZus{}3}\PY{p}{[}\PY{n}{i\PYZus{}7}\PY{o}{*}\PY{l+m+mh}{1}\PY{o}{+}\PY{o}{:}\PY{l+m+mh}{1}\PY{p}{]}\PY{+w}{ }\PY{o}{=}\PY{+w}{ }\PY{n}{c}\PY{n}{\PYZdl{}n\PYZus{}1}\PY{p}{;}
\PY{+w}{    }\PY{k}{end}

\PY{+w}{    }\PY{c+c1}{// zipWith end}

\PY{+w}{    }\PY{c+c1}{// foldr start}
\PY{+w}{    }\PY{k+kt}{wire}\PY{+w}{  }\PY{n}{intermediate\PYZus{}0}\PY{+w}{ }\PY{p}{[}\PY{l+m+mh}{0}\PY{o}{:}\PY{l+m+mh}{2}\PY{p}{]}\PY{p}{;}
\PY{+w}{    }\PY{k}{assign}\PY{+w}{ }\PY{n}{intermediate\PYZus{}0}\PY{p}{[}\PY{l+m+mh}{2}\PY{p}{]}\PY{+w}{ }\PY{o}{=}\PY{+w}{ }\PY{l+m+mh}{1}\PY{l+m+mb}{\PYZsq{}b0}\PY{p}{;}

\PY{+w}{    }\PY{k}{genvar}\PY{+w}{ }\PY{n}{i\PYZus{}8}\PY{p}{;}

\PY{+w}{    }\PY{k}{for}\PY{+w}{ }\PY{p}{(}\PY{n}{i\PYZus{}8}\PY{o}{=}\PY{l+m+mh}{0}\PY{p}{;}\PY{+w}{ }\PY{n}{i\PYZus{}8}\PY{+w}{ }\PY{o}{\PYZlt{}}\PY{+w}{ }\PY{l+m+mh}{2}\PY{p}{;}\PY{+w}{ }\PY{n}{i\PYZus{}8}\PY{o}{=}\PY{n}{i\PYZus{}8}\PY{o}{+}\PY{l+m+mh}{1}\PY{p}{)}\PY{+w}{ }\PY{k}{begin}\PY{+w}{ }\PY{o}{:}\PY{+w}{ }\PY{n}{foldr\PYZus{}0}
\PY{+w}{      }\PY{k+kt}{wire}\PY{+w}{  }\PY{n}{foldr\PYZus{}in1\PYZus{}0}\PY{p}{;}
\PY{+w}{      }\PY{k}{assign}\PY{+w}{ }\PY{n}{foldr\PYZus{}in1\PYZus{}0}\PY{+w}{ }\PY{o}{=}\PY{+w}{ }\PY{n}{c}\PY{n}{\PYZdl{}app\PYZus{}arg\PYZus{}3}\PY{p}{[}\PY{p}{(}\PY{l+m+mh}{2}\PY{o}{\PYZhy{}}\PY{l+m+mh}{1}\PY{o}{\PYZhy{}}\PY{n}{i\PYZus{}8}\PY{p}{)}\PY{o}{*}\PY{l+m+mh}{1}\PY{o}{+}\PY{o}{:}\PY{l+m+mh}{1}\PY{p}{]}\PY{p}{;}
\PY{+w}{      }\PY{k+kt}{wire}\PY{+w}{  }\PY{n}{foldr\PYZus{}in2\PYZus{}0}\PY{p}{;}
\PY{+w}{      }\PY{k+kt}{wire}\PY{+w}{  }\PY{n}{foldr\PYZus{}out\PYZus{}0}\PY{p}{;}

\PY{+w}{      }\PY{k}{assign}\PY{+w}{ }\PY{n}{foldr\PYZus{}in2\PYZus{}0}\PY{+w}{ }\PY{o}{=}\PY{+w}{ }\PY{n}{intermediate\PYZus{}0}\PY{p}{[}\PY{n}{i\PYZus{}8}\PY{o}{+}\PY{l+m+mh}{1}\PY{p}{]}\PY{p}{;}
\PY{+w}{      }\PY{k}{assign}\PY{+w}{ }\PY{n}{foldr\PYZus{}out\PYZus{}0}\PY{+w}{ }\PY{o}{=}\PY{+w}{ }\PY{n}{foldr\PYZus{}in1\PYZus{}0}\PY{+w}{ }\PY{o}{|}\PY{+w}{ }\PY{n}{foldr\PYZus{}in2\PYZus{}0}\PY{p}{;}


\PY{+w}{      }\PY{k}{assign}\PY{+w}{ }\PY{n}{intermediate\PYZus{}0}\PY{p}{[}\PY{n}{i\PYZus{}8}\PY{p}{]}\PY{+w}{ }\PY{o}{=}\PY{+w}{ }\PY{n}{foldr\PYZus{}out\PYZus{}0}\PY{p}{;}
\PY{+w}{    }\PY{k}{end}


\PY{+w}{    }\PY{k}{assign}\PY{+w}{ }\PY{n}{result\PYZus{}3}\PY{+w}{ }\PY{o}{=}\PY{+w}{ }\PY{n}{intermediate\PYZus{}0}\PY{p}{[}\PY{l+m+mh}{0}\PY{p}{]}\PY{p}{;}
\PY{+w}{    }\PY{c+c1}{// foldr end}


\PY{+w}{    }\PY{k}{assign}\PY{+w}{ }\PY{n}{result\PYZus{}0}\PY{p}{[}\PY{n}{i\PYZus{}9}\PY{o}{*}\PY{l+m+mh}{1}\PY{o}{+}\PY{o}{:}\PY{l+m+mh}{1}\PY{p}{]}\PY{+w}{ }\PY{o}{=}\PY{+w}{ }\PY{n}{map\PYZus{}out\PYZus{}4}\PY{p}{;}
\PY{+w}{  }\PY{k}{end}
\PY{+w}{  }\PY{k}{endgenerate}
\PY{+w}{  }\PY{c+c1}{// map end}

\PY{+w}{  }\PY{k}{assign}\PY{+w}{ }\PY{n}{result\PYZus{}1}\PY{+w}{ }\PY{o}{=}\PY{+w}{ }\PY{p}{\PYZob{}}\PY{n}{result\PYZus{}0}\PY{p}{,}\PY{+w}{   }\PY{n}{c}\PY{n}{\PYZdl{}app\PYZus{}arg\PYZus{}1}\PY{p}{\PYZcb{}}\PY{p}{;}

\PY{+w}{  }\PY{c+c1}{// foldr start}
\PY{+w}{  }\PY{k+kt}{wire}\PY{+w}{  }\PY{n}{intermediate\PYZus{}1}\PY{+w}{ }\PY{p}{[}\PY{l+m+mh}{0}\PY{o}{:}\PY{l+m+mh}{2}\PY{p}{]}\PY{p}{;}
\PY{+w}{  }\PY{k}{assign}\PY{+w}{ }\PY{n}{intermediate\PYZus{}1}\PY{p}{[}\PY{l+m+mh}{2}\PY{p}{]}\PY{+w}{ }\PY{o}{=}\PY{+w}{ }\PY{l+m+mh}{1}\PY{l+m+mb}{\PYZsq{}b0}\PY{p}{;}

\PY{+w}{  }\PY{k}{genvar}\PY{+w}{ }\PY{n}{i\PYZus{}10}\PY{p}{;}
\PY{+w}{  }\PY{k}{generate}
\PY{+w}{  }\PY{k}{for}\PY{+w}{ }\PY{p}{(}\PY{n}{i\PYZus{}10}\PY{o}{=}\PY{l+m+mh}{0}\PY{p}{;}\PY{+w}{ }\PY{n}{i\PYZus{}10}\PY{+w}{ }\PY{o}{\PYZlt{}}\PY{+w}{ }\PY{l+m+mh}{2}\PY{p}{;}\PY{+w}{ }\PY{n}{i\PYZus{}10}\PY{o}{=}\PY{n}{i\PYZus{}10}\PY{o}{+}\PY{l+m+mh}{1}\PY{p}{)}\PY{+w}{ }\PY{k}{begin}\PY{+w}{ }\PY{o}{:}\PY{+w}{ }\PY{n}{foldr\PYZus{}1}
\PY{+w}{    }\PY{k+kt}{wire}\PY{+w}{  }\PY{n}{foldr\PYZus{}in1\PYZus{}1}\PY{p}{;}
\PY{+w}{    }\PY{k}{assign}\PY{+w}{ }\PY{n}{foldr\PYZus{}in1\PYZus{}1}\PY{+w}{ }\PY{o}{=}\PY{+w}{ }\PY{n}{c}\PY{n}{\PYZdl{}app\PYZus{}arg\PYZus{}2}\PY{p}{[}\PY{p}{(}\PY{l+m+mh}{2}\PY{o}{\PYZhy{}}\PY{l+m+mh}{1}\PY{o}{\PYZhy{}}\PY{n}{i\PYZus{}10}\PY{p}{)}\PY{o}{*}\PY{l+m+mh}{1}\PY{o}{+}\PY{o}{:}\PY{l+m+mh}{1}\PY{p}{]}\PY{p}{;}
\PY{+w}{    }\PY{k+kt}{wire}\PY{+w}{  }\PY{n}{foldr\PYZus{}in2\PYZus{}1}\PY{p}{;}
\PY{+w}{    }\PY{k+kt}{wire}\PY{+w}{  }\PY{n}{foldr\PYZus{}out\PYZus{}1}\PY{p}{;}

\PY{+w}{    }\PY{k}{assign}\PY{+w}{ }\PY{n}{foldr\PYZus{}in2\PYZus{}1}\PY{+w}{ }\PY{o}{=}\PY{+w}{ }\PY{n}{intermediate\PYZus{}1}\PY{p}{[}\PY{n}{i\PYZus{}10}\PY{o}{+}\PY{l+m+mh}{1}\PY{p}{]}\PY{p}{;}
\PY{+w}{    }\PY{k}{assign}\PY{+w}{ }\PY{n}{foldr\PYZus{}out\PYZus{}1}\PY{+w}{ }\PY{o}{=}\PY{+w}{ }\PY{n}{foldr\PYZus{}in1\PYZus{}1}\PY{+w}{ }\PY{o}{|}\PY{+w}{ }\PY{n}{foldr\PYZus{}in2\PYZus{}1}\PY{p}{;}


\PY{+w}{    }\PY{k}{assign}\PY{+w}{ }\PY{n}{intermediate\PYZus{}1}\PY{p}{[}\PY{n}{i\PYZus{}10}\PY{p}{]}\PY{+w}{ }\PY{o}{=}\PY{+w}{ }\PY{n}{foldr\PYZus{}out\PYZus{}1}\PY{p}{;}
\PY{+w}{  }\PY{k}{end}
\PY{+w}{  }\PY{k}{endgenerate}

\PY{+w}{  }\PY{k}{assign}\PY{+w}{ }\PY{n}{c}\PY{n}{\PYZdl{}app\PYZus{}arg\PYZus{}1}\PY{+w}{ }\PY{o}{=}\PY{+w}{ }\PY{n}{intermediate\PYZus{}1}\PY{p}{[}\PY{l+m+mh}{0}\PY{p}{]}\PY{p}{;}
\PY{+w}{  }\PY{c+c1}{// foldr end}

\PY{+w}{  }\PY{k}{assign}\PY{+w}{ }\PY{n}{c}\PY{n}{\PYZdl{}vec2}\PY{+w}{ }\PY{o}{=}\PY{+w}{ }\PY{p}{\PYZob{}}\PY{l+m+mh}{1}\PY{l+m+mb}{\PYZsq{}b0}\PY{p}{,}\PY{+w}{   }\PY{l+m+mh}{1}\PY{l+m+mb}{\PYZsq{}b1}\PY{p}{\PYZcb{}}\PY{p}{;}

\PY{+w}{  }\PY{c+c1}{// zipWith start}
\PY{+w}{  }\PY{k}{genvar}\PY{+w}{ }\PY{n}{i\PYZus{}11}\PY{p}{;}
\PY{+w}{  }\PY{k}{generate}
\PY{+w}{  }\PY{k}{for}\PY{+w}{ }\PY{p}{(}\PY{n}{i\PYZus{}11}\PY{+w}{ }\PY{o}{=}\PY{+w}{ }\PY{l+m+mh}{0}\PY{p}{;}\PY{+w}{ }\PY{n}{i\PYZus{}11}\PY{+w}{ }\PY{o}{\PYZlt{}}\PY{+w}{ }\PY{l+m+mh}{2}\PY{p}{;}\PY{+w}{ }\PY{n}{i\PYZus{}11}\PY{+w}{ }\PY{o}{=}\PY{+w}{ }\PY{n}{i\PYZus{}11}\PY{+w}{ }\PY{o}{+}\PY{+w}{ }\PY{l+m+mh}{1}\PY{p}{)}\PY{+w}{ }\PY{k}{begin}\PY{+w}{ }\PY{o}{:}\PY{+w}{ }\PY{n}{zipWith\PYZus{}2}
\PY{+w}{    }\PY{k+kt}{wire}\PY{+w}{  }\PY{n}{zipWith\PYZus{}in1\PYZus{}2}\PY{p}{;}
\PY{+w}{    }\PY{k}{assign}\PY{+w}{ }\PY{n}{zipWith\PYZus{}in1\PYZus{}2}\PY{+w}{ }\PY{o}{=}\PY{+w}{ }\PY{n}{result\PYZus{}0}\PY{p}{[}\PY{n}{i\PYZus{}11}\PY{o}{*}\PY{l+m+mh}{1}\PY{o}{+}\PY{o}{:}\PY{l+m+mh}{1}\PY{p}{]}\PY{p}{;}
\PY{+w}{    }\PY{k+kt}{wire}\PY{+w}{  }\PY{n}{zipWith\PYZus{}in2\PYZus{}2}\PY{p}{;}
\PY{+w}{    }\PY{k}{assign}\PY{+w}{ }\PY{n}{zipWith\PYZus{}in2\PYZus{}2}\PY{+w}{ }\PY{o}{=}\PY{+w}{ }\PY{n}{c}\PY{n}{\PYZdl{}vec2}\PY{p}{[}\PY{n}{i\PYZus{}11}\PY{o}{*}\PY{l+m+mh}{1}\PY{o}{+}\PY{o}{:}\PY{l+m+mh}{1}\PY{p}{]}\PY{p}{;}
\PY{+w}{    }\PY{k+kt}{wire}\PY{+w}{  }\PY{n}{c}\PY{n}{\PYZdl{}n\PYZus{}2}\PY{p}{;}
\PY{+w}{    }\PY{k}{assign}\PY{+w}{ }\PY{n}{c}\PY{n}{\PYZdl{}n\PYZus{}2}\PY{+w}{ }\PY{o}{=}\PY{+w}{ }\PY{n}{zipWith\PYZus{}in1\PYZus{}2}\PY{+w}{ }\PY{o}{\PYZam{}}\PY{+w}{ }\PY{n}{zipWith\PYZus{}in2\PYZus{}2}\PY{p}{;}


\PY{+w}{    }\PY{k}{assign}\PY{+w}{ }\PY{n}{c}\PY{n}{\PYZdl{}app\PYZus{}arg\PYZus{}2}\PY{p}{[}\PY{n}{i\PYZus{}11}\PY{o}{*}\PY{l+m+mh}{1}\PY{o}{+}\PY{o}{:}\PY{l+m+mh}{1}\PY{p}{]}\PY{+w}{ }\PY{o}{=}\PY{+w}{ }\PY{n}{c}\PY{n}{\PYZdl{}n\PYZus{}2}\PY{p}{;}
\PY{+w}{  }\PY{k}{end}
\PY{+w}{  }\PY{k}{endgenerate}
\PY{+w}{  }\PY{c+c1}{// zipWith end}


\PY{k}{endmodule}
\end{Verbatim}
}

