\chapter{Results} \label{ch:outcome}
In this chapter, we describe the experiments that we performed to test
our workflow and the resulting observations.
We found that our flow produces designs that are of practical use.
Since the \gls{rgx} we consider is more expressive than the classic
regex, we can use specifications similar to the \glspl{SERE} of
\gls{PSL}.
% Though the workflow itself uses only open source software to allow for
% customizations in the future, we also tried proprietary synthesis
% tool. The open source tool chain was less resource intensive.

\section{Experiment set-up}
In order to gauge the results of our workflow, we performed
experiments by varying the size of the input \gls{rgx}.
These experiments were performed on an AMD Ryzen-7 400 series
processor with 32GB RAM equipped with an SSD for secondary
% processor with 32GB \gls{RAM} equipped with a \gls{SSD} for secondary
storage.
We used the workflow to produce matchers for \glspl{rgx} of different
sizes.
The input type of these \glspl{rgx} was \code{re (Char -> Bool)}
(Table~\ref{tab:sample-taa}).
The size of these \glspl{rgx} ranged from 1 to 61.
The available amount of RAM was insufficient for \glspl{rgx}
%% of
% The available amount of \gls{RAM} was insufficient for \glspl{rgx} of
%% input type \code{Char}
with size greater than 61.
We also tried the flow with a proprietary synthesis tool (Xilinx
Vivado) on the same computer, but it was more memory intensive when
compared to open source tools and could handle only \gls{rgx} size
up to 41.
The statistics that we report in this chapter are made using the
\gls{FOSS} tool-chain described in Section~\ref{sec:eda}.
Since a formal correctness proof has not yet been built, we used
property-based testing to ensure the correctness of the generated
Clash designs.
Verilog designs generated from these designs were further
simulated via Verilator testbenches~\cite{snyder2010verilator}.

\begin{table}
\centering
\begin{tabular}{lc}
  \toprule
  \gls{rgx}            & Size \\
  \midrule
  \code{"a"}           & 1    \\
  \code{"a"; "b"}      & 2    \\
  \code{"a"; "b"; "c"} & 3    \\
  \bottomrule
\end{tabular}
\caption{Sample test regexes used}  
\label{tab:sample-taa}  
\end{table}

% We primarily used the open source tools yosys and nextpnr to perform
% \gls{EDA}.
% In addition to our workflow using open source tools, we also tried the
% same experiments with the free version of the proprietary Xilinx
% Vivado tool for synthesis.
% We found that the open source toolchain required less computer
% resources than Vivado.

\section{Size and sparsity of transition matrix}
The transition information of the \gls{rgx} matcher automata is a 3D
Boolean matrix $\delta$.
The rows of this transition matrix are associated with the entries in
the function table $ft$.
Any given row of $\delta$ is a 2D sub-matrix containing the
transition information to be used when a particular subset of tests in the input
regular expression is activated.
These sub-matrices are square matrices with size $2|s|$ where $|s|$ is the
number of \gls{NFA}-states.
%
This is because 
each $Atom$ in the input \gls{rgx} $r$ leads to a pair of states in the
NFA model of the matcher, in addition to an entry in the function
table $ft$.
% 4t³ = t * 2t * 2t
Therefore, $\delta$ as a whole would have $O(|r|^3)$ elements.
%% As a result, size of $\delta$ grows exponentially with respect to the
%% size of the input \gls{rgx} (Fig.~\ref{graph:loc-design}).
Fig.~\ref{graph:loc-design} shows the growth in the size of the
matcher Clash design with respect to increase in input \gls{rgx} size.
% Resultant transition matrix would have $\mathcal{O}(n^3)$ elements,

Most states of the \gls{NFA}
%% The structure of the \gls{NFA} 
that we build using Thompson's \gls{NFA} construction algorithm do not
have any transition between them.
Hence, $\delta$ ends up being a sparse matrix.
%% \gls{NFA} construction algorithm is a sparse graph,
%% which makes $\delta$ a sparse matrix.
This is the reason why $\delta$ is huge for larger \glspl{rgx} even
though the information it contains can be stored using less space.
For example, a \gls{rgx} of size 60 led to almost 160K lines of
Haskell code.
Most of this is the transition matrix.
Fig.~\ref{graph:loc-design} shows the increase in the size of the
extracted Haskell code and the resulting Verilog, as the size of the
input \gls{rgx} increases.
We chose not to use a sparse matrix for our implementation as proofs
involving sparse matrices are considerably more complex.

\begin{figure}
  \begin{center}
    \begin{tikzpicture}
\begin{axis}[
  title=LOC of designs,
  xlabel={Regex size},
  ylabel={LOC},
  legend pos=north west,
  legend entries={Haskell}
  %legend entries={Haskell, Verilog}
]
  \addplot table {chapters/data/loc-data/loc-clash.dat};
  %\addplot table {chapters/data/loc-data/loc-verilog.dat};
\end{axis}
\end{tikzpicture}

    \caption{Lines of code in generated designs}
    \label{graph:loc-design}
  \end{center}
\end{figure}

We also study the time required for the overall flow to run.
The total time taken to produce the matcher bitstream from the input
regular expression appears to grow exponentially with respect to the
size of the regular expression (Fig.~\ref{graph:time-taken}).
%% Exponential growth in the size of the transition matrix $\delta$ with
%% respect to \gls{rgx} size means that there is a similar growth in the
%% time taken to produce the hardware matchers
%% %
%% The time taken by each phase of our flow, ranging from extraction to
%% synthesis, grows exponentially with increase in size of input
%% \gls{rgx}.
Among the different phases of our workflow, the Haskell-to-Verilog
conversion takes the longest amount of time, followed by the circuit
synthesis process (Fig.~\ref{graph:time-hdl-synth}).
The Haskell code that the Clash compiler accepts as input is written
in terms of functions.
These functions are processed until their applications can be reduced
down into a form which can directly be mapped to constructs available
in an \gls{HDL} like Verilog.
Reduction of functions that accept the transition matrix $\delta$
will take more time to process as $\delta$ gets larger.
%% This is especially tru
This is likely the reason for the longer time taken for the generation
of Verilog descriptions.
% TODO: Mention that clash doesn't preform any optimization
%
%
The synthesis process is time consuming, as it involves optimizations
which are resource intensive.
%% The peak RAM utilization of the synthesis process is shown in
%% Fig.~\ref{graph:yosys-ram}.
%% %
%% The largest \gls{rgx} that we could try without crashing the synthesis
%% tool was of size 61 whose peak memory utilization was 28.83 GB
%% during the synthesis process (Fig.~\ref{graph:yosys-ram}).
%
The time taken by the other phases of the workflow, such as the Coq extraction
process and the Clash project builds, also grow exponentially with respect
to the input \gls{rgx} size, but they take much lesser time than the
Verilog generation and synthesis phases.
% TODO: Time taken by nextpnr


%% Exponential growth in the size of the transition matrix $\delta$ with
%% respect to increase in \gls{rgx} size means that there is a similar
%% growth in the time taken to generate the Verilog description and the
%% circuit synthesis process itself (Fig.~\ref{graph:time-taken}).
%% %% For a \gls{rgx} of size 61, the time taken for the overall flow was
%% %% over three hours.
%% %
%% Among the 

As in the case of time taken, the memory utilized by the synthesis
tool also grows exponentially with respect to the size of the input
\gls{rgx}.
The largest \gls{rgx} we could try without crashing the synthesis tool
was of size 61 whose maximum memory utilization was 28.83 GB during
the synthesis process (Fig.~\ref{graph:yosys-ram}).

\begin{figure}
  \begin{center}
    \begin{tikzpicture}
\begin{axis}[
  title=RAM utilization,
  xlabel={Regex size},
  ylabel={RAM},
  legend pos=north west,
  legend entries={RAM}
]
  \addplot table {chapters/data/yosys/mem-yosys.dat};
\end{axis}
\end{tikzpicture}

    \caption{Peak RAM utilization for synthesis}
    \label{graph:yosys-ram}
  \end{center}
\end{figure}

%% Construction and conversion to Clash-compatible Haskell of the
%% transition matrix is quite fast, but compilation from Clash to Verilog
%% can be quite slow.
%% % This is because Clash implicitly performs rewrites for converting
%% % Haskell to HDL
%% This is because Clash does not try to minimize the sparseness of the
%% matrix and tries to faithfully convert it to Verilog as it is.
%% However, the synthesis tool is able to optimize the design so well that
%% the resource utilization of the implemented design is very low.

%% TODO: Put this in future work
%% It is possible that using sparse matrices and associated operations
%% can speed up the Clash to Verilog conversion.
%% Clash does not have a sparse matrix representation and we would have
%% to make our own computation-friendly definitions, which can be
%% converted to vectors.
%% We leave this as a possible path of future exploration.

\begin{figure}
\begin{subfigure}{1.0\linewidth}
  \begin{center}
    % \begin{tikzpicture}
% \begin{axis}[
%   title=Time taken: Build \& extraction,
%   xlabel={Regex size},
%   ylabel={Seconds},
%   % width=0.5\textwidth,
%   legend pos=north west,
%   legend entries={%
%     Clash project build,%
%     Haskell extraction,%
%     LOC in Clash design}
% ]
%   \addplot table {chapters/data/time-data/time-build.dat};
%   \addplot table {chapters/data/time-data/time-extr.dat};
%   \addplot table {chapters/data/time-data/loc.dat};
% \end{axis}
% \end{tikzpicture}


\begin{tikzpicture}
\begin{axis}[
  % width=6cm,
  % width=0.5\textwidth,
  % title=Time taken: Build \& extraction,
  xlabel={Regex size},
  ylabel={Seconds},
  % width=0.5\textwidth,
  legend pos=north west,
  legend entries={Clash project build, Haskell extraction}
]
  \addplot table {chapters/data/time-data/time-build.dat};
  \addplot table {chapters/data/time-data/time-extr.dat};
\end{axis}
\end{tikzpicture}

    \caption{Extraction and Cabal build}
    \label{graph:time-build-extr}
  \end{center}
\end{subfigure}

\begin{subfigure}{1.0\linewidth}
  \begin{center}
    \begin{tikzpicture}
\begin{axis}[
  % width=6cm,
  % title=Time taken: Verilog generation \& synthesis,
  xlabel={Regex size},
  ylabel={Seconds},
  % width=0.5\textwidth,
  legend pos=north west,
  legend entries={Verilog generation, Synthesis}
]
  \addplot table {chapters/data/time-data/time-hdl.dat};
  \addplot table {chapters/data/time-data/time-synth.dat};
\end{axis}
\end{tikzpicture}

    \caption{Verilog generation and synthesis}
    \label{graph:time-hdl-synth}
  \end{center}
\end{subfigure}

\begin{subfigure}{1.0\linewidth}
  \begin{center}
    %   \begin{tikzpicture}
    \begin{axis}[
      title=Total time taken,
      xlabel={Regex size},
      ylabel={Seconds},
      legend pos=north west,
    ]
      \addplot table {chapters/data/time-data/time-total.dat};
    \end{axis}
  \end{tikzpicture}

    % % P103 of revision 2021-05-15

% \documentclass{article}
% \usepackage{pgfplots}
% \pgfplotsset{compat=1.5}
% \begin{document}

\begin{tikzpicture}
\begin{axis}[
  ymin=0,
  minor tick num=4,
  enlarge x limits=false,
  const plot,
  axis on top,
  stack plots=y,
  % every axis plot/.append style={fill,fill opacity=0.5},
  cycle list={
  % {blue!70!black,fill=blue},
  % {blue!60!white,fill=blue!30!white},
  {draw=none,fill={rgb:red,138;green,82;blue,232}},
  % {red,thick}
  },
  xlabel={Regex size},
  ylabel={Time},
    legend pos=north west,
  legend style={
  area legend,
  % at={(0.5,-0.15)},
  % anchor=north,
  legend columns=1},
  % legend entries={Extraction, Build, HDL gen, Synthesis},
] 
  \addplot [draw=blue,fill=gray!30!white]
     table [x=size,y=extr]
     from {chapters/data/time-data/time-collection.dat}
    \closedcycle;
  \addplot [draw=blue,fill=teal!30!white]
     table [x=size,y=build]
     from {chapters/data/time-data/time-collection.dat}
    \closedcycle;
  \addplot [draw=blue,fill=blue!30!white]
     table [x=size,y=hdl]
     from {chapters/data/time-data/time-collection.dat}
    \closedcycle;
  \addplot [draw=blue,fill=orange!30!white]
     table [x=size,y=synth]
     from {chapters/data/time-data/time-collection.dat}
    \closedcycle;
  \addplot+ [stack plots=false] %, draw=blue,fill=blue!30!white]
     table [x=size,y=total]
     from {chapters/data/time-data/time-collection.dat}
    \closedcycle;
  \legend{Extraction,Build,HDL gen,Synthesis}

\end{axis}
\end{tikzpicture}

% \end{document}

  \caption{Total time}
  \label{graph:time-taken}
  \end{center}
\end{subfigure}
  \caption{Time taken}
  \label{graph:time}
\end{figure}

%% % TODO: Use this paragraph blocks
%% The extraction process that converts Coq values to Haskell performs a
%% one-to-one mapping from Coq values to Haskell values.
%% Likewise, the Clash compiler simply converts the Haskell-like code to
%% equivalent Verilog without attempting any optimizations of the kind
%% performed by synthesis tools.
%% Clash designs corresponding to matchers of larger \glspl{rgx} also
%% cause the process of building the Clash project to take exponentially
%% larger amount of time~(Fig. \ref{graph:time-build-extr}).
%% %
%% Due to the large size of the Clash design caused by the high degree
%% of sparsity of the transition matrix, generation of Verilog code by
%% the Clash compiler and the synthesis task performed by yosys to
%% produce a netlist are the most time consuming phases of our
%% flow~(Fig. \ref{graph:time-hdl-synth}),
%% out of which the time taken for Verilog generation seems to grow
%% faster than the time taken by the synthesis process.
%% %
%% The transition matrix consists of large vectors, which in many cases
%% have only a single set value.
%% This gives scope for optimizations during the synthesis phase that can
%% considerably bring down the size of the final design.
%% % Clash compiler simply tries to convert the Haskell-like code to
%% % equivalent Verilog without attempting any optimizations of the kind
%% % performed by synthesis tools.
%% The synthesis tool accepts the Verilog code generated by the Clash
%% compiler and performs optimizations that consume both time and
%% computer memory.

%% %% As in the case of time taken, memory utilized by the synthesis tool
%% %% also grows exponentially with respect to the size of the input
%% %% \gls{rgx}.
%% %% The largest \gls{rgx} that we could try without crashing the synthesis
%% %% tool was of size 61 whose maximum memory utilization was 28.83 GB
%% %% during the synthesis process (Fig.~\ref{graph:yosys-ram}).

\section{Hardware resource utilization}
%% We used a Gowin LittleBee \gls{FPGA} of the LittleBee family
%% developed by Gowin Semiconductor GW1N-9C (GW1NR-LV9QN88PC6/I5) made by Gowin
%% Semiconductor for trying out our designs.

% Device: GW1NR-LV9QN88PC6/I5
% Family: GW1N-9C

We used a Sipeed Tang Nano 9K board (Table~\ref{tab:sipeed-resources})
for test our designs.
This board contains a LittleBee \gls{FPGA} made by Gowin Semiconductor
(device: GW1NR-LV9QN88PC6/I5, family: GW1N-9C).

% We used a LittleBee \gls{FPGA} developed by Gowin Semiconductor for
% trying out our designs (device: GW1NR-LV9QN88PC6/I5, family: GW1N-9C).
%% Some of the resources offered by this \gls{FPGA} are s
Among the resources available in this \gls{FPGA}, the only resources
used by the designs generated by our workflow that varied with the
input \gls{rgx} size were the \gls{LUT} count and the number of
registers used.
%% corresponding change in the number of slices utilized.
%
%% We used an \gls{FPGA} GW1N-9C (GW1NR-LV9QN88PC6/I5) made by Gowin
%% Semiconductor for trying out our designs.
%% Among the resources available in this \gls{FPGA}
%% (Table~\ref{tab:fpga-resources}), the only resources used by the
%% designs generated by our workflow that varied with input \gls{rgx}
%% size were 5 and 6 input \glspl{LUT} made by combining the LUT4
%% elements available in the \gls{FPGA}, which by extension also meant
%% number of slices utilized.
This makes sense as our designs are simple.
Since the 
\glspl{LUT} store transition information and the relation between the
size of the transition matrix and the input \gls{rgx} size is linear,
the number of \glspl{LUT} and slices utilized grows almost linearly
with respect to the input \gls{rgx} size
(Fig.~\ref{graph:utilization}).
% As expected, the number of \glspl{LUT} and slices utilized grows almost linearly
% with respect to input \gls{rgx} size, which makes sense as the
% \glspl{LUT} store transition information and the relation between size
% of the transition matrix and the input \gls{rgx} size is
% linear~\ref{graph:utilization}.
The occasional drops in the \gls{LUT} utilization graph are likely
due to the synthesis tool being able to find better courses of
optimizations in certain cases.
%
Although the transition matrix $\delta$ of the \glspl{NFA} can be
quite large, its sparseness means that the information it holds can be
represented using much lesser space in hardware.
%
The synthesis tool performs optimizations to produce designs whose
resource utilization is very small even for large $\delta$ values.

\begin{figure}
  \begin{center}
      \begin{tikzpicture}
    \begin{axis}[
      title=LUT and Slice count,
      xlabel={Regex size},
      ylabel={Count},
      % width=0.5\textwidth,
      legend pos=north west,
      legend entries={LUT5,LUT6,Slice}
    ]
      \addplot table {chapters/data/util-data/util-lut5.dat};
      \addplot table {chapters/data/util-data/util-lut6.dat};
      \addplot table {chapters/data/util-data/util-slice.dat};
    \end{axis}
  \end{tikzpicture}

  \caption{Resource utilization}
  \label{graph:utilization}
  \end{center}
\end{figure}


% \begin{table}
%   \begin{center}
%   \begin{tabular}{ll}
%     \toprule
% Resource                     & Count \\
%     \midrule
% LUTs (LUT4)                  & 8640  \\
% Registers / FFs              & 6480  \\
% PLLs                         & 2     \\
% Buttons                      & 2     \\
% LEDs                         & 6     \\
% % Crystal oscillator frequency & 27MHz \\
%     \bottomrule
%   \end{tabular}
%   \end{center}
% \caption{Resources available in a GW1NR-LV9QN88PC6/I5 \gls{FPGA}}
% \label{tab:fpga-resources}
% \end{table}

\begin{table}
  % \centering
  \begin{minipage}{0.5\linewidth}
  \begin{tabular}{ll}
\toprule
    Resource           & Count          \\
\midrule
    LUTs (LUT4)        & 8640           \\
    Registers / FFs    & 6480           \\
    PLLs               & 2              \\
    Buttons            & 2              \\
    LEDs               & 6              \\
    Buttons            & 2              \\ 
    Block SRAM         & 26             \\
    18 x 18 multiplier & 20             \\
\bottomrule
\end{tabular}
\end{minipage}
\hfill
  \begin{minipage}{0.5\linewidth}
  \begin{tabular}{ll}
\toprule
    Resource           & Size (bits)    \\
\midrule
    Block SRAM         & 468K           \\
    Shadow SRAM        & 17280          \\
    SDRAM size         & 64M            \\
    SPI flash          & 32M            \\
    User flash         & 608K           \\
\bottomrule
  \end{tabular}
  \begin{tabular}{ll}
\toprule
    Resource           & Type           \\
\midrule
    Display interface  & HDMI, SPI, RGB \\
    Debugger           & JTAG, UART     \\
    % IO               &                \\
    % Connector        &                \\
\bottomrule
  \end{tabular}
  \end{minipage}
  \caption{Resources available in a Sipeed Tang Nano 9K board}
  \label{tab:sipeed-resources}
\end{table}


The utilization of other \gls{FPGA} resources is constant regardless
of the size of the input \gls{rgx}.
%
% 25 IOB
All designs have a constant number of input and output signals.
Each of them has 4 inputs (clock, enable, reset and input character)
and a single Boolean result as output.
Likewise, each design needs only one each of ground and $V_{cc}$ pins.
No other \gls{FPGA} resources were used.



%% \section{Trusted code base}
% \section{Correctness}
% Our formalization of \glspl{NFA} is made in a form that can be
% extracted and converted to synthesizable hardware designs.
% Building such extractable designs while also having ease of
% correctness proofs is tricky to get right.
% %
% Ideally, we should find a sweet spot between extractability and ease
% of proofs.
% %
% As mentioned earlier,
% most of the existing formalizations of finite sets are optimized for
% proofs rather than extractability.
% %% like the \code{finset} type of mathcomp
% %
% For example, we were able to define and prove correctness of
% \gls{rgx}-\gls{NFA} conversions using the \code{finset} type from
% mathcomp but the \glspl{NFA} so constructed do not extract in a
% convenient way.

% On the other hand, types like \code{Vector.t}, while being
% computation-friendly and extractable, are so notoriously
% unwieldy that it is often difficult to define functions using them
% \footnote{\url{https://github.com/coq/coq/pull/18032}}.
% %
% Since manipulating \code{Vector.t} values is well known to be difficult,
% %\footnote{\url{https://github.com/coq/coq/pull/18032}},
% we chose to define \glspl{NFA} using the \code{func} type.
% But the \glspl{NFA} correctness proofs involving \code{func} turned
% out to be difficult as well due to the latter's tree structure.
% Due to this reason, we could not complete the correctness proof of the
% Coq procedure converting \glspl{rgx} to \glspl{NFA}.


% % While using \code{func} gave us the flexibility to build
% % extractable \gls{NFA}, proofs involving them are still tricky.
% % One reason for that is that is the tree structure of \code{func}.
% % We were not able to complete the correctness proof of our \gls{NFA}
% % generation procedure.

% Despite the advantages offered by \code{func}, its flexibility
% % while making disjoint union when compared to \code{Vector.t}
% comes at a cost.
% Values of type $s \vto T$ are structured as trees where each node is a
% constructor. The shape of this tree is indicated by the \code{sort} value $s$,
% where the elements will always be at a leaf position.
% %% A given finite set can be represented in 
% % A A given $s \vto T$ type can have several different possible values inhabiting
% % it.
% % Whereas a given \code{Vector.t} value would have 
% This means that case analysis on a given \code{func} value would have
% %% to deal with more than one vector structure unlike for \code{Vector.t}
% to deal with more than one structure unlike for \code{Vector.t}
% which can be split only in one way due to its linear structure.
% A few possible structures for a vector \code{[a, b, c]} encoded as a
% \code{func} value is shown in Fig.~\ref{fig:vec-tree}.
% %% The strucuture of a \code{func} value can be 'flattened' to make it
% %% linear.
% %% Such a flattening is done to convert the \code{func} values in the
% %% \gls{NFA} that we construct to corresponding \code{Vector.t} values.
% %% This is done because the built-in \code{Vec} type of Clash has a
% %% linear shape.
% % A given value of type \code{Vector.t A n} can be  inhabited by a f 
% %
% There is only one way to decompose a \code{Vector.t A 3} value: into a
% \code{A} and a \code{Vector.t A 2}.
% But a corresponding $\mathbb{3} \vto A$ value, where $\mathbb{3}$ is a
% \code{sort} value indicative of a finite type with cardinality 3, can
% be decomposed into one of five different possibilities some of which
% are illustrated in Fig.~\ref{fig:vec-tree}.
% This creates considerable difficulty in building correctness proofs
% for definitions made using \code{func}.

% \begin{figure}
%   %\centering
%   \hfill%
%   \begin{minipage}{0.7\textwidth}%
%   {\small
\begin{alectryon}
  % Generator: Alectryon
  \sep
  \begin{sentence}
    \begin{input}
      \PY{k+kn}{Inductive}~\PY{n+nf}{t}~\PY{n+nv}{A}~\PY{o}{:}~\PY{n}{nat}~\PY{o}{\PYZhy{}\PYZgt{}}~\PY{k+kt}{Type}~\PY{o}{:=}\nl
      \PY{o}{|}~\PY{n}{nil}~\PY{o}{:}~\PY{n}{t}~\PY{n}{A}~\PY{l+m+mi}{0}\nl
      \PY{o}{|}~\PY{n}{cons}~\PY{o}{:}~\PY{k+kr}{forall}~\PY{o}{(}\PY{n+nv}{h}~\PY{o}{:}~\PY{n}{A}\PY{o}{)}~\PY{o}{(}\PY{n+nv}{n}~\PY{o}{:}~\PY{n}{nat}\PY{o}{),}\nl
      ~~~~\PY{n}{t}~\PY{n}{A}~\PY{n}{n}~\PY{o}{\PYZhy{}\PYZgt{}}~\PY{n}{t}~\PY{n}{A}~\PY{o}{(}\PY{n}{S}~\PY{n}{n}\PY{o}{).}
    \end{input}
  \end{sentence}
\end{alectryon}
}
%   \end{minipage}%
%   \hfill%
%   \caption{Vector type from Coq standard library}
%   \label{fig:vector-type}
% \end{figure}

% \begin{figure}
% \begin{subfigure}{0.5\textwidth}
%   \centering
% \begin{forest}
% [+ 
%    [\{a\}]
%    [+
%      [\{b\}]
%      [\{c\}]
%    ]
% ]
% \end{forest}
% \caption{$\{a\} \cup (\{b\} \cup \{c\})$}
% \label{fig:vec-tree-1}
% \end{subfigure}
% %
% \begin{subfigure}{0.5\textwidth}
%   \centering
% \begin{forest}
% [+ 
%    [+
%      [\{a\}]
%      [\{b\}]
%    ]
%    [\{c\}]
% ]
% \end{forest}
% \caption{$(\{a\} \cup \{b\}) \cup \{c\}$}
% \label{fig:vec-tree-2}
% \end{subfigure}

% \begin{subfigure}{0.5\textwidth}
%   \centering
% \begin{forest}
% [+ 
%    [+
%      [\{a\}]
%      [$\emptyset$]
%    ]
%    [+
%      [\{b\}]
%      [\{c\}]
%    ]
% ]
% \end{forest}
% \caption{$(\{a\} \cup \emptyset) \cup (\{b\} \cup \{c\})$}
% \label{fig:vec-tree-3}
% \end{subfigure}
% %
% \begin{subfigure}{0.5\textwidth}
%   \centering
% \begin{forest}
% [+ 
%    [+
%      [\{a\}]
%      [+ [$\emptyset$] [$\emptyset$]]
%    ]
%    [+ [\{b\}] [\{c\}]]
% ]
% \end{forest}
% \caption{$(\{a\} \cup (\emptyset \cup \emptyset)) \cup (\{b\} \cup \{c\})$}
% \label{fig:vec-tree-4}
% \end{subfigure}
% \caption{Different 'shapes' of vector \code{[a,b,c]} encoded as a \code{func} value}
% \label{fig:vec-tree}
% \end{figure}

% %%%%
% \section{Correctness}
% \gls{TCB} of a formalization is the part of its implementation that is
% assumed to be correct since no formal correctness proof is offered
% for it.
% As the correctness proof of our \gls{rgx} to \gls{NFA} conversion
% procedure is not yet complete, the entirety of the codebase involved
% in our workflow is currently part of the \gls{TCB}.
% Once a formal correctness proof of the \gls{NFA} generation procedure
% is complete, the \gls{TCB} can be reduced to the Coq type checker, the
% extraction procedure, the Clash compiler and the \gls{EDA} toolchain.
% Metaprogramming within Coq could be helpful to verify the extraction
% procedure and reduce the \gls{TCB} even further.


\section{Properties expressible by regex}
In \gls{RV}, the monitor checks whether a finite prefix of the output
trace is in violation of the correctness specification.
This prefix is the part of the \gls{SuV}'s output trace that is
currently visible to the monitor.
Since this prefix is of a finite length, it can be matched by means of a
\gls{rgx}.
%
% In \gls{RV}, the currently available part of the output trace is being
% checked against a correctness specification.
% This is a finite prefix of the trace that can be matched by a regex.
%
Therefore, a certain class of properties useful for \gls{RV} can be
expressed with our \gls{rgx}.
% TODO: Classification of RV props. Bad prefix, good prefix, ugly prefix  
\emph{Liveness properties}, which state that a condition eventually
occurs in the future, can be satisfied by examining the finite prefix
of the trace.
Hence, such properties can be specified with a \gls{rgx}.
In contrast to liveness, \emph{safety properties} are those properties
that specify that a condition always holds true.
Specifications expressing conformance to safety properties cannot be
written as a \gls{rgx} since no finite trace prefix can provide
sufficient information to conclude that the property is satisfied.
% is for that unless the property has been violated.
Violation of safety properties can be expressed as a liveness
properties instead.

% TODO: Make the example a safety
Let us consider a simple example. Suppose that there is a door equipped
with an intrusion detection system. When this door is opened while the
system is armed, an alarm should be triggered. There is no need for
alarm if the system is not armed.
This situation can be specified as a regex:
$(armed \land \lnot open)*;\ (armed \land open);\ alert$.
Here $armed$, $open$ and $alert$ are atoms.
An \gls{NFA} corresponding to this \gls{rgx} constructed using our
workflow is shown in Fig~\ref{fa:eg-property}.
%
% TODO:
% This scenario is a simplified form of a safety-critical system wher

% \begin{tabular}{rl}
%   \colorbox{teal!10}{
%     \parbox{0.10\textwidth}{
%       $a*;\ b;\ c$ \\
%       ${} \ {}\ {}\ \downarrow$ \\
%       $(a+|\varepsilon);\ b;\ c$ \\
%     }
%   } &
%   % \colorbox{blue!20}{
%   \colorbox{teal!10}{
%     \parbox{0.23\textwidth}{
%       \begin{tabular}{ccl}
%         $armed \land \lnot open$ & $\mapsto$ & $a$ \\
%         $armed \land open$       & $\mapsto$ & $b$ \\
%         $alert$                  & $\mapsto$ & $c$ \\
%       \end{tabular}
%     }
%   }
% \end{tabular}

% Eg nfa picture
\begin{figure}
  \begin{center}
    \begin{tikzpicture}[node distance=1.5cm]
  
  % b
  \node[state] (bp1) {};
  \node[right of=bp1,, state, accepting] (bp2) {};
  \draw[->] (bp1) edge node[above] {\texttt b} (bp2);

  % eps
  \node[left=of bp1, xshift=5mm, initial, state, accepting] (eps) {};

  % a-plus
  \node[below of=bp1, state, accepting] (ap2) {};
  \node[below of=eps, initial, state] (ap1) {};
  \draw[->] (ap1) edge node[above] {\texttt a} (ap2);
  
  % c
  \node[below of=bp2, state] (cp1) {};
  \node[right of=cp1,, state, accepting] (cp2) {};
  \draw[->] (cp1) edge node[above] {\texttt c} (cp2);
  

  \draw[->] (eps) edge node[above] {$\varepsilon$} (bp1);
  \draw[->] (ap2) edge node[right] {$\varepsilon$} (bp1);
  \draw[->] (bp2) edge node[left] {$\varepsilon$} (cp1);
  \path (ap2) edge[loop below] node {\texttt a} (ap2);

  \node (abox)[
    left of=ap2,
    % draw=teal,
    %label=$Verified$,
    % fill=red!10,
    fill=blue!20!green,
    % fill=orange!20!yellow,
    opacity=.3,
    thick,
    minimum width=3.28cm,
    minimum height=2cm,
    rectangle,
    label={[font=\bfseries,text=blue]below: $(armed \land \lnot open)+$},
    % label={[font=\bfseries,text=blue,shift=(ap2.south)]below right: $(\lnot armed)+$},
    xshift=0.5cm,
    yshift=-0.4cm
  ] {};  

  \node (bbox)[
    left of=bp2,
    % draw=teal,
    %label=$Verified$,
    % fill=red!10,
    label={[font=\bfseries,text=blue] $armed \land open$},
    fill=purple!40!green,
    % fill=yellow!20!green,
    opacity=.3,
    thick,
    minimum width=3cm,
    minimum height=1.3cm,
    rectangle,
    xshift=0.8cm,
    yshift=-0.0cm
  ] {};  

  \node (cbox)[
    left of=cp2,
    % draw=teal,
    %label=$Verified$,
    % fill=red!10,
    % label={[font=\bfseries,text=blue] $alert$},
    label={[font=\bfseries,text=blue]below: $alert$},
    fill=orange!20!green,
    opacity=.3,
    thick,
    minimum width=3cm,
    minimum height=1.3cm,
    rectangle,
    xshift=0.8cm,
    yshift=-0.0cm
  ] {};  

  \node (epsbox)[
    left of=eps,
    % label={\textbf{$\varepsilon$}},
    label={[font=\bfseries,text=blue] $\varepsilon$},
    % draw=teal,
    %label=$Verified$,
    % fill=red!10,
    fill=red!20!blue,
    opacity=.2,
    thick,
    minimum width=1.5cm,
    minimum height=1.3cm,
    rectangle,
    xshift=1.5cm,
    yshift=-0.0cm
  ] {};  

\end{tikzpicture}

  \end{center}
  \caption{$(armed \land \lnot open)*;\ (armed \land open);\ alert$}
  \label{fa:eg-property}
\end{figure}


% TODO: Did property based testing

% Even when coq part is verified, remaining isn't. Could be fixed by
% verifying extraction. MetaCoq??


