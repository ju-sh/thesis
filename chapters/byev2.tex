\chapter{Conclusion}\label{ch:bye}
In this thesis, we described a workflow by which hardware
implementations of \gls{rgx} matchers can be produced from within the
Coq theorem prover.
%
% The \gls{rgx} that we consider are more powerful than classic
% \gls{rgx} since we use Boolean tests on input as atoms.
%
% The core component of our flow is implemented within the Coq proof
% assistant, which allows for the possibility of reasoning about the
% generated designs in the future.
% Though the \gls{rgx} is provided within Coq, users need not have
% significant theorem proving experience as we provide an \gls{eDSL}
% with friendly syntax to write the \gls{rgx}.
%
Our workflow is built entirely with open source tools and further
configuration is possible if required.
The end result produced by this workflow is an \gls{FPGA} loadable
bitstream.


We implemented the core part of our flow within Coq to leave open the
possibility of proving its correctness.
% It is possible to prove the correctness of the \glspl{NFA} that our
% workflow generates within Coq.
This correctness proof is planned to be taken up in a follow-up work.
%
Once such a proof is completed, we additionally need to trust the
correctness of the following sub-systems involved in our workflow to
consider the hardware that we generate correct:
the Coq kernel, the Coq extraction procedure, the Clash compiler and
the \gls{EDA} tools.
These subsystems collectively form the \gls{TCB} of our formalization.
It is desirable to minimize the size of the \gls{TCB}.
%
Since our entire development relies on the type safety of Coq, we
cannot remove the Coq kernel from the \gls{TCB}.
% since we use Coq's
% type checker for proofs.
It is possible to reduce the size of our \gls{TCB} by generating
Verilog code directly from Coq, instead of relying on a full-blown,
sophisticated compiler like Clash.
%
In principle, it is possible to remove the \gls{EDA} tools from the
\gls{TCB} if we build an entire verified \gls{EDA} toolchain similar
to how the CompCert compiler~\cite{leroy2009formal} was built.
However, building such a verified toolchain is a highly challenging
task as these tools are very complex.
% removing the \gls{EDA} tools from \gls{TCB} is difficult since
% they are very complex and formally verifying their correctness is a
% highly non-trivial task.


Our experiments show that our method results in matchers that are
usable for practical applications.
% We believe that the matchers produced by our flow is suitable fo
The matchers have reasonably low resource utilization, which we
believe is sufficient for use in real-life scenarios like monitoring
liveliness properties.
%
We were able to achieve this despite not considering possible avenues
of optimization on the side of the input \gls{rgx} and the
corresponding \gls{NFA}.
The present work can be extended in the future to expand the
generalized form of \glspl{rgx} that we consider to the whole of
\gls{SERE} and eventually to \gls{PSL} itself.

