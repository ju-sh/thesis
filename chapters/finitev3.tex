\chapter{Matching regular expressions} \label{ch:method}
In this chapter, we describe how \glspl{rgx} and \glspl{NFA} are
represented in Coq.
The finite automata that we saw in Chapter~\ref{ch:prelims} are
defined in terms of finite sets.
The finiteness of the state space of the automata is a crucial factor
in automata theory that makes algorithms like subset
construction~\cite{rabin1959finite} possible.
This chapter also describes our formalization of finite sets and how
operations on finite sets are used to build \glspl{NFA} corresponding
to a given \gls{rgx} within Coq.

\section{Representing regular expressions}
We represent \glspl{rgx} in Coq using the \code{re T} type
(Fig.~\ref{code:re-syntax}).
This type corresponds to the grammar shown in
Fig.~\ref{gram:classic-re-syntax}.
The \code{re T} is a polymorphic type where \code{T} is a type variable
that can be instantiated to any type.
We represent our generalized \glspl{rgx} matching words over an alphabet
$\Sigma$ using values of the type \code{re (Σ → \bbool)}.
%
% We represent \glspl{rgx} matching words over an alphabet $\Sigma$ as
% values of a type \code{re (Σ → \bbool)} in Coq.
% This type is made by instantiating the type parameter of the
% polymorphic type \code{re} with the type of Boolean functions from the
% input alphabet (Fig.~\ref{code:re-syntax}).

We formally specify when a regex r matches a word w, denoted by $w
\models r$, using the rules shown in Fig.~\ref{gram:re-sem-rules}.
In Coq, these inductive rules can be captured using a propositional
type \code{sem r w} as shown in Fig.~\ref{code:re-sem}.
% We use the notation $w \models r$ to denote that the \gls{rgx} $r$
% matches a word $w$.
% % The rules of \gls{rgx} matching is shown in Fig.~\ref{gram:re-sem-rules}.
% A \gls{rgx} $r$ is said to match a word $w$ if a $w \models r$ can be
% derived from the rules of \gls{rgx} matching show in  
% % The semantics of \gls{rgx} matching is given as per the rules in
% Fig.~\ref{gram:re-sem-rules}.
% We represent these rules as an inductive type named $sem$ in Coq
% (Fig.~\ref{code:re-sem}).
% % A \gls{rgx} $r$ is said to match a word $w$ if $w \models r$ can be
% % derived as per these rules.
% % We represent the semantics of regular expression matching in Coq as a
% % propositional type \code{sem} (Fig.~\ref{code:re-sem}).
% % The constructors of this type correspond to the rules shown in
% % Fig.~\ref{gram:re-sem-rules}.
For better readability, we use the Coq notation \code{w\;\cmodels \;r}
to represent the type \code{sem r w}.
A value of type \code{w\;\cmodels \;r} corresponds to a derivation
demonstrating how the \gls{rgx} \code{r} matched the word \code{w}.

\begin{figure}
  \hfill%
  \begin{minipage}{0.7\textwidth}%
  \begin{Verbatim}[commandchars=\\\{\}]
\PY{k+kn}{Inductive} \PY{n}{re} \PY{o}{(}\PY{n}{T} \PY{o}{:} \PY{k+kt}{Type}\PY{o}{)} \PY{o}{:} \PY{k+kt}{Type} \PY{o}{:=}
\PY{o}{|} \PY{n}{Eps} \PY{o}{:} \PY{n}{re} \PY{n}{T}
\PY{o}{|} \PY{n}{Atom} \PY{o}{:} \PY{n}{T} \PY{o}{\PYZhy{}\PYZgt{}} \PY{n}{re} \PY{n}{T}
\PY{o}{|} \PY{n}{Cat} \PY{o}{:} \PY{n}{re} \PY{n}{T} \PY{o}{\PYZhy{}\PYZgt{}} \PY{n}{re} \PY{n}{T} \PY{o}{\PYZhy{}\PYZgt{}} \PY{n}{re} \PY{n}{T}
\PY{o}{|} \PY{n}{Alt} \PY{o}{:} \PY{n}{re} \PY{n}{T} \PY{o}{\PYZhy{}\PYZgt{}} \PY{n}{re} \PY{n}{T} \PY{o}{\PYZhy{}\PYZgt{}} \PY{n}{re} \PY{n}{T}
\PY{o}{|} \PY{n}{Plus} \PY{o}{:} \PY{n}{re} \PY{n}{T} \PY{o}{\PYZhy{}\PYZgt{}} \PY{n}{re} \PY{n}{T}\PY{o}{.}
\end{Verbatim}

  \end{minipage}%
  \hfill%
  \caption{Coq type of regex}
  \label{code:re-syntax}
\end{figure}

\begin{figure}
\begin{mathpar}
  \inferrule*[right=SEps]
    { }
    {[] \models \varepsilon}

  \inferrule*[right=SAtom]
    {f: T \to \mathbb{B} \\ c: T \\ f \ c = True}
    {[c] \models Atom \ f}

  \inferrule*[right=SCat]
    {w_1 \models r_1 \\ w_2 \models r_2}
    {w_1 w_2 \models r_1;r_2}

  \inferrule*[right=SAltL]
    {w \models r_1}
    {w \models r_1|r_2}

  \inferrule*[right=SAltR]
    {w \models r_2}
    {w \models r_1|r_2}

  \inferrule*[right=SPlusD]
    {w \models r}
    {w \models r^+}

  \inferrule*[right=SPlusM]
    {w_1 \models r \\ w_2 \models r^+}
    {w_1 w_2 \models r^+}
\end{mathpar}
\caption{Rules of regex matching}
\label{gram:re-sem-rules}
\end{figure}

\begin{figure}
  \begin{Verbatim}[commandchars=\\\{\}]
\PY{k+kn}{Inductive} \PY{n}{sem} \PY{o}{\PYZob{}}\PY{n}{Σ} \PY{o}{:} \PY{k+kt}{Type}\PY{o}{\PYZcb{}} \PY{o}{:} \PY{n}{re} \PY{o}{(}\PY{n}{Σ} \PY{o}{\PYZhy{}\PYZgt{}} \PY{n}{bool}\PY{o}{)} \PY{o}{\PYZhy{}\PYZgt{}} \PY{n}{list} \PY{n}{Σ} \PY{o}{\PYZhy{}\PYZgt{}} \PY{k+kt}{Prop} \PY{o}{:=}
\PY{o}{|} \PY{n}{SEps} \PY{o}{:} \PY{n+nb+bp}{[]} \PY{n+nb+bp}{\alecmodels} \PY{n}{Eps}
\PY{o}{|} \PY{n}{SAtom} \PY{o}{:} \PY{k}{forall} \PY{o}{(}\PY{n}{f} \PY{o}{:} \PY{n}{Σ} \PY{o}{\PYZhy{}\PYZgt{}} \PY{n}{bool}\PY{o}{)} \PY{o}{(}\PY{n}{a}\PY{o}{:} \PY{n}{Σ}\PY{o}{)}\PY{o}{,}
    \PY{n}{f} \PY{n}{a} \PY{o}{=} \PY{n+nb+bp}{true} \PY{o}{\PYZhy{}\PYZgt{}} \PY{o}{[}\PY{n}{a}\PY{o}{]} \PY{n+nb+bp}{\alecmodels} \PY{o}{(}\PY{n}{Atom} \PY{n}{f}\PY{o}{)}
\PY{o}{|} \PY{n}{SCat} \PY{o}{:} \PY{k}{forall} \PY{o}{(}\PY{n}{r1} \PY{n}{r2} \PY{o}{:} \PY{n}{re.t} \PY{o}{(}\PY{n}{Σ} \PY{o}{\PYZhy{}\PYZgt{}} \PY{n}{bool}\PY{o}{)}\PY{o}{)} \PY{o}{(}\PY{n}{w1} \PY{n}{w2} \PY{o}{:} \PY{n}{list} \PY{n}{Σ}\PY{o}{)}\PY{o}{,}
    \PY{n}{w1} \PY{n+nb+bp}{\alecmodels} \PY{n}{r1} \PY{o}{\PYZhy{}\PYZgt{}} \PY{n}{w2} \PY{n+nb+bp}{\alecmodels} \PY{n}{r2} \PY{o}{\PYZhy{}\PYZgt{}} \PY{o}{(}\PY{n}{w1} \PY{o}{+}\PY{o}{+} \PY{n}{w2}\PY{o}{)} \PY{n+nb+bp}{\alecmodels} \PY{n}{Cat} \PY{n}{r1} \PY{n}{r2}
\PY{o}{|} \PY{n}{SAltL} \PY{o}{:} \PY{k}{forall} \PY{o}{(}\PY{n}{r1} \PY{n}{r2} \PY{o}{:} \PY{n}{re.t} \PY{o}{(}\PY{n}{Σ} \PY{o}{\PYZhy{}\PYZgt{}} \PY{n}{bool}\PY{o}{)}\PY{o}{)} \PY{o}{(}\PY{n}{w} \PY{o}{:} \PY{n}{list} \PY{n}{Σ}\PY{o}{)}\PY{o}{,}
    \PY{n}{w} \PY{n+nb+bp}{\alecmodels} \PY{n}{r1} \PY{o}{\PYZhy{}\PYZgt{}} \PY{n}{w} \PY{n+nb+bp}{\alecmodels} \PY{o}{(}\PY{n}{Alt} \PY{n}{r1} \PY{n}{r2}\PY{o}{)}
\PY{o}{|} \PY{n}{SAltR} \PY{o}{:} \PY{k}{forall} \PY{o}{(}\PY{n}{r1} \PY{n}{r2} \PY{o}{:} \PY{n}{re.t} \PY{o}{(}\PY{n}{Σ} \PY{o}{\PYZhy{}\PYZgt{}} \PY{n}{bool}\PY{o}{)}\PY{o}{)} \PY{o}{(}\PY{n}{w} \PY{o}{:} \PY{n}{list} \PY{n}{Σ}\PY{o}{)}\PY{o}{,}
    \PY{n}{w} \PY{n+nb+bp}{\alecmodels} \PY{n}{r2} \PY{o}{\PYZhy{}\PYZgt{}} \PY{n}{w} \PY{n+nb+bp}{\alecmodels} \PY{o}{(}\PY{n}{Alt} \PY{n}{r1} \PY{n}{r2}\PY{o}{)}
\PY{o}{|} \PY{n}{SPlusD} \PY{o}{:} \PY{k}{forall} \PY{o}{(}\PY{n}{r} \PY{o}{:} \PY{n}{re.t} \PY{o}{(}\PY{n}{Σ} \PY{o}{\PYZhy{}\PYZgt{}} \PY{n}{bool}\PY{o}{)}\PY{o}{)} \PY{o}{(}\PY{n}{w}\PY{o}{:} \PY{n}{list} \PY{n}{Σ}\PY{o}{)}\PY{o}{,}
    \PY{n}{w} \PY{n+nb+bp}{\alecmodels} \PY{n}{r} \PY{o}{\PYZhy{}\PYZgt{}} \PY{n}{w} \PY{n+nb+bp}{\alecmodels} \PY{o}{(}\PY{n}{Plus} \PY{n}{r}\PY{o}{)}
\PY{o}{|} \PY{n}{SPlusM} \PY{o}{:} \PY{k}{forall} \PY{o}{(}\PY{n}{r} \PY{o}{:} \PY{n}{re.t} \PY{o}{(}\PY{n}{Σ} \PY{o}{\PYZhy{}\PYZgt{}} \PY{n}{bool}\PY{o}{)}\PY{o}{)} \PY{o}{(}\PY{n}{w1} \PY{n}{w2}\PY{o}{:} \PY{n}{list} \PY{n}{Σ}\PY{o}{)}\PY{o}{,}
    \PY{n}{w1} \PY{n+nb+bp}{\alecmodels} \PY{n}{r} \PY{o}{\PYZhy{}\PYZgt{}} \PY{n}{w2} \PY{n+nb+bp}{\alecmodels} \PY{o}{(}\PY{n}{Plus} \PY{n}{r}\PY{o}{)} \PY{o}{\PYZhy{}\PYZgt{}} \PY{o}{(}\PY{n}{w1} \PY{o}{+}\PY{o}{+} \PY{n}{w2}\PY{o}{)} \PY{n+nb+bp}{\alecmodels} \PY{o}{(}\PY{n}{Plus} \PY{n}{r}\PY{o}{)}
\PY{n}{where} \PY{l+s+s2}{\PYZdq{}}\PY{l+s+s2}{w \alecmodels r}\PY{l+s+s2}{\PYZdq{}} \PY{o}{:=} \PY{o}{(}\PY{n}{sem} \PY{n}{r} \PY{n}{w}\PY{o}{)}\PY{o}{:} \PY{n}{re\PYZus{}scope}\PY{o}{.}
\end{Verbatim}

\caption{Regex semantics as Coq type}
\label{code:re-sem}
\end{figure}

\subsection*{Utility of tests as atoms}
% The \glspl{rgx} that we deal with in this work are a generalized form
% for classic \gls{rgx}.
% This is because the atoms of our \glspl{rgx} have Boolean functions on
% input as atoms instead of specific input symbols.
The \glspl{rgx} that we consider in this thesis are a generalized form
of \glspl{rgx}.
Such \glspl{rgx} have many advantages over classic \glspl{rgx}.
%
For instance, the generality of the \glspl{rgx} that we consider allows
for a more succinct representation.
It is possible to merge atoms which would be distinct in classic
\gls{rgx} into a single atom in our \gls{rgx}.
Lesser number of atoms in the input \gls{rgx} means lesser number of
states in the matcher automata and thereby lower resource
utilization in the corresponding hardware implementation.
%
Consider the example shown in Fig~\ref{fig:state-count}.
Here, three states (corresponding to $x$, $y$ and $z$) can be merged
into a single state when a boolean test (\code{isin [x, y, z]}) is
used.
This is especially useful while implementing character classes in
\glspl{rgx}.

Although regular languages are closed under complementation and
intersection, construction of corresponding automata is difficult.
However, these operations are relatively easy to implement when
restricted to the atom level for the \glspl{NFA} that we construct.
This is because of the atoms of the \gls{rgx} that we consider being
Boolean functions.
%
% Operations like complementation and intersection of \gls{rgx}
% are difficult to attain without a state space explosion in the
% corresponding finite automata.
% But a form of these operations are achievable in the \gls{rgx}
% that we consider if the complementation and intersection are limited
% as atom-wise operations.
% Another point worth mention is that the atom-wise complement
% operation is quite easy to implement when the atoms are Boolean
% tests.
For example, complementation operation is just a matter of applying
Boolean negation to the test associated with the atom.
% For example, the test associated with an atom matching all numbers
% except 3 and 5 can be written as
% \code{fun n : nat => negb ((n=?3) || (n=?5))}.
We chose to consider \gls{rgx} with Boolean tests as atoms to be in
line with the \glspl{SERE} of \gls{PSL}.

\begin{figure}
  \centering
  \begin{subfigure}{0.5\textwidth}
      \centering
            \begin{tikzpicture}[
        node distance=2.5cm, % Minimum distance between two nodes. Change if necessary.
        every state/.style={ % Sets the properties for each state
        semithick,
        fill=gray!10},
        initial text={}, % No label on start arrow
        double distance=2pt, % Adjust appearance of accept states
        every edge/.style={ % Sets the properties for each transition
        draw,
        ->,>=stealth', % Makes edges directed with bold arrowheads
        auto,
        semithick}
      ]
        \node[state, initial] (s0) {};
        \node[state, right of=s0] (s1) {};
        \node[state, right of=s1] (s21) {};
        \node[state, accepting, right of=s21] (s3) {};
      
        \node[state, above of=s21] (s20) {};
        \node[state, below of=s21] (s22) {};
      
        \draw (s0) edge node {$a$} (s1);
        \draw (s1) edge node {$x$} (s20);
        \draw (s1) edge node {$y$} (s21);
        \draw (s1) edge node {$z$} (s22);
        \draw (s20) edge node {$b$} (s3);
        \draw (s21) edge node {$b$} (s3);
        \draw (s22) edge node {$b$} (s3);
      \end{tikzpicture}

      \caption{Classic regex}
      \label{fig:state-count-re}
  \end{subfigure}
  %\vspace{5em}
  \par\bigskip
  \begin{subfigure}{0.5\textwidth}
      \centering
            \begin{tikzpicture}[
        node distance=2.5cm, % Minimum distance between two nodes. Change if necessary.
        every state/.style={ % Sets the properties for each state
        semithick,
        fill=gray!10},
        initial text={}, % No label on start arrow
        double distance=2pt, % Adjust appearance of accept states
        every edge/.style={ % Sets the properties for each transition
        draw,
        ->,>=stealth', % Makes edges directed with bold arrowheads
        auto,
        semithick}
      ]
        \node[state, initial] (s0) {};
        \node[state, right of=s0] (s1) {};
        \node[state, right of=s1] (s2) {};
        \node[state, accepting, right of=s2] (s3) {};
      
        \draw (s0) edge node {$a$} (s1);
        \draw (s1) edge node {$x|y|z$} (s2);
        \draw (s2) edge node {$b$} (s3);
      \end{tikzpicture}

      \caption{\Gls{rgx}}
      \label{fig:state-count-taa}
  \end{subfigure}
  \caption{Difference in state count for $a;(x|y|z);b$ when atoms are tests}
  \label{fig:state-count}
\end{figure}

\subsection{Use of $r^+$ instead of $r^*$}
Classic definition of \glspl{rgx} as shown in
Fig.~\ref{gram:classic-re-syntax} uses $r^*$ as a constructor.
Instead, we chose to have $r^+$ as a basic operation, with $r^*$
implemented as a derived operation ($r^+|\varepsilon$) in terms of
$r^+$.
We made this choice to reduce the number of states in the \glspl{NFA}
corresponding to the input \glspl{rgx}.
%
\gls{NFA} construction for $n^*$ involves adding an extra state to the
\gls{NFA} for $n$.
Whereas the \gls{NFA} for $n^+$ does not lead to an increase in number
of states.
This fact is illustrated by examples shown in Figures~\ref{fig:nfa1}
and \ref{fig:nfa4}.
% %
% A general way of constructing an \gls{NFA} $n^*$ from a given
% \gls{NFA} $n$ consists of adding a new state which leads to an
% increase in the size of type representing the state space.
% %
% On the other hand, a general construction of $n^+$ can be made without
% adding extra states.
% Examples of this approach are shown in
% Fig.~\ref{fig:nfa-star-correct-correct} and
% Fig.~\ref{fig:nfa4-star-correct}.
% Observe that the \glspl{NFA} in these figures are made up of
% components corresponding to the \glspl{NFA} for $\varepsilon$ and the
% base \gls{NFA} itself.
% %
% For a given \gls{NFA} $n$, we might as well construct $n^+$ and define
% $n^*$ in terms of $n^+$ instead.
% For example, Fig.~\ref{fig:nfa4-plus-short} shows the automata for an
% $r^+$ when $Plus$ is used as a base operation instead of $Star$.
% Fig.~\ref{fig:nfa4-plus-long} shows an automata accepting the same
% language $L(r^+)$ but made as an automata for $r;r^*$.
% %
% The latter automata has lesser number of states.
% It is clear that the overall number of states in the automata can be
% reduced if $Plus$ is used as a base operation with $Star$ made as a
% derived operation in terms of $Plus$.

\begin{figure}
  \begin{subfigure}{1.0\linewidth}
    \centering
\begin{tikzpicture}[node distance=2.5cm]
\node[state, initial] (a) {A}; 
\node[state, accepting, right of=a] (b) {B}; 
\draw[->] (a) edge node[above] {$a$} (b);
\end{tikzpicture}
\caption{$L(N) = L(r)$}
\label{fig:nfa1-orig}
  \end{subfigure}
  \begin{subfigure}{0.5\linewidth}
    \centering
\begin{tikzpicture}[node distance=2.5cm]
\node[state, initial] (a) {A}; 
\node[state, accepting, right of=a] (b) {B}; 
\draw[->] (a) edge node[above] {$a$} (b);
\draw[->] (b) edge[bend left] node[below] {$\varepsilon$} (a);
\end{tikzpicture}
\caption{$L(N^+) = L(r^+)$}
\label{fig:nfa-star-wrong-correct}
  \end{subfigure}
\hfill
  \begin{subfigure}{0.5\linewidth}
    \centering
\begin{tikzpicture}[node distance=2.5cm]
  \node[state] (a) {A}; 
  \node[state, right of=a] (b) {B}; 
  \node[initial, accepting, state, left of=a, xshift=0mm] (s) {S}; 
  \draw[->] (s) edge node[above] {$\varepsilon$} (a);
  \draw[->] (a) edge node[above] {$a$} (b);
  \draw[->] (b) edge[bend left] node[below] {$\varepsilon$} (s);
\end{tikzpicture}
\caption{$L(N^*) = L(r^*)$}
\label{fig:nfa-star-correct-correct}
  \end{subfigure}
\caption{NFAs based on $r = a$}
\label{fig:nfa1}
\end{figure}

\begin{figure}
    \centering
  \begin{subfigure}{0.5\linewidth}
    \centering
\begin{tikzpicture}[node distance=2.5cm]
  \node[state, initial] (a) {A}; 
  \node[state, right of=a] (b) {B}; 
  \node[state, accepting, right of=b] (c) {C}; 
  \draw[->] (a) edge[bend left] node[above] {$a$} (b);
  \draw[->] (b) edge[bend left] node[below] {$b$} (a);
  \draw[->] (b) edge node[above] {$c$} (c);
\end{tikzpicture}
\caption{$L(N) = L(r)$}
\label{fig:nfa4-orig}
  \end{subfigure}
%   \begin{subfigure}{0.5\linewidth}
%     \centering
% \begin{tikzpicture}[node distance=2.5cm]
% \node[state, accepting, initial] (a) {A}; 
% \node[state, right of=a] (b) {B}; 
% \node[state, right of=b] (c) {C}; 
% \draw[->] (a) edge[bend left] node[above] {$a$} (b);
% \draw[->] (b) edge[bend left] node[below] {$b$} (a);
% \draw[->] (b) edge node[above] {$c$} (c);
% \draw[->] (c.50) edge[out=70, in=110, looseness=0.8] node[below] {$\varepsilon$} (a.130);
% \end{tikzpicture}
% \caption{$L(N') \neq L(r^*)$}
% \label{fig:nfa-star-wrong-wrong}
%   \end{subfigure}
%
  \begin{subfigure}{0.5\linewidth}
    \centering
\begin{tikzpicture}[node distance=2.5cm]
  \node[state] (a) {A}; 
  \node[state, right of=a] (b) {B}; 
  \node[state, right of=b] (c) {C}; 
  \node[state, accepting, initial, left of=a] (s) {S}; 
  \draw[->] (a) edge[bend left] node[above] {$a$} (b);
  \draw[->] (b) edge[bend left] node[below] {$b$} (a);
  \draw[->] (b) edge node[above] {$c$} (c);
  \draw[->] (s) edge node[below] {$\varepsilon$} (a);
  \draw[->] (c) edge[bend left] node[below] {$\varepsilon$} (s);
\end{tikzpicture}
\caption{$L(N^*)$}
\label{fig:nfa4-star-correct}
  \end{subfigure}
%
  \begin{subfigure}{0.5\linewidth}
    \centering
\begin{tikzpicture}[%
  node distance=2.5cm,
  decoration={
    markings,
    mark=at position 0.5 with {\arrow{>}}
 }
]
  \node[state, initial] (a) {A}; 
  \node[state, right of=a] (b) {B}; 
  \node[state, accepting, right of=b] (c) {C}; 
  \draw[->] (a) edge[bend left] node[above] {$a$} (b);
  \draw[->] (b) edge[bend left] node[above] {$b$} (a);
  \draw[->] (b) edge node[above] {$c$} (c);
  \draw[->]
  %\draw[postaction={decorate}]
    (c) edge[bend left] node[below] {$\varepsilon$} (a);
\end{tikzpicture}
\caption{$L(N^+) = L(r^+)$}
\label{fig:nfa4-plus-short}
  \end{subfigure}
%
  \begin{subfigure}{0.5\linewidth}
    \centering
\begin{tikzpicture}[%
  node distance=2.5cm,
  decoration={
    markings,
    mark=at position 0.5 with {\arrow{>}}
 }
]
  \node[state, initial] (a0) {A}; 
  \node[state, right of=a0] (b0) {B}; 
  \node[state, accepting, right of=b0] (c0) {C}; 
  \draw[->] (a0) edge[bend left] node[above] {$a$} (b0);
  \draw[->] (b0) edge[bend left] node[below] {$b$} (a0);
  \draw[->] (b0) edge node[above] {$c$} (c0);

  \node[state, below of=a0] (a) {A}; 
  \node[state, right of=a] (b) {B}; 
  \node[state, accepting, right of=b] (c) {C}; 
  \node[state, left of=a] (s) {S}; 
  \draw[->] (a) edge[bend left] node[below] {$a$} (b);
  \draw[->] (b) edge[bend left] node[below] {$b$} (a);
  \draw[->] (b) edge node[above] {$c$} (c);
  \draw[->] (s) edge node[below] {$\varepsilon$} (a);
  \draw[->] (c) edge[bend left] node[below] {$\varepsilon$} (s);

  \draw[postaction={decorate}]
     (c0) .. controls +(-1.5cm,-2.0cm)
          .. +(-3.5cm, -1.5cm) node[above] {$\varepsilon$}
          .. controls +(-2.0cm,0.5cm) and +(0.5cm, 1.0cm)
          .. +(-7.5cm, -2.02cm)
     (s);
\end{tikzpicture}
\caption{$L(N^+) = L(r;r^*)$}
\label{fig:nfa4-plus-long}
  \end{subfigure}
\caption{NFAs based on $r = (ab)^*ac$}
\label{fig:nfa4}
\end{figure}

\subsection{eDSL}
% The input to our workflow are values of the \code{re (Σ → \bbool)}
% type.
Manually writing the Coq terms corresponding to values of the type
\code{re (Σ → \bbool)} is tedious and error-prone due to their
verbosity.
Hence, we provide a light-weight \gls{eDSL} which offers a 
convenient syntax with which \glspl{rgx} may be written within Coq
(Table~\ref{tab:edsl-constructors}).
This \gls{eDSL} is made using Coq's notation mechanism which lets us
customize the default parsing of Coq.
Constructions from Gallina can also be used as needed while using this
\gls{eDSL}.
An example of an \gls{rgx} in Coq written using our \gls{eDSL} is
shown in Fig.~\ref{code:edsl-eg}.
Expressions written in our \gls{eDSL} are enclosed between \code{<\{}
and \code{\}>} to distinguish it from normal Gallina code.
Note how the \code{let} expressions of Gallina is used together with
the \gls{eDSL} constructs to build Coq terms.

\begin{figure}
  \begin{subfigure}{0.5\linewidth}
\begin{mathpar}
\begin{array}{ll}
\toprule
  \text{eDSL} & \text{Coq term}          \\
\midrule
  e           & \text{$Eps$}             \\
  \#a         & \text{$Atom$ $a$}        \\
  r_1;r_2     & \text{$Cat$ $r_1$ $r_2$} \\
  r_1|r_2     & \text{$Alt$ $r_1$ $r_2$} \\
  r+          & \text{$Plus$ $r$}        \\
\bottomrule
\end{array}
\end{mathpar}
\caption{Base operations}
\label{tab:edsl-base}
\end{subfigure}
  \hfill
    \begin{subfigure}{0.5\linewidth}%
      \begin{mathpar}%
        \begin{array}{ll}
          \toprule
          \text{eDSL} & \text{Meaning}                       \\
          \midrule
          r*         & \varepsilon│r^+                      \\
          r?          & \varepsilon│r                        \\
          r\{n,m\}    & \text{Match $r$, $n$ to $m$ times}   \\
          r\{,m\}     & \text{Match $r$, at most $m$ times}  \\
          r\{n,\}     & \text{Match $r$, at least $n$ times} \\
          r\{n\}      & \text{Match $r$, exactly $n$ times}  \\
          \bottomrule
        \end{array}%
      \end{mathpar}%
      \label{tab:edsl-derived}
      \caption{Derived operations}
    \end{subfigure}
\caption{eDSL syntax for \code{re} constructors}
\label{tab:edsl-constructors}
\end{figure}

\begin{figure}
  \centering
  \begin{subfigure}{0.5\linewidth}
  \centering
% Cat (Cat (Alt Eps
%   (Plus (Atom (fun '(r, o, a) =>
%      andb r (negb o)))))
%   (Atom (fun '(r, o, a) => andb r o)))
%   (Atom (fun '(r, o, a) => a))
\begin{verbatim}
Cat (Alt Eps (Atom (fun v => 5 <=? v)))
    ((Cat
     (Alt (Atom (fun v => 5 <=? v))
          (Atom (fun v => v <? 5)))
     (Atom (fun v => v <? 5)))).
\end{verbatim}
\caption{Without \gls{eDSL}}
\label{code:edsl-eg-no}
\end{subfigure}%
%
\begin{subfigure}{0.5\linewidth}
  \centering
  % (#(fun '(r, o, a) => r && negb o)+ | e);
  %  #(fun '(r, o, a) => r && o);       
  %  #(fun '(r, o, a) => a)             
\begin{verbatim}
  let hi := fun v => 5 <=? v in
  let low := fun v => v <? 5 in
  <{
    #hi*; (#hi | #low); #low
  }>.
\end{verbatim}
\caption{With \gls{eDSL}}
\label{code:edsl-eg-yes}
\end{subfigure}
\caption{Example of an \code{re} term}
\label{code:edsl-eg}
\end{figure}

\section{Representing finiteness} \label{sec:repr-finset}
The notion of finiteness is intrinsically linked to the definition of
\glspl{NFA}.
Moreover, as we intend to convert the \glspl{NFA} that we generate in
Coq to hardware,
we need to make this finiteness explicit in the Coq representation of
\glspl{NFA}.
% This is because we want our matchers to be made as hardware
% implementations.
This is because hardware resources are inherently finite.

In order to build a representation of finiteness that can be used in
our constructions of \glspl{NFA},
% as described in
% Section~\ref{sec:thompson-overview},
we need reifications in Coq of the following:

\begin{itemize}
\item Finite sets
\item Subsets of finite sets
\item Functions from finite sets
\end{itemize}

% For a finite set $U$, subsets of $U$ can be represented as functions
% whose domain is $U$.
Subsets of a finite set $U$ can be represented as its characteristic
function from $U$ to Boolean.
%
% Subsets of finite sets can be implemented as functions from finite types.
% This is because the former can be represented by means of their
% characteristic function.
% A subset of a finite set $X$ can be indicated with a Boolean function
% from $X$.
In our constructions of automata, we need a few basic operations like
union of subsets of finite sets (Table.~\ref{tab:set-coq-corr}).
All these operations can be implemented as operations on the
characteristic functions of finite sets.
% We use a representation of finite sets and associated operations 


% Similarly, operations on such subsets can be implemented in terms
% of operations over a representation of finite sets. 
% We need finite sets, subsets of finite sets and corresponding
% operations to construct \glspl{NFA} corresponding to \glspl{rgx} in
% such a way that the resultant automata can be used to generate a
% hardware implementation of a corresponding matcher via extraction.

The operations that we need are available as part of third party Coq
libraries like the \code{finset} type of mathcomp.
However, such types are designed primarily for carrying out proofs and
are not suitable for use in definitions meant for conversion to
hardware.

% There are several implementations of finite set available as
% part of third party Coq libraries.
% But they are usually optimized for carrying out proofs and are not
% meant for extraction.
% An example is the \code{finset} type of the mathcomp library. This
% type represents finite sets and has a large number of lemmas
% and theorems associated with it already defined.
% However, we found that this type does not extract in a way that is
% convenient for conversion to hardware designs.

For the purposes of this thesis, we represent finiteness using user
defined types.
These types, together with associated definitions, gives a minimal
implementation that is sufficient for our use case.
%
Our design goals when
% The following aspects need to be considered when
choosing the way to represent finiteness in Coq are as follows:
\begin{itemize}
  \item
    The representation must be in a form where it is convenient to
    implement the operations needed to construct \glspl{NFA}.
  \item
    The representation must be computation-friendly and in a form
    which can be mapped to primitive hardware operations.
\end{itemize}

\subsection{Finite sets}
We define the type \code{sort} to reify finite sets in Coq
(Fig.~\ref{code:coq-sort-type}).
The constructors \code{Zero}, \code{One} and \code{Add} are
reifications of the empty set, singleton set and the disjoint union
operation, respectively.
% Values \code{Zero} and \code{One} represents the empty and singleton
% sets respectively.
% The constructor \code{Add} denotes the disjoint union of
% two finite sets.
The semantics of \code{sort} is shown in
Fig.~\ref{code:coq-sort-sem}.
Each \code{sort} value $s$ corresponds to a finite type $⟦s⟧$ in Coq
that represents a finite set.
%fixed cardinality.
% The cardinality of a \code{sort} value is the number of \code{One}
% constructors used to build it.
We use the notation $|s|$ to indicate the cardinality of the finite
set represented by a \code{sort} value $s$.

\begin{figure}
\begin{subfigure}{0.5\textwidth}
  \begin{Verbatim}[commandchars=\\\{\}]
\PY{k+kn}{Inductive} \PY{n}{sort} \PY{o}{:} \PY{k+kt}{Type} \PY{o}{:=}
\PY{o}{|} \PY{n}{Zero} \PY{o}{:} \PY{n}{sort}
\PY{o}{|} \PY{n}{One} \PY{o}{:} \PY{n}{sort}
\PY{o}{|} \PY{n}{Add} \PY{o}{:} \PY{n}{sort} \PY{o}{\PYZhy{}\PYZgt{}} \PY{n}{sort} \PY{o}{\PYZhy{}\PYZgt{}} \PY{n}{sort}\PY{o}{.}

\PY{k+kn}{Notation} \PY{l+s+s2}{\PYZdq{}}\PY{l+s+s2}{0}\PY{l+s+s2}{\PYZdq{}} \PY{o}{:=} \PY{n}{Zero} \PY{o}{:} \PY{n}{sort\PYZus{}scope}\PY{o}{.}
\PY{k+kn}{Notation} \PY{l+s+s2}{\PYZdq{}}\PY{l+s+s2}{1}\PY{l+s+s2}{\PYZdq{}} \PY{o}{:=} \PY{n}{One} \PY{o}{:} \PY{n}{sort\PYZus{}scope}\PY{o}{.}
\PY{n}{Infix} \PY{l+s+s2}{\PYZdq{}}\PY{l+s+s2}{+}\PY{l+s+s2}{\PYZdq{}} \PY{o}{:=} \PY{n}{Add} \PY{o}{:} \PY{n}{sort\PYZus{}scope}\PY{o}{.}
\end{Verbatim}

  \caption{Syntax}
  \label{code:coq-sort-type}
\end{subfigure}
\begin{subfigure}{0.5\textwidth}
  \begin{Verbatim}[commandchars=\\\{\}]
\PY{k+kn}{Fixpoint} \PY{n}{sortDenote} \PY{o}{(}\PY{n}{a} \PY{o}{:} \PY{n}{sort}\PY{o}{)} \PY{o}{:} \PY{k+kt}{Type} \PY{o}{:=}
  \PY{k}{match} \PY{n}{a} \PY{k}{with}
  \PY{o}{|} \PY{l+m+mi}{𝟶} \PY{o}{=\PYZgt{}} \PY{n}{Empty\PYZus{}set}
  \PY{o}{|} \PY{l+m+mi}{𝟷} \PY{o}{=\PYZgt{}} \PY{n}{unit}
  \PY{o}{|} \PY{n}{s1} \PY{n+nb+bp}{✛} \PY{n}{s2} \PY{o}{=\PYZgt{}} \PY{n+nb+bp}{⟦}\PY{n}{s1}\PY{n+nb+bp}{⟧} \PY{o}{+} \PY{n+nb+bp}{⟦}\PY{n}{s2}\PY{n+nb+bp}{⟧}
  \PY{k}{end}
\PY{n}{where} \PY{l+s+s2}{\PYZdq{}}\PY{l+s+s2}{\PYZsq{}⟦\PYZsq{} x \PYZsq{}⟧\PYZsq{}}\PY{l+s+s2}{\PYZdq{}} \PY{o}{:=} \PY{o}{(}\PY{n}{sortDenote} \PY{n}{x}\PY{o}{)}
  \PY{o}{:} \PY{n}{sort\PYZus{}scope}\PY{o}{.}
\end{Verbatim}

  \caption{Semantics}
  \label{code:coq-sort-sem}
\end{subfigure}
\caption{\code{sort} type and semantics}
\label{code:coq-sort-types}
\end{figure}

% A given \code{sort} value represents the \enquote{shape} of a finite
% set.
%
A given \code{sort} value \code{s} shows how a finite set is built
inductively by means of disjoint union operations over singleton and
empty sets.
% on other finite sets all the way down to the
% singleton and empty sets.
%
% The \code{sort} value is indicative of the order of disjoint union
% operations by which its constituent elements were combined together to
% form the overall set.
A few examples of different ways in which a finite set of cardinality
4 may be formed is shown in Fig.~\ref{fig:sort-shape}, where each
\blackcircle{} represents an element of the set.

\subsection{Representing functions with finite domain}
% A definition of \glspl{NFA} needs the concept of finite sets.
% In this context, we use the term \emph{universe} to refer to the set
% associated with a given sort.
% We need the following constructs for defining \glspl{NFA}:
% \begin{itemize}
% \item Functions with a finite universe as domain.
% \item
%   Subsets of a finite universe, represented as Boolean functions from the
%   finite universe.
% \item 
%   Ability to use functions from finite universes to build functions
%   from bigger universes.
% \end{itemize}

\begin{figure}
  % \documentclass{article}
% \usepackage{forest}
% \usepackage{fontspec}
% \usepackage{subcaption}

% \newcommand\blackcircle{{\fontspec{Symbola}●}}

% \newcommand\utimes{\mathbin{\ooalign{$\cup$\cr%
%    \hfil\raise0.42ex\hbox{$\scriptscriptstyle\times$}\hfil\cr}}}
% \newcommand\bigutimes{\mathop{\ooalign{$\bigcup$\cr%
%    \hfil\raise0.36ex\hbox{$\scriptscriptstyle\boldsymbol{\times}$}\hfil\cr}}}

% \begin{document}

% \begin{figure}

\begin{subfigure}{0.2\textwidth}
  \centering
% balanced
\begin{forest}
[$\uplus$
  [$\uplus$ [\blackcircle] [\blackcircle]]
  [$\uplus$ [\blackcircle] [\blackcircle]]
]
\end{forest}
\caption{(1+1)+(1+1)}
\label{fig:sort-shape-balanced}
\end{subfigure}
\hfill
\begin{subfigure}{0.2\textwidth}
  \centering
% right-left
\begin{forest}
[$\uplus$
  [\blackcircle]
  [$\uplus$
    [$\uplus$ [\blackcircle] [\blackcircle]]
    [\blackcircle]
  ]
]
\end{forest}
\caption{1+((1+1)+1)}
\label{fig:sort-shape-right-left}
\end{subfigure}
\hfill
\begin{subfigure}{0.2\textwidth}
  \centering
% left-right
\begin{forest}
[$\uplus$
  [$\uplus$ [\blackcircle]
     [$\uplus$ [\blackcircle] [\blackcircle]]
  ]
  [\blackcircle]
]
\end{forest}
\caption{(1+(1+1))+1}
\label{fig:sort-shape-left-right}
\end{subfigure}
\hfill
\begin{subfigure}{0.2\textwidth}
  \centering
% left leaning
\begin{forest}
[$\uplus$
  [$\uplus$ 
    [$\uplus$ [\blackcircle] [\blackcircle]]
    [\blackcircle]
  ]
  [\blackcircle]
]
\end{forest}
\caption{((1+1)+1)+1}
\label{fig:sort-shape-left-lean}
\end{subfigure}
% \hfill
% \begin{subfigure}{0.2\textwidth}
%   \centering
% % right leaning
% \begin{forest}
% [$\uplus$ [\blackcircle]
%   [$\uplus$ [\blackcircle]
%     [$\uplus$ [\blackcircle] [\blackcircle]]
%   ]
% ]
% \end{forest}
% \caption{1+(1+(1+1))}
% \label{fig:sort-shape-right-lean}
% \end{subfigure}
% \caption{Different shapes of a finite set with 4 elements}
% \label{fig:sort-shape}
% \end{figure}

% \end{document}

\caption{Few possible structures of a finite set with 4 elements}
\label{fig:sort-shape}
\end{figure}

% Along with finite sets, we also need a representation of functions
% from finite types within Coq.
% This is needed to define transition relation of \glspl{NFA}.
% % functions within Coq.
We use a type \code{func} to represent such functions
(Fig.~\ref{code:vec}).
% This definition satisfies all the requirements listed above.
Values of type \code{func T s} correspond to functions from a finite
type $⟦s⟧$ to a type $T$, where $s$ is a \code{sort} value.
For better readability, we use the notation $s \vto T$ to indicate
the type \code{func T s}.
%
This notation resembles that of the function type and is chosen since
\code{func T s} values are essentially a computation-friendly form of
functions from finite types.
%
% The constructors \code{Zero}, \code{One} and \code{Add} indicate the
% empty function, constant function and the union of two functions,
% respectively.

\begin{figure}
\begin{subfigure}{0.5\textwidth}
  {\small
\begin{Verbatim}[commandchars=\\\{\}]
\PY{k+kn}{Inductive} \PY{n}{func} \PY{o}{(}\PY{n}{A} \PY{o}{:} \PY{k+kt}{Type}\PY{o}{)}
  \PY{o}{:} \PY{n}{sort} \PY{o}{\PYZhy{}\PYZgt{}} \PY{k+kt}{Type} \PY{o}{:=}
\PY{o}{|} \PY{n}{Zero}\PY{o}{:} \PY{l+m+mi}{𝟶}\PY{o}{\PYZpc{}}\PY{n}{sort} \PY{n+nb+bp}{↝} \PY{n}{A}
\PY{o}{|} \PY{n}{One}\PY{o}{:} \PY{n}{A} \PY{o}{\PYZhy{}\PYZgt{}} \PY{l+m+mi}{𝟷}\PY{o}{\PYZpc{}}\PY{n}{sort} \PY{n+nb+bp}{↝} \PY{n}{A}
\PY{o}{|} \PY{n}{Add}\PY{o}{:} \PY{k}{forall} \PY{o}{(}\PY{n}{s1} \PY{n}{s2} \PY{o}{:} \PY{n}{sort}\PY{o}{)}\PY{o}{,}
       \PY{n}{s1} \PY{n+nb+bp}{↝} \PY{n}{A}
    \PY{o}{\PYZhy{}\PYZgt{}} \PY{n}{s2} \PY{n+nb+bp}{↝} \PY{n}{A}
    \PY{o}{\PYZhy{}\PYZgt{}} \PY{o}{(}\PY{n}{s1} \PY{o}{+} \PY{n}{s2}\PY{o}{)}\PY{o}{\PYZpc{}}\PY{n}{sort} \PY{n+nb+bp}{↝} \PY{n}{A}
\PY{n}{where} \PY{l+s+s2}{\PYZdq{}}\PY{l+s+s2}{s ↝ A}\PY{l+s+s2}{\PYZdq{}} \PY{o}{:=} \PY{o}{(}\PY{n}{func} \PY{n}{A} \PY{n}{s}\PY{o}{)} \PY{o}{:} \PY{n}{func\PYZus{}scope}\PY{o}{.}
\PY{n}{Infix} \PY{l+s+s2}{\PYZdq{}}\PY{l+s+s2}{+}\PY{l+s+s2}{\PYZdq{}} \PY{o}{:=} \PY{n}{Add} \PY{o}{:} \PY{n}{func\PYZus{}scope}\PY{o}{.}
\end{Verbatim}
}

  \caption{Syntax}
  \label{code:vec-syntax}
\end{subfigure}
\begin{subfigure}{0.5\textwidth}
  {\small
\begin{Verbatim}[commandchars=\\\{\}]
\PY{k+kn}{Fixpoint} \PY{n}{funcDenote} \PY{o}{\PYZob{}}\PY{n}{A} \PY{o}{:} \PY{k+kt}{Type}\PY{o}{\PYZcb{}} \PY{o}{\PYZob{}}\PY{n}{s} \PY{o}{:} \PY{n}{sort}\PY{o}{\PYZcb{}}
  \PY{o}{(}\PY{n}{v} \PY{o}{:} \PY{o}{(}\PY{n}{s} \PY{n+nb+bp}{↝} \PY{n}{A}\PY{o}{)}\PY{o}{\PYZpc{}}\PY{n}{func}\PY{o}{)} \PY{o}{(}\PY{n}{t} \PY{o}{:} \PY{n+nb+bp}{⟦}\PY{n}{s}\PY{n+nb+bp}{⟧}\PY{o}{)} \PY{o}{:} \PY{n}{A} \PY{o}{:=}
  \PY{k}{match} \PY{n}{v} \PY{k}{with}
  \PY{o}{|} \PY{n}{Zero} \PY{o}{=\PYZgt{}} \PY{k}{fun} \PY{n}{x} \PY{o}{=\PYZgt{}} \PY{k}{match} \PY{n}{x} \PY{k}{with} \PY{k}{end}
  \PY{o}{|} \PY{n}{One} \PY{n}{k} \PY{o}{=\PYZgt{}} \PY{k}{fun} \PY{n}{x} \PY{o}{=\PYZgt{}} \PY{n}{k}
  \PY{o}{|} \PY{n}{Add} \PY{n}{f1} \PY{n}{f2} \PY{o}{=\PYZgt{}}
      \PY{k}{fun} \PY{n}{x} \PY{o}{:} \PY{n+nb+bp}{⟦}\PY{n}{s1} \PY{o}{+} \PY{n}{s2}\PY{n+nb+bp}{⟧} \PY{o}{=\PYZgt{}}
      \PY{k}{match} \PY{n}{x} \PY{k}{with}
      \PY{o}{|} \PY{n}{inl} \PY{n}{x\PYZsq{}} \PY{o}{=\PYZgt{}} \PY{n}{funcDenote} \PY{n}{f1} \PY{n}{x\PYZsq{}}
      \PY{o}{|} \PY{n}{inr} \PY{n}{x\PYZsq{}} \PY{o}{=\PYZgt{}} \PY{n}{funcDenote} \PY{n}{f2} \PY{n}{x\PYZsq{}}
      \PY{k}{end}
  \PY{k}{end} \PY{n}{t}\PY{o}{.}
\end{Verbatim}
}

  \caption{Semantics}
  \label{code:vec-sem}
\end{subfigure}
\caption{Vector type used to represent functions from finite sets}
\label{code:vec}
\end{figure}

Having obtained a way to represent functions from finite sets,
% by means of the \code{func} type,
we can use it to represent subsets of finite sets via their
characteristic functions.
% Finite sets can be represented by means of their characteristic
% function.
%
Let \code{s} be a value of type \code{sort}.
Given a subset $A$ of $⟦s⟧$, the characteristic function $\chi_A$
associated with $A$ is a Boolean function $\chi_A(x)$ that is true
if and only if $x ∈ A$.
%
% The characteristic function of a finite set is a Boolean function
% which tests whether a given element is part of that set or not.
% For example, let \code{s} be a \code{sort} value representing a finite
% set $⟦s⟧$.
% Subsets of the set $⟦s⟧$ can be represented by a function of type $⟦s⟧
% \to \mathbb{B}$.
Since $s \vto \mathbb{B}$ are representations of functions from $⟦s⟧$,
it can double up as the type of characteristic functions of subsets of
$⟦s⟧$.
% Thus, $s \vto \mathbb{B}$ values are a viable means to represent
% subsets of the finite set $⟦s⟧$.
%
% %% This means that a finite set $X$ consisting of elements with type $T$
% %% can be represented by means of a function of type $T \to \mathbb{B}$.
% %% Since $T$ is a finite type, this function can be represented as a
% %% value of type $s \vto \mathbb{B}$, where \code{s} is a \code{sort}
% %% value such that $|s| = |T|$.
% This means that a subset of a finite set $T$ can be represented by means of a
% function having type $T \to \mathbb{B}$.
% % This means that a finite set $X$ consisting of elements with type $T$
% % can be represented by means of a function having type $T \to \mathbb{B}$.
% %
% Since the set $T$ corresponds to a finite type $s$, this function can be
% represented as a value of type $s \vto \mathbb{B}$, where \code{s} is a
% \code{sort} value such that $|s| = |T|$.
% % Therefore, we can represent finite sets and operations on finite sets
% % using \code{func} values and functions over the \code{func} type.
% Therefore, we can represent finite sets using \code{func} values and
% operations on finite sets with functions over the \code{func} type.
% %% A relation $R \subseteq X \times Y$ where $X$ and $Y$ are finite
% %% types, can be represented as a $X \vto Y \vto \mathbb{B}$ value.
% %% Likewise, a predicate on a finite set $X$ corresponds to a value of
% %% type $X \vto \mathbb{B}$.
%
% A predicate on a finite set $X$ corresponds to a value of type $⟦s_X⟧ \vto
% \mathbb{B}$ where $s_X$ is a \code{sort} value corresponding to the
% finite set $X$.
Likewise, a relation on the finite sets $⟦X⟧$ and $⟦Y⟧$, where $X$ and
$Y$ are \code{sort} values, can be represented as $X \vto Y \vto
\mathbb{B}$ value.
% \subseteq X \times Y$ where $X$ and $Y$ are
% finite sets, can be represented as a $⟦s_X⟧ \vto ⟦s_Y⟧ \vto
% \mathbb{B}$ value.
We use this to represent transition function of the \glspl{NFA} that
we build (Section~\ref{sec:repr-nfa}).

We aim to construct \gls{NFA} corresponding to a given \gls{rgx} using
Thompson construction.
This construction needs certain set operations like union and intersection.
We define these operations as functions on $s \vto \mathbb{B}$ where
$s$ is a \code{sort} (Fig.~\ref{code:vec-ops}).
% % In order to do this in terms of 
% % In order to build an \gls{NFA} corresponding to a given \gls{rgx}
% % using Thompson construction in terms of \code{func},
% we need appropriate \code{func} functions corresponding to operations
% on finite sets, like union and intersection.
% % The \code{func} version of the empty set and the singleton set are
% % defined as the functions \code{empty} (Fig.~\ref{code:empty}) and
% % \code{singleton} (Fig.~\ref{code:singleton}) respectively.
% % We define the functions \code{union} and \code{intersection} to
% % represent set union and intersection respectively
% % (Fig.~\ref{code:union-intersect}).
Higher order functions like \code{zipWith}(Fig.~\ref{code:zipwith}) and
\code{fold} (Fig. ~\ref{code:fold}) play an important rule in defining
these operations.
% We define these operations in terms o
% These two functions are defined using a helper function named
% \code{zipWith} (Fig.~\ref{code:zipwith}).
% Similarly, the 'big union' operation is implemented as the function
% \code{bigUnion} (Fig.~\ref{code:bigUnion}) which is defined using a
% \code{fold} function (Fig. ~\ref{code:fold}).
% The operation $\bigcup\limits_{x \in X}^{} f x$ corresponds to
% \code{bigUnion X f}.
A summary of these functions and their corresponding set operations is
given in Table~\ref{tab:set-coq-corr}.

\begin{table}
  \centering
\begin{tabular}{ll}
  \toprule
  Coq                     & Sets                                \\
  \midrule
  \code{empty}            & $\{\}$                              \\
  \code{singleton x}      & $\{x\}$                             \\
  \code{intersection A B} & $A \cap B$                          \\
  \code{union A B}        & $A \cup B$                          \\
  \code{bigUnion X f}     & $\bigcup\limits_{x \in X}^{}\ f\ x$ \\
  \bottomrule
\end{tabular}
\caption{Coq functions and corresponding set operations}
\label{tab:set-coq-corr}
\end{table}

\begin{figure}
\begin{subfigure}{0.5\textwidth}
% \begin{verbatim}
% Fixpoint empty (s: sort): s ↝ bool :=
%   match s with
%   | 0 => Zero
%   | 1 => One false
%   | (s1 ✛ s2)%sort =>
%       (empty s1) ✛ (empty s2)
%   end%vec.
% \end{verbatim}
\begin{Verbatim}[commandchars=\\\{\}]
\PY{k+kn}{Fixpoint} \PY{n}{empty} \PY{o}{(}\PY{n}{s}\PY{o}{:} \PY{n}{sort}\PY{o}{)}\PY{o}{:} \PY{n}{s} \PY{n+nb+bp}{↝} \PY{n}{bool} \PY{o}{:=}
  \PY{k}{match} \PY{n}{s} \PY{k}{with}
  \PY{o}{|} \PY{l+m+mi}{0} \PY{o}{=\PYZgt{}} \PY{n}{Zero}
  \PY{o}{|} \PY{l+m+mi}{1} \PY{o}{=\PYZgt{}} \PY{n}{One} \PY{n+nb+bp}{false}
  \PY{o}{|} \PY{o}{(}\PY{n}{s1} \PY{n+nb+bp}{✛} \PY{n}{s2}\PY{o}{)}\PY{o}{\PYZpc{}}\PY{n}{sort} \PY{o}{=\PYZgt{}}
      \PY{o}{(}\PY{n}{empty} \PY{n}{s1}\PY{o}{)} \PY{n+nb+bp}{✛} \PY{o}{(}\PY{n}{empty} \PY{n}{s2}\PY{o}{)}
  \PY{k}{end}\PY{o}{\PYZpc{}}\PY{n}{vec}\PY{o}{.}
\end{Verbatim}
  \caption{Empty set}
  \label{code:empty}
  \vspace{1cm}
\end{subfigure}
\hfill
\begin{subfigure}{0.5\textwidth}
% \begin{verbatim}
% Fixpoint singleton {s: sort}
%   : ⟦s⟧%sort -> s ↝ bool :=
%   match s with
%   | 0 => fun _ => Zero
%   | 1 => fun _ => One true
%   | (s1 ✛ s2)%sort => fun x =>
%     match x with
%     | inl x' => (singleton x') ✛ (empty s2)
%     | inr x' => (empty s1) ✛ (singleton x')
%     end%vec
%   end.
% \end{verbatim}
\begin{Verbatim}[commandchars=\\\{\}]
\PY{k+kn}{Fixpoint} \PY{n}{singleton} \PY{o}{\PYZob{}}\PY{n}{s}\PY{o}{:} \PY{n}{sort}\PY{o}{\PYZcb{}}
  \PY{o}{:} \PY{n+nb+bp}{⟦}\PY{n}{s}\PY{n+nb+bp}{⟧}\PY{o}{\PYZpc{}}\PY{n}{sort} \PY{o}{\PYZhy{}\PYZgt{}} \PY{n}{s} \PY{n+nb+bp}{↝} \PY{n}{bool} \PY{o}{:=}
  \PY{k}{match} \PY{n}{s} \PY{k}{with}
  \PY{o}{|} \PY{l+m+mi}{0} \PY{o}{=\PYZgt{}} \PY{k}{fun} \PY{o}{\PYZus{}} \PY{o}{=\PYZgt{}} \PY{n}{Zero}
  \PY{o}{|} \PY{l+m+mi}{1} \PY{o}{=\PYZgt{}} \PY{k}{fun} \PY{o}{\PYZus{}} \PY{o}{=\PYZgt{}} \PY{n}{One} \PY{n+nb+bp}{true}
  \PY{o}{|} \PY{o}{(}\PY{n}{s1} \PY{n+nb+bp}{✛} \PY{n}{s2}\PY{o}{)}\PY{o}{\PYZpc{}}\PY{n}{sort} \PY{o}{=\PYZgt{}} \PY{k}{fun} \PY{n}{x} \PY{o}{=\PYZgt{}}
    \PY{k}{match} \PY{n}{x} \PY{k}{with}
    \PY{o}{|} \PY{n}{inl} \PY{n}{x\PYZsq{}} \PY{o}{=\PYZgt{}} \PY{o}{(}\PY{n}{singleton} \PY{n}{x\PYZsq{}}\PY{o}{)} \PY{n+nb+bp}{✛} \PY{o}{(}\PY{n}{empty} \PY{n}{s2}\PY{o}{)}
    \PY{o}{|} \PY{n}{inr} \PY{n}{x\PYZsq{}} \PY{o}{=\PYZgt{}} \PY{o}{(}\PY{n}{empty} \PY{n}{s1}\PY{o}{)} \PY{n+nb+bp}{✛} \PY{o}{(}\PY{n}{singleton} \PY{n}{x\PYZsq{}}\PY{o}{)}
    \PY{k}{end}\PY{o}{\PYZpc{}}\PY{n}{vec}
  \PY{k}{end}\PY{o}{.}
\end{Verbatim}
  \caption{Singleton set}
  \label{code:singleton}
\end{subfigure}

\begin{subfigure}{0.5\textwidth}
% \begin{verbatim}
% Fixpoint zipWith {A B C: Type} {s: sort}
%   (f: A -> B -> C)
%   (fa: s ↝ A) (fb: s ↝ B): s ↝ C :=
% zipWith f Zero Zero := Zero;
% zipWith f (One a) (One b) := One (f a b);
% zipWith f (Add fa1 fa2) (Add fb1 fb2) :=
%   let ff1 := zipWith f fa1 fb1 in
%   let ff2 := zipWith f fa2 fb2 in
%   Add ff1 ff2.
% \end{verbatim}
\begin{Verbatim}[commandchars=\\\{\}]
\PY{k+kn}{Fixpoint} \PY{n}{zipWith} \PY{o}{\PYZob{}}\PY{n}{A} \PY{n}{B} \PY{n}{C}\PY{o}{:} \PY{k+kt}{Type}\PY{o}{\PYZcb{}} \PY{o}{\PYZob{}}\PY{n}{s}\PY{o}{:} \PY{n}{sort}\PY{o}{\PYZcb{}}
  \PY{o}{(}\PY{n}{f}\PY{o}{:} \PY{n}{A} \PY{o}{\PYZhy{}\PYZgt{}} \PY{n}{B} \PY{o}{\PYZhy{}\PYZgt{}} \PY{n}{C}\PY{o}{)}
  \PY{o}{(}\PY{n}{fa}\PY{o}{:} \PY{n}{s} \PY{n+nb+bp}{↝} \PY{n}{A}\PY{o}{)} \PY{o}{(}\PY{n}{fb}\PY{o}{:} \PY{n}{s} \PY{n+nb+bp}{↝} \PY{n}{B}\PY{o}{)}\PY{o}{:} \PY{n}{s} \PY{n+nb+bp}{↝} \PY{n}{C} \PY{o}{:=}
\PY{n}{zipWith} \PY{n}{f} \PY{n}{Zero} \PY{n}{Zero} \PY{o}{:=} \PY{n}{Zero}\PY{o}{;}
\PY{n}{zipWith} \PY{n}{f} \PY{o}{(}\PY{n}{One} \PY{n}{a}\PY{o}{)} \PY{o}{(}\PY{n}{One} \PY{n}{b}\PY{o}{)} \PY{o}{:=} \PY{n}{One} \PY{o}{(}\PY{n}{f} \PY{n}{a} \PY{n}{b}\PY{o}{)}\PY{o}{;}
\PY{n}{zipWith} \PY{n}{f} \PY{o}{(}\PY{n}{Add} \PY{n}{fa1} \PY{n}{fa2}\PY{o}{)} \PY{o}{(}\PY{n}{Add} \PY{n}{fb1} \PY{n}{fb2}\PY{o}{)} \PY{o}{:=}
  \PY{k}{let} \PY{n}{ff1} \PY{o}{:=} \PY{n}{zipWith} \PY{n}{f} \PY{n}{fa1} \PY{n}{fb1} \PY{k}{in}
  \PY{k}{let} \PY{n}{ff2} \PY{o}{:=} \PY{n}{zipWith} \PY{n}{f} \PY{n}{fa2} \PY{n}{fb2} \PY{k}{in}
  \PY{n}{Add} \PY{n}{ff1} \PY{n}{ff2}\PY{o}{.}
\end{Verbatim}
  \caption{Helper function}
  \label{code:zipwith}
  \vspace{1cm}
\end{subfigure}
\hfill
\begin{subfigure}{0.3\textwidth}
% \begin{verbatim}
% Definition union a b :=
%   zipWith orb a b.

% Definition intersection a b :=
%   zipWith andb a b.

% Definition any a :=
%   fold orb false a.
% \end{verbatim}
\begin{Verbatim}[commandchars=\\\{\}]
\PY{k+kn}{Definition} \PY{n}{union} \PY{n}{a} \PY{n}{b} \PY{o}{:=}
  \PY{n}{zipWith} \PY{n}{orb} \PY{n}{a} \PY{n}{b}\PY{o}{.}

\PY{k+kn}{Definition} \PY{n}{intersection} \PY{n}{a} \PY{n}{b} \PY{o}{:=}
  \PY{n}{zipWith} \PY{n}{andb} \PY{n}{a} \PY{n}{b}\PY{o}{.}

\PY{k+kn}{Definition} \PY{n}{any} \PY{n}{a} \PY{o}{:=}
  \PY{k}{fold} \PY{n}{orb} \PY{n+nb+bp}{false} \PY{n}{a}\PY{o}{.}
\end{Verbatim}
  \caption{Union and intersection}
  \label{code:union-intersect}
\end{subfigure}
%
\begin{subfigure}{0.5\textwidth}
% \begin{verbatim}
% Fixpoint fold {s: sort} {A B: Type}
%   (farg: B -> A -> A)
%   (acc: A) (x: s ↝ B) : A :=
% fold farg acc x with x := {
%   | Zero => acc
%   | One b => farg b acc
%   | f1 ✛ f2 =>
%       let acc' := fold farg acc f1 in
%       fold farg acc' f2
% }.
% \end{verbatim}
\begin{Verbatim}[commandchars=\\\{\}]
\PY{k+kn}{Fixpoint} \PY{k}{fold} \PY{o}{\PYZob{}}\PY{n}{s}\PY{o}{:} \PY{n}{sort}\PY{o}{\PYZcb{}} \PY{o}{\PYZob{}}\PY{n}{A} \PY{n}{B}\PY{o}{:} \PY{k+kt}{Type}\PY{o}{\PYZcb{}}
  \PY{o}{(}\PY{n}{farg}\PY{o}{:} \PY{n}{B} \PY{o}{\PYZhy{}\PYZgt{}} \PY{n}{A} \PY{o}{\PYZhy{}\PYZgt{}} \PY{n}{A}\PY{o}{)}
  \PY{o}{(}\PY{n}{acc}\PY{o}{:} \PY{n}{A}\PY{o}{)} \PY{o}{(}\PY{n}{x}\PY{o}{:} \PY{n}{s} \PY{n+nb+bp}{↝} \PY{n}{B}\PY{o}{)} \PY{o}{:} \PY{n}{A} \PY{o}{:=}
\PY{k}{fold} \PY{n}{farg} \PY{n}{acc} \PY{n}{x} \PY{k}{with} \PY{n}{x} \PY{o}{:=} \PY{o}{\PYZob{}}
  \PY{o}{|} \PY{n}{Zero} \PY{o}{=\PYZgt{}} \PY{n}{acc}
  \PY{o}{|} \PY{n}{One} \PY{n}{b} \PY{o}{=\PYZgt{}} \PY{n}{farg} \PY{n}{b} \PY{n}{acc}
  \PY{o}{|} \PY{n}{f1} \PY{n+nb+bp}{✛} \PY{n}{f2} \PY{o}{=\PYZgt{}}
      \PY{k}{let} \PY{n}{acc\PYZsq{}} \PY{o}{:=} \PY{k}{fold} \PY{n}{farg} \PY{n}{acc} \PY{n}{f1} \PY{k}{in}
      \PY{k}{fold} \PY{n}{farg} \PY{n}{acc\PYZsq{}} \PY{n}{f2}
\PY{o}{\PYZcb{}}\PY{o}{.}
\end{Verbatim}
  \caption{Fold operation}
  \label{code:fold}
\end{subfigure}
%
\begin{subfigure}{0.5\textwidth}
% \begin{verbatim}
% Definition bigUnion {s1 s2: sort}
%   (m: s1 ↝ bool) (f: s1 ↝ s2 ↝ bool)
%   : s2 ↝ bool :=
%   let H := zipWith
%     (fun b fs2b =>
%        if b then fs2b else empty _)
%      m f in
%   fold union (empty _) H.
% \end{verbatim}
\begin{Verbatim}[commandchars=\\\{\}]
\PY{k+kn}{Definition} \PY{n}{bigUnion} \PY{o}{\PYZob{}}\PY{n}{s1} \PY{n}{s2}\PY{o}{:} \PY{n}{sort}\PY{o}{\PYZcb{}}
  \PY{o}{(}\PY{n}{m}\PY{o}{:} \PY{n}{s1} \PY{n+nb+bp}{↝} \PY{n}{bool}\PY{o}{)} \PY{o}{(}\PY{n}{f}\PY{o}{:} \PY{n}{s1} \PY{n+nb+bp}{↝} \PY{n}{s2} \PY{n+nb+bp}{↝} \PY{n}{bool}\PY{o}{)}
  \PY{o}{:} \PY{n}{s2} \PY{n+nb+bp}{↝} \PY{n}{bool} \PY{o}{:=}
  \PY{k}{let} \PY{n}{H} \PY{o}{:=} \PY{n}{zipWith}
    \PY{o}{(}\PY{k}{fun} \PY{n}{b} \PY{n}{fs2b} \PY{o}{=\PYZgt{}}
       \PY{k}{if} \PY{n}{b} \PY{k}{then} \PY{n}{fs2b} \PY{k}{else} \PY{n}{empty} \PY{o}{\PYZus{}}\PY{o}{)}
     \PY{n}{m} \PY{n}{f} \PY{k}{in}
  \PY{k}{fold} \PY{n}{union} \PY{o}{(}\PY{n}{empty} \PY{o}{\PYZus{}}\PY{o}{)} \PY{n}{H}\PY{o}{.}
\end{Verbatim}
  \caption{'Big union'}
  \label{code:bigUnion}
\end{subfigure}
  \caption{Set operations in terms of \code{func}}
  \label{code:vec-ops}
\end{figure}


\section{Representing NFAs} \label{sec:repr-nfa}
We represent \glspl{NFA} in Coq using the record type
\code{nfa} (Fig.~\ref{code:nfa}).
A value of type \code{nfa A Q} denotes an \gls{NFA} whose states are
elements of $⟦Q⟧$ and alphabet type is $⟦A⟧$.
The fields \code{start} and \code{final} are subsets of $⟦Q⟧$
representing the set of start and final states respectively.
%
Transition information is maintained as a value \code{trans} of type
$A \vto Q \vto Q \vto \mathbb{B}$ which corresponds to a finite
function of type $⟦A⟧ \to ⟦Q⟧ \to ⟦Q⟧ \to \mathbb{B}$.

We run the \glspl{NFA} by simulating its execution as described in
Section~\ref{sec:nfa-sim-overview}.
Each simulation step consists of considering the elements present in the set
of currently active states with respect to an input symbol and finding
the next set of states as 
illustrated in Fig.~\ref{fig:nfa-next-state-type}.
% A value \code{cur} of type $Q \vto \mathbb{B}$ represents the set of
% currently active states of the \gls{NFA} being simulated.
% The transition function $\delta$ is applied to the input symbol
% \code{a} to obtain the set of transitions from a given state.
% The next set of states with respect to each \gls{NFA} state in
% \code{cur} is then calculated and their union is produced by using the
% \code{bigUnion} function.

\begin{figure}
  \hfill%
  \begin{minipage}{0.7\textwidth}
  \begin{Verbatim}[commandchars=\\\{\}]
\PY{k+kn}{Record} \PY{n}{nfa} \PY{o}{\PYZob{}}\PY{n}{A} \PY{n}{Q} \PY{o}{:} \PY{n}{sort}\PY{o}{\PYZcb{}} \PY{o}{:} \PY{k+kt}{Type} \PY{o}{:=} \PY{o}{\PYZob{}}
  \PY{n}{start}\PY{o}{:} \PY{n}{Q} \PY{n+nb+bp}{↝} \PY{n}{bool}\PY{o}{;}
  \PY{n}{final}\PY{o}{:} \PY{n}{Q} \PY{n+nb+bp}{↝} \PY{n}{bool}\PY{o}{;}
  \PY{n}{trans}\PY{o}{:} \PY{n}{A} \PY{n+nb+bp}{↝} \PY{n}{Q} \PY{n+nb+bp}{↝} \PY{n}{Q} \PY{n+nb+bp}{↝} \PY{n}{bool}\PY{o}{;}
\PY{o}{\PYZcb{}}\PY{o}{.}
\end{Verbatim}

  \end{minipage}%
  \hfill
\caption{NFA type in Coq}
\label{code:nfa}
\end{figure}

\begin{figure}
\begin{mathpar}
\begin{array}{rcl}
       a & : & A \\
     cur & : & Q \vto \mathbb{B} \\
  \delta & : & A \vto Q \vto Q \vto \mathbb{B} \\
bigUnion & : & (s_1 \vto bool) \to
               (s_1 \vto s_2 \vto bool) \to
               (s_2 \vto \mathbb{B}) \\
\hline
  bigUnion\; cur\ (\delta\ a) & : & Q \vto \mathbb{B} \\
\end{array}
\end{mathpar}
\caption{Finding next state of an NFA}
\label{fig:nfa-next-state-type}
\end{figure}

% Though \gls{NFA} simulation gives us a way to run the matcher, there
% is still a problem which we describe in the next section.

\section{Decoupling tests from simulation} \label{sec:c3-ft}
Notice that the alphabet of an \gls{NFA} as described with an
\code{nfa A Q} value is a finite type $⟦A⟧$ rather than the alphabet type
$\Sigma$ of the input words.
This is to facilitate the conversion of the \glspl{NFA} made within Coq
to a form which can be used to produce a corresponding hardware.
% We need to be able to extract the \gls{NFA} data from Coq in a way
% that can be used to build hardware. 

The atoms of the input \glspl{rgx} to our workflow are Boolean tests.
It is possible build corresponding \glspl{NFA} whose transition functions
are defined directly in terms of these tests.
Though such \glspl{NFA} can be extracted out from Coq and be used by
software applications, they are not suitable for conversion to
hardware.
% The input \gls{rgx} consists of atoms over the type \code{Σ → \bbool}.
% Though the components of a corresponding \gls{NFA} of type
% \code{nfa (Σ → \bbool) Q} can be extracted in a way that may be used
% for software, it is not suitable for hardware designs.
%
% The difficulty arises is due to the presence of functions in the
% transition function.
% of the \gls{NFA}, which makes it difficult to
% convert the automata to a corresponding hardware design.
%
However, the transition function of \glspl{NFA} represented using
values of the \code{nfa} type can be seen as tables of Boolean values.
It is straightforward to convert such a representation to hardware.
We now show how the input \gls{rgx} can be converted to a
corresponding \code{nfa} value.

% On the other hand, \gls{NFA} can be implemented as tables of Boolean
% values.
% Values of the \code{nfa A Q} type can be used to express \gls{NFA}
% information in such a form.
% For conversion to hardware, we want \glspl{NFA} to be in this form.

Consider a value \code{r} of type \code{re (Σ → \bbool)} which denotes
a regular expression.
Let $F$ be the set of tests associated with the atoms of the input
\gls{rgx} \code{r}.
Since $F$ is a finite set, its elements
can be mapped to values of a finite type $⟦A⟧$ where \code{A} is a
\code{sort} value.
% The number of atoms in \code{r} is a finite number and hence the tests
% associated with them can be represented as a finite set $⟦A⟧$ where
% $A$ is a \code{sort} value.  
%
Thus, we can use the input \gls{rgx} \code{r} to create a pair
\code{(rA, ft)} where \code{rA} is of type \code{re A} and \code{ft}
is a value of type $A \vto (\Sigma \to \mathbb{B})$ mapping elements
of $⟦A⟧$ back to the values in $F$
(Fig.~\ref{fig:re-ft-nfa-run-overview}).
%
For any \code{rA} and associated \code{ft}, the input \gls{rgx}
\code{r} may be recovered by a simple map operation.
Thus, given an \code{re (Σ → \bbool)} value, we can produce a
corresponding \gls{rgx} whose alphabet is a finite type.
% This \gls{rgx} may then be converted to an \code{nfa} value whose
% alphabet is represented by a \code{sort} value as shown in
% Fig.~\ref{fig:re-ft-nfa-run-overview}.
% By using $ft$, we are able to decouple the input \gls{rgx} from the
% tests present in it.
% This allows us to create \glspl{NFA} as values of the type \code{nfa A
% Q}.

% The input \gls{rgx} is a value of type \code{re (Σ → \bbool)}. 
% However, as the number of tests in a given regular expression $r$ is a
% known finite number, we can think of $r$ as a pair $(rA, ft)$ where
% $rA$ is a \code{re A} value and $ft$ is a $A \vto (\Sigma \to bool)$
% value (Fig.~\ref{fig:re-ft-nfa-run-overview}).
%
The \code{rA} value can be converted to a value of type \code{nfa A Q}
and eventually be converted to hardware % as shown in
Fig.~\ref{fig:delta-ft-yes-no} (See Section~\ref{sec:ft-clash}).
%
Such an \gls{NFA} is simulated with the help of \code{ft}.
Each simulation step involves calculating the next set of states of
the \gls{NFA} as
illustrated in Fig.~\ref{fig:re-ft-nfa-run-detail}.
The operations needed for this simulation (eg: \code{bigUnion}) are
already made available in terms of \code{func}.
%
% The manner in which Coq terms are reduced means that the computation
% to calculate the next state of the \gls{NFA} would essentially be a
% function depending on future inputs rather than being a concrete,
% fully reduced \code{func} value.
% This computation stops at the point where a future input symbol is
% required. A representation of this scenario is shown in
% Fig.~\ref{fig:delta-no-ft}.
% In our experience, the Clash compiler cannot use such a transition
% function to produce corresponding Verilog.
% This is because Clash expects
% the \gls{NFA} state should be a value that can be stored in a
% register.
% Functions correspond to an entire circuit in Clash and hence cannot be
% stored in a register.
% % which is not suited to store functions.
% % This is because functions correspond to an entire circuit instead of
% % specific values.
% %
% We avoid this problem by decoupling the tests from the \gls{NFA}.
Thus through the use of \code{ft}, we are able decouple the transition
function of \glspl{NFA} from tests.
This allows us to produce matchers which can be converted to
synthesizable hardware designs.

\begin{figure}
  \centering
  \begin{subfigure}{0.5\linewidth}
  \centering
  {\small
\begin{Verbatim}[commandchars=\\\{\}]
\PY{k}{fun} \PY{n}{i} \PY{n}{s} \PY{o}{=}
  \PY{k}{if} \PY{o}{(}\PY{n}{f} \PY{n}{i}\PY{o}{)} \PY{k}{then}
    \PY{n}{Add} \PY{o}{(}\PY{n}{Add} \PY{o}{(}\PY{n}{One} \PY{n+nb+bp}{false}\PY{o}{)} \PY{o}{(}\PY{n}{One} \PY{n+nb+bp}{true}\PY{o}{)}\PY{o}{)}
        \PY{o}{(}\PY{n}{Add} \PY{o}{(}\PY{n}{One} \PY{n+nb+bp}{false}\PY{o}{)} \PY{o}{(}\PY{n}{One} \PY{n+nb+bp}{false}\PY{o}{)}\PY{o}{)}\PY{o}{)}
  \PY{k}{else}
    \PY{n}{Add} \PY{o}{(}\PY{n}{Add} \PY{o}{(}\PY{n}{One} \PY{n+nb+bp}{false}\PY{o}{)} \PY{o}{(}\PY{n}{One} \PY{n+nb+bp}{false}\PY{o}{)}\PY{o}{)}
        \PY{o}{(}\PY{n}{Add} \PY{o}{(}\PY{n}{One} \PY{n+nb+bp}{false}\PY{o}{)} \PY{o}{(}\PY{n}{One} \PY{n+nb+bp}{false}\PY{o}{)}\PY{o}{)}\PY{o}{)}
\end{Verbatim}
}
\caption{Without $ft$}
\label{fig:delta-no-ft}
\end{subfigure}%
%
\begin{subfigure}{0.4\linewidth}
  \centering
  {\small
\begin{Verbatim}[commandchars=\\\{\}]
\PY{k}{fun} \PY{n}{a} \PY{n}{s} \PY{o}{=}
  \PY{n}{One}
    \PY{o}{(}\PY{n}{Add}
      \PY{o}{(}\PY{n}{Add} \PY{o}{(}\PY{n}{One} \PY{n+nb+bp}{false}\PY{o}{)} \PY{o}{(}\PY{n}{One} \PY{n+nb+bp}{true}\PY{o}{)}\PY{o}{)}
      \PY{o}{(}\PY{n}{Add} \PY{o}{(}\PY{n}{One} \PY{n+nb+bp}{false}\PY{o}{)} \PY{o}{(}\PY{n}{One} \PY{n+nb+bp}{false}\PY{o}{)}\PY{o}{)}\PY{o}{)}
\end{Verbatim}
}
\caption{With $ft$}
\label{fig:delta-yes-ft}
\end{subfigure}
\caption{Representations of $\delta$ with and without $ft$}
\label{fig:delta-ft-yes-no}
\end{figure}

% Let $F$ be the set of tests associated with the atoms of the input
% \gls{rgx} \code{r}.
% Since $F$ is a finite set, its elements
% can be mapped to values of a finite type $⟦A⟧$ where \code{A} is a
% \code{sort} value.
% Thus, we can use the input \gls{rgx} $r$ to create a pair $(rA, ft)$
% where 
% % to produce a \gls{rgx}
% \code{rA} is of type \code{re A} and $ft$ is a \code{func} value
% mapping elements of $⟦s⟧$ back to the values in $F$. 
% We term the mapping $ft$ as the 
% % We save this bijective mapping as a \code{func} value with type
% % $A \vto (\Sigma \to \mathbb{B})$ that we term \emph{function table}
% % ($ft$).
% \emph{function table}.
% %% The input \gls{rgx} \code{r} may be recovered from a \code{rA} via a
% %% map operation 
% For any \code{rA} and associated $ft$, the input \gls{rgx} \code{r}
% may be recovered by a simple map operation.
% Thus, given an \code{re (Σ → \bbool)} value, we can produce a
% corresponding \gls{rgx} whose alphabet is a finite type.
% This \gls{rgx} may then be converted to an \code{nfa} value whose
% alphabet is represented by a \code{sort} value as shown in
% Fig.~\ref{fig:re-ft-nfa-run-overview}.
% By using $ft$, we are able to decouple the input \gls{rgx} from the
% tests present in it.
% This allows us to create \glspl{NFA} as values of the type \code{nfa A
% Q}.

\begin{figure}
  \centering
  % https://tex.stackexchange.com/questions/461790/how-to-create-two-arrow-breaks
    \begin{tikzpicture}[node distance = 2cm, auto, scale=1.0,
  block/.style = {
    rectangle, draw,  
    text width=7em, text centered, rounded corners, minimum height=2em
  }, 
  arrow/.style = {draw, -latex',rounded corners},
  invisible/.style = {rectangle},
  circular/.style = {draw, circle, radius=1.0cm}
    ]
        \node (r-orig) {\small $r: re\ (\Sigma \to \mathbb{B})$};
        \node [invisible, below of=r-orig] (split) {};
        \node [left of=split] (ft) {\small $ft: A \vto (\Sigma \to \mathbb{B})$};
        \node [right of=split] (rA) {\small $rA: re\ A$};
        \node [circular, below of=ft] (map) {\small \texttt{map}};
        \node [left of=map] (inp) {\small $\sigma : \Sigma$};
        \node [block, below of=rA] (nfa) {\small $n: nfa\ A$};
        \node [right of=nfa, xshift=1cm] (out) {\small Output};

        \path [arrow] (inp) -- (map) edge node[above] {$A \vto \mathbb{B}$} (nfa);
        \path [arrow] (ft) -- (map);
        \path [arrow] (rA) -- (nfa);
        \path [arrow] (nfa) -- (out);
        \path [arrow] (r-orig) |- ($(r-orig)+(0,-1)$) -| (ft);
        \path [arrow] (r-orig) |- ($(r-orig)+(0,-1)$) -|  (rA);
    \end{tikzpicture}

  \caption{Creation of \code{nfa} value from $re\ (\Sigma \to \mathbb{B})$}
  \label{fig:re-ft-nfa-run-overview}
\end{figure}

% % TODO: Semantics of rA and r is same.
% We run resultant \code{nfa} by making use of $ft$.
% The subset of tests from the input \gls{rgx} that return true for
% the current input symbol $\sigma$ is obtained using the function
% table.
% Afterwards, we use the appropriate \code{func} operations to produce
% the next set of active states as shown in
% Fig.~\ref{fig:re-ft-nfa-run-detail}.
% Instead of using the tests from $F$ directly, the function table $ft$
% is used by the \gls{NFA} to process input as it runs.

\begin{figure}
  \centering
      \begin{tikzpicture}[node distance=1.5cm, auto, scale=1.0,
  block/.style = {
    rectangle, draw,
  }, 
  arrow/.style = {draw, -latex',rounded corners},
  invisible/.style = {rectangle},
  circular/.style = {draw, ellipse}
    ]

        \node[block] (ft) {$ft$};
        \node[right of=ft, xshift=1cm, block] (inp) {$\sigma$};
        \node[right of=inp, block] (tfun) {$\delta$};
        \node[right of=tfun, block] (q) {$S_{cur}$};

        \node[circular, xshift=1.25cm, below of=ft] (map) {$map$};
        \node[circular, yshift=-5mm, below of=map] (zw1) {$zipWith$};
        \node[circular, yshift=-5mm, below of=zw1] (u1) {$bigUnion$};
        \node[circular, yshift=-5mm, below of=u1] (zw2) {$zipWith$};
        \node[circular, yshift=-5mm, below of=zw2] (u2) {$bigUnion$};
        \node[below of=u2, yshift=-5mm, block] (q') {$S_{nxt}$};

        \path [arrow] (ft)  |- (map) node[left,  pos=0.25] {\scriptsize $A \vto (\Sigma \to \mathbb{B})$};
        \path [arrow] (inp) |- (map) node[right, pos=0.25] {\scriptsize $\Sigma$};
        \path [arrow] (tfun)  |- (zw1) node[pos=0.65] {\scriptsize $A \vto Q \vto Q \vto \mathbb{B}$};
        \path [arrow] (q) |- (zw2) node[pos=0.75] {\scriptsize $Q \vto \mathbb{B}$};
        %%
        \path [arrow] (map)  edge node[left] {\scriptsize $A \vto \mathbb{B}$} (zw1);
        \path [arrow] (zw1) edge node[left] {\scriptsize $A \vto Q \vto Q \vto \mathbb{B}$} (u1);
        \path [arrow] (u1) edge node[left] {\scriptsize $Q \vto Q \vto \mathbb{B}$} (zw2);
        \path [arrow] (zw2) edge node[left] {\scriptsize $Q \vto Q \vto \mathbb{B}$} (u2);
        \path [arrow] (u2)  edge node[left] {\scriptsize $Q \vto \mathbb{B}$} (q');
    \end{tikzpicture}

  \caption{Running \code{nfa} using $ft$}
  \label{fig:re-ft-nfa-run-detail}
\end{figure}

\section{NFA construction using set operations} \label{sec:nfacon-set}
We use Thompson's algorithm for constructing an \gls{NFA}
corresponding to the input \gls{rgx}.
Section~\ref{sec:thompson-overview} gave a brief description of this
algorithm.
This section shows how the \glspl{NFA} as defined in
Definition~\ref{def:nfa-set} can be constructed using Thompson's
algorithm using sets and set operations.
The \code{nfa} values corresponding to the operations described below
are shown in Fig.~\ref{fig:nfacon-nfa-vals}.

\begin{figure}
  \centering
  \begin{subfigure}{0.5\linewidth}
  \centering
  \begin{tikzpicture}[node distance=1.5cm]
    % Eps
    \node[state, initial, accepting] (s0) {};
  \end{tikzpicture}%
  \caption{eps}
  \label{fa:con-eps}
\end{subfigure}
% \vfill%
\begin{subfigure}{0.5\linewidth}
  \centering
  \begin{tikzpicture}[node distance=1.5cm]
    % Atom
    \node[state, initial] (s00) {};
    \node[state, accepting] (s01)  [right=of s00] {};
    \draw (s00) edge node[above] {$\sigma$} (s01);
  \end{tikzpicture}%
  \caption{atom}
  \label{fa:con-atom}
\end{subfigure}%

  \caption{NFA for base regex}
  \label{fa:con-base}
\end{figure}

% \begin{figure}
%   \centering
%   \input{chapters/pics/fa-con-cat.tex}
%   \caption{NFA for \code{Cat}}
%   \label{fa:con-cat}
% \end{figure}

\begin{figure}
  \centering
  % \documentclass{article}
% \usepackage{tikz}
% \usetikzlibrary{automata} % Import library for drawing automata
% \usetikzlibrary{positioning} % ...positioning nodes
% \usetikzlibrary{arrows} % ...customizing arrows
% \usetikzlibrary{shapes.misc}
% \usetikzlibrary{arrows.meta,
%                 automata,
%                 decorations.markings,
%                 decorations.pathreplacing}

% \begin{document}

\begin{tikzpicture}[node distance=1.5cm]
  % Alt
  \node[state, initial] (s00) {};
  \node (s02) [right=of s00, xshift=-1cm] {$\cdots$};
  \node[state, accepting, right of=s02] (s03) {};

  \draw[->] (s00) edge node[above] {$\sigma_1$} (s02);
  \draw[->] (s02) edge node[above] {$\sigma_n$} (s03);

  \node[below of=s02,scale=2] (altopfa) {$|$};

  \node (s12) [below of=altopfa] {$\cdots$};
  \node[state, initial, xshift=2mm, left of=s12] (s10) {};
  \node[state, accepting, right of=s12] (s13) {};

  \draw[->] (s10) edge node[above] {$\alpha_1$} (s12);
  \draw[->] (s12) edge node[above] {$\alpha_m$} (s13);

  \node[right of=altopfa, xshift=1.5cm, scale=1.5] (eqopfa) {$\Rightarrow$};
  
  \node[state, initial, right of=eqopfa, yshift=8.0mm] (sres00) {};
  \node (dots02) [right of=sres00] {$\cdots$};
  \node[state, accepting, right of=dots02] (sres03) {};

  \node[state, initial, right of=eqopfa, yshift=-8.0mm] (sres10) {};
  \node (dots12) [right of=sres10] {$\cdots$};
  \node[state, accepting, right of=dots12] (sres13) {};

  \draw[->] (sres00) edge node[above] {$\sigma_1$} (dots02);
  \draw[->] (dots02) edge node[above] {$\sigma_n$} (sres03);

  \draw[->] (sres10) edge node[above] {$\alpha_1$} (dots12);
  \draw[->] (dots12) edge node[above] {$\alpha_m$} (sres13);
\end{tikzpicture}

% \end{document}
  \caption{NFA for \code{Alt}}
  \label{fa:con-alt}
\end{figure}

\begin{figure}
  \centering
  % \documentclass{article}
% \usepackage{tikz}
% \usetikzlibrary{automata} % Import library for drawing automata
% \usetikzlibrary{positioning} % ...positioning nodes
% \usetikzlibrary{arrows} % ...customizing arrows
% \usetikzlibrary{shapes.misc}
% \usetikzlibrary{arrows.meta,
%                 automata,
%                 decorations.markings,
%                 decorations.pathreplacing}
% \begin{document}

\begin{tikzpicture}[node distance=1.5cm]

  \node[state, initial] (sres00) {};
  \node (sres02) [right=of sres00, xshift=-1cm] {$\cdots$};
  \node[state, accepting, right of=sres02] (sres03) {};
  \draw (sres03) edge[out=270,
                      in=110,
                      looseness=1.2,
                      bend right,
                      above]
     node (mid) {$\varepsilon$} (sres00);
  \draw[->] (sres00) edge node[below] {$\sigma_1$} (sres02);
  \draw[->] (sres02) edge node[below] {$\sigma_n$} (sres03);

  % Plus
  \node (s02) [above=of mid, yshift=5mm] {$\cdots$};
  \node[state, accepting, right of=s02] (s03) {};
  \node[state, initial, left of=s02] (s00) {};
  \node[below of=s02,scale=1.5] (eqopfa) {$\Downarrow$};

  \draw[->] (s00) edge node[above] {$\sigma_1$} (s02);
  \draw[->] (s02) edge node[above] {$\sigma_n$} (s03);
\end{tikzpicture}

% \end{document}

  \caption{NFA for \code{Plus}}
  \label{fa:con-plus}
\end{figure}

% \begin{definition}
% \label{def:nfa-set-nul}
% An \gls{NFA} $N = (Q, \Sigma, \delta, I, F)$ corresponding to the
% \gls{rgx} $\emptyset$ may be defined as follows:

% \begin{mathpar}
% Q = \emptyset \and
% I = \emptyset \and
% F = \emptyset \and
% \delta(q, \varepsilon) = \emptyset \and
% \delta(q, \sigma) = \emptyset
% \end{mathpar}
% \end{definition}

\begin{definition}
\label{def:nfa-set-eps}
An \gls{NFA} $N = (Q, \Sigma, \delta, I, F)$ corresponding to the
\gls{rgx} $\varepsilon$ may be defined as follows:

\begin{mathpar}
Q = \{init\} \and
I = \{init\} \and
F = \{init\} \and
\delta(q, \varepsilon) = \emptyset \and
\delta(q, \sigma) = \emptyset
\end{mathpar}
\end{definition}

\begin{definition}
\label{def:nfa-set-atom}
An \gls{NFA} $N = (Q, \Sigma, \delta, I, F)$ corresponding to the
\gls{rgx} $c$ may be defined as follows:

\begin{mathpar}
Q = \{init, end\} \and
I = \{init\} \and
F = \{end\} \\
%
\delta(q, \varepsilon) = \emptyset \and
%
\delta(q, \sigma) = 
  \left\{
    \begin{aligned}
      & \{end\}\ if\ \sigma = c\ and\ q = init\\
      & \emptyset\ otherwise \\
    \end{aligned}
  \right.
\end{mathpar}
\end{definition}

\begin{definition}
\label{def:nfa-set-cat}
An \gls{NFA} $N = (Q, \Sigma, \delta, I, F)$ formed as a result of
concatenating two \glspl{NFA} $N_1 = (Q_1, \Sigma, \delta_1, I_1,
F_1)$ and $N_2 = (Q_2, \Sigma, \delta_2, I_2, F_2)$ can be defined
as follows:

\begin{mathpar}
Q = Q_1 \cup Q_2 \and
I = I_1 \and
F = F_2 \\
%
\delta(q, \sigma) = 
  \left\{
    \begin{aligned}
      & \delta_1(q, \sigma)\ if\ q \in Q_1 \\
      & \delta_2(q, \sigma)\ if\ q \in Q_2 \\
    \end{aligned}
  \right.\and
%
\delta(q, \varepsilon) = 
  \left\{
    \begin{aligned}
      & \delta_1(q, \varepsilon)\ \cup I_2\ if\ q \in Q_1\ and\ q \in F_1 \\
      & \delta_2(q, \varepsilon)\ if\ q \in Q_1\ and\ q \notin F_1 \\
      & \delta_2(q, \varepsilon)\ if\ q \in Q_2 \\
    \end{aligned}
  \right. \\
\end{mathpar}
\end{definition}


\begin{definition}
\label{def:nfa-set-alt}
An \gls{NFA} $N = (Q, \Sigma, \delta, I, F)$ formed as a result of a
choice between two \glspl{NFA} $N_1 = (Q_1, \Sigma, \delta_1, I_1,
F_1)$ and $N_2 = (Q_2, \Sigma, \delta_2, I_2, F_2)$ can be defined
as follows:

\begin{mathpar}
Q = Q_1 \cup Q_2 \and
I = I_1 \cup I_2 \and
F = F_1 \cup F_2 \and
\delta(q, \sigma) = 
  \left\{
    \begin{aligned}
      & \delta_1(q, \sigma)\ if\ q \in Q_1 \\
      & \delta_2(q, \sigma)\ if\ q \in Q_2 \\
    \end{aligned}
  \right.
\end{mathpar}
\end{definition}

% \begin{definition}
% \label{def:nfa-set-star}
% An \gls{NFA} $N = (Q, \Sigma, \delta, I, F)$ corresponding to a
% \gls{rgx} $r$, the \gls{NFA} for $r^*$ may be defined as  
% $N_* = (Q_*, \Sigma, \delta_*, Q_{0^*}, F_*)$ as follows:

% \begin{mathpar}
% Q = \{init\} \cup Q \and
% I = \{init\} \and
% F = \{init\} \\
% %
% \delta(q, \sigma) = \delta_*(q, \sigma) \and
% %
% \delta(q, \varepsilon) = 
%   \left\{
%     \begin{aligned}
%       & \{init\}\ if\ q \in F \\
%       & I\ if\ q = init \\
%     \end{aligned}
%   \right. \\
% \end{mathpar}
% \end{definition}

\begin{definition}
\label{def:nfa-set-plus}
Given an \gls{NFA} $N_r = (Q_r, \Sigma, \delta_r, I_r, F_r)$
corresponding to a \gls{rgx} $r$,
the \gls{NFA} for $r^+$ may be defined as  
$N = (Q, \Sigma, \delta, I, F)$ as follows:

\begin{mathpar}
Q = Q_r \and
I = I_r \and
F = F_r \\
%
\delta(q, \sigma) = \delta_r(q, \sigma) \and
%
\delta(q, \varepsilon) = 
  \left\{
    \begin{aligned}
      & \delta_r(q, \varepsilon) \cup I_r\ if\ q \in F_r \\
      & \delta_r(q, \varepsilon)\ otherwise \\
    \end{aligned}
  \right. \\
\end{mathpar}
\end{definition}


\begin{definition}
\label{def:nfa-set-run}
Given an \gls{NFA} $N = (Q, \Sigma, \delta, I, F)$ whose current
state is $cur \in P(Q)$, the next state is $nxt(cur)$ defined as shown
below. A state $cur$ is a final state if $isfinal(cur)$ is true.

\begin{mathpar}
nxt(cur)\ = \bigcup\limits_{s\ \in\ cur}^{} \delta(s, \sigma) \and
isfinal(cur)\ = cur \cap F \neq \emptyset
\end{mathpar}
\end{definition}

\begin{figure}
\begin{minipage}{0.4\textwidth}
% \vspace{3em}
\begin{subfigure}{1.0\textwidth}
  \centering
% \begin{verbatim}
% eps := {|
%    start := singleton tt;
%    final := singleton tt;
%    trans q i := empty _
% |}
% \end{verbatim}
\begin{Verbatim}[commandchars=\\\{\}]
\PY{k+kn}{Definition} \PY{n}{eps} \PY{o}{:=} \PY{o}{\PYZob{}|}
   \PY{n}{start} \PY{o}{:=} \PY{n}{singleton} \PY{n}{tt}\PY{o}{;}
   \PY{n}{final} \PY{o}{:=} \PY{n}{singleton} \PY{n}{tt}\PY{o}{;}
   \PY{n}{trans} \PY{n}{q} \PY{n}{i} \PY{o}{:=} \PY{n}{empty} \PY{o}{\PYZus{}}
\PY{o}{|\PYZcb{}}
\end{Verbatim}
\caption{Eps}
\vspace{3em}
\end{subfigure}
%
\begin{subfigure}[c]{1.0\textwidth}
  \centering
% \begin{verbatim}
% atom f := {|
%    start := singleton (inl tt);
%    final := singleton (inr tt);
%    trans q i :=
%      match q with
%      | inl _ => 
%          if f i then
%            singleton (inr tt)
%          else
%            empty _
%      | _ => empty _
%      end
% |}
% \end{verbatim}
\begin{Verbatim}[commandchars=\\\{\}]
\PY{k+kn}{Definition} \PY{n}{atom} \PY{n}{f} \PY{o}{:=} \PY{o}{\PYZob{}|}
   \PY{n}{start} \PY{o}{:=} \PY{n}{singleton} \PY{o}{(}\PY{n}{inl} \PY{n}{tt}\PY{o}{)}\PY{o}{;}
   \PY{n}{final} \PY{o}{:=} \PY{n}{singleton} \PY{o}{(}\PY{n}{inr} \PY{n}{tt}\PY{o}{)}\PY{o}{;}
   \PY{n}{trans} \PY{n}{q} \PY{n}{i} \PY{o}{:=}
     \PY{k}{match} \PY{n}{q} \PY{k}{with}
     \PY{o}{|} \PY{n}{inl} \PY{o}{\PYZus{}} \PY{o}{=\PYZgt{}}
         \PY{k}{if} \PY{n}{f} \PY{n}{i} \PY{k}{then}
           \PY{n}{singleton} \PY{o}{(}\PY{n}{inr} \PY{n}{tt}\PY{o}{)}
         \PY{k}{else} \PY{n}{empty} \PY{o}{\PYZus{}}
     \PY{o}{|} \PY{o}{\PYZus{}} \PY{o}{=\PYZgt{}} \PY{n}{empty} \PY{o}{\PYZus{}}
     \PY{k}{end}
\PY{o}{|\PYZcb{}}
\end{Verbatim}
\caption{Atom}
\vspace{3em}
\end{subfigure}
%
\begin{subfigure}{1.0\textwidth}
  \centering
% \begin{verbatim}
% plus n := {|
%    start := (start n);
%    final := (final n);
%    trans q i :=
%      let q' := (trans n) q i in
%      if intersection q' (final n)
%        union q' (start n)
%      else q'
% |}
% \end{verbatim}
\begin{Verbatim}[commandchars=\\\{\}]
\PY{k+kn}{Definition} \PY{n}{plus} \PY{n}{n} \PY{o}{:=} \PY{o}{\PYZob{}|}
   \PY{n}{start} \PY{o}{:=} \PY{o}{(}\PY{n}{start} \PY{n}{n}\PY{o}{)}\PY{o}{;}
   \PY{n}{final} \PY{o}{:=} \PY{o}{(}\PY{n}{final} \PY{n}{n}\PY{o}{)}\PY{o}{;}
   \PY{n}{trans} \PY{n}{q} \PY{n}{i} \PY{o}{:=}
     \PY{k}{let} \PY{n}{q\PYZsq{}} \PY{o}{:=} \PY{o}{(}\PY{n}{trans} \PY{n}{n}\PY{o}{)} \PY{n}{q} \PY{n}{i} \PY{k}{in}
     \PY{k}{if} \PY{n}{intersection} \PY{n}{q\PYZsq{}} \PY{o}{(}\PY{n}{final} \PY{n}{n}\PY{o}{)}
       \PY{n}{union} \PY{n}{q\PYZsq{}} \PY{o}{(}\PY{n}{start} \PY{n}{n}\PY{o}{)}
     \PY{k}{else} \PY{n}{q\PYZsq{}}
\PY{o}{|\PYZcb{}}
\end{Verbatim}
\caption{Plus}
\end{subfigure}
\end{minipage}
%
\hspace{3em}
\begin{minipage}{0.6\textwidth}
%
\begin{subfigure}{1.0\textwidth}
  \centering
% \begin{verbatim}
% cat n1 n2 := {|
%    start := union (map inl (start n1))
%       (empty _);
%    final := union (empty _)
%       (map inr (final n2));
%    trans q i :=
%      match q with
%      | inl q1 =>
%          let q1' := (trans n1) q1 i in
%          if intersection q1' (final n1)
%            union q1' (start n2)
%          else q1'
%      | inl q2 => (trans n2) q2 i
%      end
% |}
% \end{verbatim}
\begin{Verbatim}[commandchars=\\\{\}]
\PY{k+kn}{Definition} \PY{n}{cat} \PY{n}{n1} \PY{n}{n2} \PY{o}{:=} \PY{o}{\PYZob{}|}
   \PY{n}{start} \PY{o}{:=} \PY{n}{union} \PY{o}{(}\PY{n}{map} \PY{n}{inl} \PY{o}{(}\PY{n}{start} \PY{n}{n1}\PY{o}{)}\PY{o}{)}
      \PY{o}{(}\PY{n}{empty} \PY{o}{\PYZus{}}\PY{o}{)}\PY{o}{;}
   \PY{n}{final} \PY{o}{:=} \PY{n}{union} \PY{o}{(}\PY{n}{empty} \PY{o}{\PYZus{}}\PY{o}{)}
      \PY{o}{(}\PY{n}{map} \PY{n}{inr} \PY{o}{(}\PY{n}{final} \PY{n}{n2}\PY{o}{)}\PY{o}{)}\PY{o}{;}
   \PY{n}{trans} \PY{n}{q} \PY{n}{i} \PY{o}{:=}
     \PY{k}{match} \PY{n}{q} \PY{k}{with}
     \PY{o}{|} \PY{n}{inl} \PY{n}{q1} \PY{o}{=\PYZgt{}}
         \PY{k}{let} \PY{n}{q1\PYZsq{}} \PY{o}{:=} \PY{o}{(}\PY{n}{trans} \PY{n}{n1}\PY{o}{)} \PY{n}{q1} \PY{n}{i} \PY{k}{in}
         \PY{k}{if} \PY{n}{intersection} \PY{n}{q1\PYZsq{}} \PY{o}{(}\PY{n}{final} \PY{n}{n1}\PY{o}{)}
           \PY{n}{union} \PY{n}{q1\PYZsq{}} \PY{o}{(}\PY{n}{start} \PY{n}{n2}\PY{o}{)}
         \PY{k}{else} \PY{n}{q1\PYZsq{}}
     \PY{o}{|} \PY{n}{inl} \PY{n}{q2} \PY{o}{=\PYZgt{}} \PY{o}{(}\PY{n}{trans} \PY{n}{n2}\PY{o}{)} \PY{n}{q2} \PY{n}{i}
     \PY{k}{end}
\PY{o}{|\PYZcb{}}
\end{Verbatim}
\caption{Cat}
\vspace{3em}
\end{subfigure}
%
\begin{subfigure}{1.0\textwidth}
  \centering
% \begin{verbatim}
% alt n1 n2 := {|
%    start := union (map inl (start n1))
%                   (map inr (start n2));
%    final := union (map inl (final n1))
%                   (map inr (final n2));
%    trans q i :=
%      match q with
%      | inl q => (trans n1) q i
%      | inr q => (trans n2) q i
%      end
% |}
% \end{verbatim}
\begin{Verbatim}[commandchars=\\\{\}]
\PY{k+kn}{Definition} \PY{n}{alt} \PY{n}{n1} \PY{n}{n2} \PY{o}{:=} \PY{o}{\PYZob{}|}
   \PY{n}{start} \PY{o}{:=} \PY{n}{union} \PY{o}{(}\PY{n}{map} \PY{n}{inl} \PY{o}{(}\PY{n}{start} \PY{n}{n1}\PY{o}{)}\PY{o}{)}
                  \PY{o}{(}\PY{n}{map} \PY{n}{inr} \PY{o}{(}\PY{n}{start} \PY{n}{n2}\PY{o}{)}\PY{o}{)}\PY{o}{;}
   \PY{n}{final} \PY{o}{:=} \PY{n}{union} \PY{o}{(}\PY{n}{map} \PY{n}{inl} \PY{o}{(}\PY{n}{final} \PY{n}{n1}\PY{o}{)}\PY{o}{)}
                  \PY{o}{(}\PY{n}{map} \PY{n}{inr} \PY{o}{(}\PY{n}{final} \PY{n}{n2}\PY{o}{)}\PY{o}{)}\PY{o}{;}
   \PY{n}{trans} \PY{n}{q} \PY{n}{i} \PY{o}{:=}
     \PY{k}{match} \PY{n}{q} \PY{k}{with}
     \PY{o}{|} \PY{n}{inl} \PY{n}{q} \PY{o}{=\PYZgt{}} \PY{o}{(}\PY{n}{trans} \PY{n}{n1}\PY{o}{)} \PY{n}{q} \PY{n}{i}
     \PY{o}{|} \PY{n}{inr} \PY{n}{q} \PY{o}{=\PYZgt{}} \PY{o}{(}\PY{n}{trans} \PY{n}{n2}\PY{o}{)} \PY{n}{q} \PY{n}{i}
     \PY{k}{end}
\PY{o}{|\PYZcb{}}
\end{Verbatim}
\caption{Alt}
\end{subfigure}
\end{minipage}
%
\caption{Construction \code{nfa} values with Thompson's algorithm}
\label{fig:nfacon-nfa-vals}
\end{figure}

