\chapter{Preliminaries}\label{ch:prelims}
This Chapter outlines concepts that serve as background information to
the ideas presented in later chapters.

\section{Finite sets}
Finite sets are sets whose cardinality is a finite number.
% The number of elements in such sets is a value less than a specific
% natural number.
% We can have an isomorphism between the set of all natural numbers less
% than this limit and the members of a finite set.
It is possible to enumerate all the elements in a finite set.
By definition, a subset of a finite set is also a finite set.
The sets $\mathbb{Z}_3$ and $\{cyan, magenta, yellow, black\}$ are
both examples of finite sets, whereas $\mathbb{N}$ and $\mathbb{R}$
are not finite sets.            %

\section{Languages}
A language is a collection of words over a finite set of symbols
called \emph{alphabet}.
Words over an alphabet $\Sigma$ are finite sequences of symbols from
$\Sigma$.
Given two words $w_1$ and $w_2$ over the same alphabet $\Sigma$, a new
word $w_1 w_2$ over $\Sigma$ can be formed by concatenating $w_1$ and
$w_2$.
A word $w$ is said to be part of a language $L$, if $w \in \Sigma^*$,
where $\Sigma^*$ is the set of all sequences of symbols from $\Sigma$
with length zero or more.
The length of a word $w$ is the number of symbols present in it and is
denoted using $|w|$.
The word whose length is zero is known as the \emph{empty word} and is
represented with $\varepsilon$.
% The $\varepsilon$ symbol is used to denote the \emph{empty word},
% which is a word whose length is zero.
Languages over an alphabet $\Sigma$ are essentially subsets of
$\Sigma^*$.
A language is said to be empty if there is no word associated with
that language.
The empty language is denoted with $\emptyset$.
Similar to how words can be concatenated, two languages $L_1$ and
$L_2$ can be concatenated to form a new language $L_{12}$ by
concatenating every word $w_1 \in L_1$ with each word $w_2 \in L_2$.
i.e., $L_{12} = \{w_1w_2\ |\ w_1 \in L_1,\ w_2 \in L_2\}$.

\section{Finite automata}
A finite automaton is an abstract representation of a limited form of
computer that is capable of recognizing a class of languages called
\emph{regular languages}.
It consists of a finite set of states among which state transitions
occur in response to input symbols from an alphabet $\Sigma$.
Finite automata can be used to model many kinds of systems, including
software like string tokenizers and hardware like control units of
digital circuits.

% The language accepted by an automaton $A$ is denoted as $L(A)$.
%Finite automata can function as language recognizers.
Every finite automaton has a language associated with it.
A finite automaton $A$ starts running from a state or set of states
marked as initial states.
A given word $w$ is accepted by $A$ if the state in which $A$ ends up
after processing all characters of $w$, is a state that is considered
a final state of $A$.
The language accepted by the automaton $A$ is denoted as $L(A)$.
A word $w$ is accepted by $A$ only if $w \in L(A)$.
% The automaton $A$ starts running from a state or set of states marked
% % as initial states.
% A given word $w$ is accepted by $A$ if $w \in L(A)$.
% In this case $A$ would be in a state designated as a final state by
% the time it finishes processing the last symbol of $w$.
% The input word is rejected by $A$ otherwise.
% Eg: String tokenizers, control units of hardware.

A given finite automaton $A$ is defined in terms of its set of states $Q$,
sets of initial and final states, along with its transition relation
$\delta$ that indicates the transitions which are possible between the
different states with respect to the current input symbol $\sigma$.
%
The language $L(A)$ accepted by $A$ is defined in terms of $\delta^*$,
which is the reflexive transitive closure of $\delta$.
While $\delta$ relates the states in $Q$ with respect to a single
symbol, $\delta^*$ relates the states with respect to an entire word.
The automata $A$ accepts a word $w$ if $F \cap \delta^*(q_0, w) \neq
\emptyset$ where $q_0$ is an initial state and $F$ is the set of all
final states.
% input word $w$ if it ends up in a final state after processing $w$.

\glsreset{NFA}
Depending on the nature of the transition relation $\delta$, finite
automata can be classified into two: \gls{DFA} and \gls{NFA}.
A \gls{DFA} has exactly one next state from a given state with respect
to an input symbol, whereas a \gls{NFA} can have multiple such next
states.

\begin{definition}
\label{def:dfa-set}
A \gls{DFA} $D$ is defined as a quintuple $D = (Q, \Sigma, \delta,
q_0, F)$, where

\begin{itemize}
\item $Q$: Finite set of states
\item $\Sigma$: Finite set of input symbols
\item $\delta: Q \times \Sigma \to Q$: Transition function
\item $q_0 \in Q$: Initial state
\item $F
  \subseteq Q$: Set of final states
\end{itemize}
\end{definition}

\begin{figure}
\begin{mathpar}
  \begin{array}{rcl}
  δ^*(q, ε)   & = & q                                             \\
  δ^*(q, σ:w) & = & δ^*(δ(q,\ σ),\ w)                             \\
              &   &                                               \\
  L(D)        & = & \left\{w\ |\ (F ∩ δ^*(q_0, w)) ≠ ∅ \right\} \\
  \end{array}
\end{mathpar}
\caption{Language accepted by a DFA $D = (Q, \Sigma, \delta, q_0, F)$}
\label{fig:dfa-lang}
\end{figure}

NFAs are a generalization of DFAs whose transition relation δ can
relate the current state to multiple next states for the same input.

\begin{definition}
\label{def:nfa-set}
An \gls{NFA} $N$ is defined as a quintuple $N = (Q, \Sigma, \delta,
Q_{0}, F)$, where

\begin{itemize}
\item $Q$: Finite set of states
\item $\Sigma$: Finite set of input symbols
\item $\delta: Q \times \Sigma \to P(Q)$: Transition function
\item $Q_0 \subseteq Q$: Set of initial states
\item $F \subseteq Q$: Set of final states
\end{itemize}
\end{definition}

The class of languages accepted by both DFAs~(Fig. \ref{fig:dfa-lang})
and NFAs~(Fig. \ref{fig:nfa-lang}) are the same, which is the class of
regular languages.
A language is regular only if there is a finite automata capable of
recognizing it.
\cite{kleene1951regex}

\begin{figure}
\begin{mathpar}
  \begin{array}{rcl}
  δ^*(qs, ε)   & = & qs                                            \\
  δ^*(∅, w)    & = & ∅                                             \\
  δ^*(qs, σ:w) & = & \left\{δ^*(δ(q,σ),\ w)\ |\ q ∈ qs \right\}  \\
               &   &                                               \\
  L(N)         & = & \left\{w\ |\ (F ∩ δ^*(Q_0, w)) ≠ ∅ \right\} \\
  \end{array}
\end{mathpar}
\caption{Language accepted by an NFA $N = (Q, \Sigma, \delta, Q_0, F)$}
\label{fig:nfa-lang}
\end{figure}

% Why non-determinism
Non-determinism allows non-deterministic automata to be described more
concisely than their deterministic counterparts.
This often makes \glspl{NFA} easier to build and understand when
compared to \glspl{DFA}.
For example, many algorithms for DFA construction construct
the corresponding NFA as an intermediate step.

\section{Regular expressions} \label{sec:re-overview}
\Glspl{rgx} are notations representing regular languages.
They are used in a wide variety of applications ranging from matching
patterns in text editors to protein sequencing in computational
biology~\cite{stephens2005oracle}.
\Glspl{rgx} are defined inductively as shown in
Fig~\ref{gram:classic-re-syntax}.
Each \gls{rgx} corresponds to a regular language as shown in
Fig~\ref{gram:classic-re-lang}~\cite{kleene1951regex}.
The empty language and the empty string are represented with
$\emptyset$ and $\varepsilon$ respectively.
\emph{Atoms} indicate the actual characters that are to be matched.
$r_1; r_2$ represents the concatenation of languages $L(r_1)$ and
$L(r_2)$ and $r_1 | r_2$ denotes the union of $L(r_1)$ and $L(r_2)$.
% The language formed by concatenating words in a language $L(r)$
% corresponding to a given \gls{rgx} $r$ zero or more times is
% represented by $r^*$.
The regex corresponding to a language $L(r^*)$ is formed by
concatenating the words matched by a regex $r$ zero or more times is
denoted with $r^*$.
%
The size of a \gls{rgx} $r$ is the number of atoms it contains
and is represented with $|r|$.

\begin{figure}
\begin{subfigure}{0.3\linewidth}
\begin{mathpar}%\scriptsize
\begin{array}{rcll}
r         & :=    & \emptyset                                   \\
          & \Big| & \varepsilon                                 \\
          & \Big| & \sigma                                      \\ 
          & \Big| & r_1;r_2                                     \\ 
          & \Big| & r_1|r_2                                     \\ 
          & \Big| & r^*                                         \\
\end{array}
\end{mathpar}
\caption{Syntax}
\label{gram:classic-re-syntax}
\end{subfigure}%
\hfill%
\begin{subfigure}{0.7\linewidth}
\begin{mathpar}
\begin{array}{rcl}
L(∅)      & =     & \{\}                                        \\
L(ε)      & =     & \{\varepsilon\}                             \\
L(\sigma) & =     & \{c \mid c = \sigma\}                       \\
L(r₁;r₂)  & =     & \{w₁w₂ \mid w₁ ∈ L(r₁) ∧ w₂ ∈ L(r₂)\}       \\
L(r₁|r₂)  & =     & \{w \mid w ∈ L(r₁) ∨ w ∈ L(r₂)\}            \\
L(r^*)    & =     & \{w \mid w ∈ L(r;r^*) ∨ w ∈ L(\varepsilon)\} \\
\end{array}
\end{mathpar}
\caption{Language}
\label{gram:classic-re-lang}
\end{subfigure}
\caption{Classic regex}
\label{fig:rgx}
\end{figure}

\section{State explosion problem} \label{sec:nfa-sim-overview}
We saw that any given \gls{rgx} $r$ has a corresponding finite
automata that accepts the same language as $L(r)$.
A \gls{DFA} corresponding to the \gls{rgx} $r$ can be used to build a
machine runnable matcher capable of checking whether a given
word $w$ is part of $L(r)$.

However, a \gls{rgx}-to-\gls{DFA} conversion results in a state space
explosion.
For a given \gls{rgx} $r$ of size $|r|$, the number of states in a
corresponding \gls{DFA} is $O(2^{|r|})$.
This means that the space required to store the $\delta$ of such a
\gls{DFA} would be in the order of $O(2^{2|r|})$.
Such an exponential blow-up of state count makes running of
\glspl{DFA} for large \glspl{rgx} intractable.
In such a scenario, the transition matrix can get prohibitively huge
for large regexes.
%
In contrast, the state count of the \gls{NFA} corresponding to the
same \gls{rgx} $r$ is in the order of $O(|r|)$.
Though \glspl{NFA} can have multiple states active at the same time,
the set of active states of an \gls{NFA} can be represented with
$O(|r|)$ space.
This is possible by encoding the \gls{NFA} states as bitvectors.
% The set of \gls{NFA} states may be represented as a bitvector, where
Each bit of such a bitvector corresponds to an \gls{NFA}-state.
The set bits indicate the set of currently active states.
Using a bitvector representation, it is possible to represent the
$\delta$ of an \gls{NFA} using $O(|r|^2)$ space.
In this way, the running of a \gls{NFA} can be \emph{simulated} while
keeping the space required to store the transition information of the
finite automaton low.

\begin{mathpar}
  \delta: Q \times \Sigma \to P(Q) \and
  cur \subseteq Q \and
  \sigma \in \Sigma \\
  nxt(cur)\ = \bigcup \bigcup\limits_{q \in cur}^{} \delta(q, \sigma) \and
  isfinal(cur)\ = cur \cap F \neq \emptyset
\end{mathpar}


\section{NFA construction algorithm} \label{sec:thompson-overview}
As we saw earlier, every \gls{rgx} has a corresponding finite automata
capable of accepting words from the language associated with that
\gls{rgx}.
There are many algorithms to generate the finite automata
corresponding to a given \gls{rgx}.
One such algorithm is credited to McNaughton, Yamada and
Thompson~\cite{mcnaughton1960regular,thompson1968nfa} and is commonly
known as Thompson's algorithm.
In this algorithm, the given \gls{rgx} $r$ is converted to a
corresponding \gls{NFA} $N(r)$.
This conversion is performed recursively based on the structure of the
given \gls{rgx}.

\begin{figure}
  \centering
  \begin{subfigure}{0.5\linewidth}
  \centering
  \begin{tikzpicture}[node distance=1.5cm]
    % Nul
    \node[state, initial] (s0) {};
    \node[state, accepting, right of=s0, xshift=2mm] (s1) {};
  \end{tikzpicture}
  \caption{nul}
  \label{fa:thompson-nul}
\end{subfigure}%
\hfill%
\begin{subfigure}{0.5\linewidth}
  \centering
  \begin{tikzpicture}[node distance=1.5cm]
    % Eps
    \node[state, initial] (s0) {};
    \node[state, accepting, right of=s0, xshift=5mm] (s1) {};
    \draw[->] (s0) edge node[above] {$\varepsilon$} (s1);
  \end{tikzpicture}%
  \caption{eps}
  \label{fa:thompson-eps}
\end{subfigure}
\begin{subfigure}{1.0\linewidth}
  \centering
  \begin{tikzpicture}[node distance=1.5cm]
    % Atom
    \node[state, initial] (s00) {};
    \node[state, accepting] (s01)  [right=of s00] {};
    \draw[->] (s00) edge node[above] {$\sigma$} (s01);
  \end{tikzpicture}%
  \caption{atom}
  \label{fa:thompson-atom}
\end{subfigure}%

  \caption{Thompson base}
  \label{fa:thompson-base}
\end{figure}


\begin{figure}
  \centering
  \begin{tikzpicture}[node distance=1.5cm]
  % Cat
  \node[state, initial] (s00) {};
  \node (s02) [right=of s00, xshift=-1cm] {$\cdots$};
  \node[state, accepting, right of=s02] (s03) {};

  \draw[->] (s00) edge node[above] {$\sigma_1$} (s02);
  \draw[->] (s02) edge node[above] {$\sigma_n$} (s03);


  \node[right of=s03,scale=3] (catopfa) {$;$};
  \node[below of=catopfa,scale=1.5] (eqopfa) {$\Downarrow$};

  \node[state, right of=catopfa, initial] (s10) {};
  \node (s12) [right=of s10, xshift=-1cm] {$\cdots$};
  \node[state, accepting, right of=s12] (s13) {};

  \draw[->] (s10) edge node[above] {$\alpha_1$} (s12);
  \draw[->] (s12) edge node[above] {$\alpha_m$} (s13);

  % Cat
  \node[state, initial, below of=s00, xshift=0.3cm, yshift=-1.8cm] (sres00) {};
  \node (sres02) [right=of sres00, xshift=-1cm] {$\cdots$};
  \node[state, right of=sres02] (sres03) {};

  \node[state, right of=sres03, xshift=1cm] (sres10) {};
  \node (sres12) [right=of sres10, xshift=-1cm] {$\cdots$};
  \node[state, accepting, right of=sres12] (sres13) {};


  \draw[->] (sres00) edge node[above] {$\sigma_1$} (sres02);
  \draw[->] (sres02) edge node[above] {$\sigma_n$} (sres03);

  \draw[->] (sres10) edge node[above] {$\alpha_1$} (sres12);
  \draw[->] (sres12) edge node[above] {$\alpha_m$} (sres13);

  \draw[->] (sres03) edge[bend left, above] node[above] (mid) {$\varepsilon$} (sres10);

\end{tikzpicture}

  \caption{Thompson cat}
  \label{fa:thompson-cat}
\end{figure}


\begin{figure}
  \centering
  % \documentclass{article}
% \usepackage{tikz}
% \usetikzlibrary{automata} % Import library for drawing automata
% \usetikzlibrary{positioning} % ...positioning nodes
% \usetikzlibrary{arrows} % ...customizing arrows
% \usetikzlibrary{shapes.misc}
% \usetikzlibrary{arrows.meta,
%                 automata,
%                 decorations.markings,
%                 decorations.pathreplacing}

% \begin{document}

\begin{tikzpicture}[node distance=1.5cm]
  % Alt
  \node[state, initial] (s00) {};
  \node (s02) [right=of s00, xshift=-1cm] {$\cdots$};
  \node[state, accepting, right of=s02] (s03) {};

  \draw[->] (s00) edge node[above] {$\sigma_1$} (s02);
  \draw[->] (s02) edge node[above] {$\sigma_n$} (s03);

  \node[right of=s03,scale=3] (altopfa) {$|$};
  \node[below of=altopfa,scale=1.5] (eqopfa) {$\Downarrow$};

  \node[state, right of=altopfa, initial] (s10) {};
  \node (s12) [right=of s10, xshift=-1cm] {$\cdots$};
  \node[state, accepting, right of=s12] (s13) {};

  \draw[->] (s10) edge node[above] {$\alpha_1$} (s12);
  \draw[->] (s12) edge node[above] {$\alpha_m$} (s13);

  \node[state, initial, below of=altopfa, yshift=-3cm, xshift=-3.1cm] (sres0) {};

  \node[state, right of=sres0, yshift=1.5cm] (sres00) {};
  % \node[state, right of=sres00] (sres01) {s01};
  \node (dots02) [right of=sres00] {$\cdots$};
  \node[state, right of=dots02] (sres03) {};

  \node[state, right of=sres0, yshift=-1.5cm] (sres10) {};
  % \node[state, right of=sres10] (sres11) {s11};
  \node (dots12) [right of=sres10] {$\cdots$};
  \node[state, right of=dots12] (sres13) {};

  \node[state, accepting, right of=sres03, yshift=-1.5cm] (sres1) {};

  % \draw (sres00) edge node {$\sigma$} (sres01);
  \draw[->] (sres00) edge node[above] {$\sigma_1$} (dots02);
  \draw[->] (dots02) edge node[above] {$\sigma_n$} (sres03);

  % \draw (sres10) edge node {$\sigma$} (sres11);
  \draw[->] (sres10) edge node[above] {$\alpha_1$} (dots12);
  \draw[->] (dots12) edge node[above] {$\alpha_m$} (sres13);

  \draw[->] (sres0) edge node[above] {$\varepsilon$} (sres00);
  \draw[->] (sres0) edge node[above] {$\varepsilon$} (sres10);
  \draw[->] (sres03) edge node[above] {$\varepsilon$} (sres1);
  \draw[->] (sres13) edge node[above] {$\varepsilon$} (sres1);
\end{tikzpicture}

% \end{document}

  \caption{Thompson alt}
  \label{fa:thompson-alt}
\end{figure}

\begin{figure}
  \centering
  \begin{tikzpicture}[node distance=1.5cm]

  \node[state] (sres00) {};
  \node (sres02) [right=of sres00, xshift=-1cm] {$\cdots$};
  \node[state, right of=sres02] (sres03) {};
  \node[state, initial, left of=sres00, xshift=-5mm] (start) {};
  \node[state, accepting, right of=sres03, xshift=5mm] (end) {};

  \draw[->] (start) edge node[above] {$\varepsilon$} (sres00);
  \draw[->] (sres03) edge node[above] {$\varepsilon$} (end);
  \draw[->] (sres03) edge[out=270,
                      in=110,
                      looseness=1.2,
                      bend right,
                      above]
     node (mid) {$\varepsilon$} (sres00);
  \draw[->] (start) edge[out=70,
                     in=110,
                     looseness=1,
                     bend right,
                     above]
                node {$\varepsilon$} (end);
  \draw[->] (sres00) edge node[below] {$\alpha_1$} (sres02);
  \draw[->] (sres02) edge node[below] {$\alpha_m$} (sres03);


  % Star
  \node (s02) [above=of mid, yshift=5mm] {$\cdots$};
  \node[state, accepting, right of=s02] (s03) {};
  \node[state, initial, left of=s02] (s00) {};
  \node[below of=s02,scale=1.5] (eqopfa) {$\Downarrow$};

  \draw[->] (s00) edge node[above] {$\sigma_1$} (s02);
  \draw[->] (s02) edge node[above] {$\sigma_n$} (s03);
\end{tikzpicture}

  \caption{Thompson star}
  \label{fa:thompson-star}
\end{figure}
% One such algorithm is credited to McNaughton, Yamada and Thompson
% \cite{mcnaughton1960regular,thompson1968nfa}.
% This algorithm converts a given \gls{rgx} $r$ to a correpsonding
% \gls{NFA} $N(r)$.
%This conversion is performed recursively based on the structure of the
%given \gls{rgx}.


The \gls{NFA} corresponding to the \gls{rgx} $\emptyset$ can be made
as an automaton with two states~\cite{hopcroft2001book}, where one
state is the initial state and the other is the final state
(Fig.~\ref{fa:thompson-nul}).
Since $\emptyset$ represents the empty language, this \gls{NFA} does
not accept any words.
Hence, no transition is possible from the initial state to the final
state.
The \gls{NFA} $N(\varepsilon)$ may also be made as a two-state
automata, but with an $\varepsilon$-transition from the initial state
to the final state (Fig.~\ref{fa:thompson-eps}).
Two states suffice for a \gls{rgx} for the \gls{NFA} $N(\sigma)$
corresponding atoms as well, but the transition from initial to final
state happens only when input symbol $i$ is same as the expected
symbol $\sigma$ (Fig.~\ref{fa:thompson-atom}).
%
Given two \glspl{NFA} $N_1$ and $N_2$ corresponding to two \glspl{rgx}
$r_1$ and $r_2$ respectively, the \gls{NFA} $N_{1;2}$ corresponding to
the \gls{rgx} $r_1;r_2$ can be obtained by adding
$\varepsilon$-transitions from the final states of $N_1$ to the
initial state of $N_2$ (Fig.~\ref{fa:thompson-cat}).
The start states of $N_1$ and final states of $N_2$ would become the
start state and final states respectively of $N_{1;2}$.
Similarly, the \gls{NFA} $N_{1|2}$ for $r_1|r_2$ can be obtained by
making two new states as the start state and final state of $N_{1|2}$
and adding $\varepsilon$-transitions from the start state of $N_{1|2}$
to the start states of both $N_1$ and $N_2$ and from the final states
of $N_1$ and $N_2$ to the final state of $N_{1|2}$
(Fig.~\ref{fa:thompson-alt}).
Like in the case of $r_1|r_2$, given an \gls{NFA} $N$ corresponding to
a \gls{rgx} $r$, the \gls{NFA} $N^*$ associated with the \gls{rgx}
$r^*$ can be made by inserting two new states to function as the start
and final states of $N^*$
(Fig.~\ref{fa:thompson-star}).
We add $\varepsilon$-transitions from the start state of $N^*$ to its
final state, final state of $N$ to its start state, from the start
state of $N^*$ to start state of $N$ and from the final state of $N$
to the final state of $N^*$.
Thus, we can construct an \gls{NFA} $N(r)$ corresponding to a given
\gls{rgx} $r$.

Instead of converting the \gls{NFA} to an equivalent \gls{DFA} to run
the matcher, Thompson's algorithm simulates the \gls{NFA} itself by
keeping track the set of active states as a bitvector.
Thereby avoiding the exponential growth in the transition table size
as was mentioned in Section~\ref{sec:nfa-sim-overview}.

%% \begin{mathpar}
%% q'\ = \bigcup\limits_{q \in cur}^{}
%%       \bigcup\limits_{a \in H}^{}
%%       \delta(a, q)
%% \end{mathpar}

\section{Mealy machines}
The \glspl{DFA} and \glspl{NFA} whose definitions that we saw in
earlier sections are finite state acceptors.
For a given automaton $A$ and an input word $w$, state transitions
occur within $A$ while processing the symbols of $w$, and depending on
the active state of $A$ after processing all the symbols in $w$, we
are able to say whether $w$ was part of $L(A)$ or not.
Note that finite automata by itself do not give any output.
%
Mealy machines~\cite{mealy1955method} are a kind of finite state
machines that generate output as they run.
Unlike finite automata, Mealy machines do not have a final state.
Each transition of a Mealy machine has an output symbol associated
with it.
Definition of a Mealy machine is similar to that of a \gls{DFA} except
for the absence of the notion of final state and the addition of an
output alphabet and output function.

\begin{definition}
\label{def:mealy-set}
A Mealy machine $M$ is defined as a 6-tuple $M = (Q, \Sigma, \Gamma,
\delta, \theta, q_0)$, where

\begin{itemize}
\item $Q$: Finite set of states
\item $\Sigma$: Finite set of input symbols
\item $\Gamma$: Finite set of output symbols
\item $\delta: Q \times \Sigma \to Q$: Transition function
\item $\theta: Q \times \Sigma \to \Gamma$: Output function
\item $q_0 \in Q$: Initial state
\end{itemize}

% With addition of an output function, the transition information of an
% \gls{NFA} built using Thompson's algorithm that we earlier can be used
% to build a Mealy machine.

% TODO: hardware can be modeled as Mealy machines

An \gls{NFA} constructed using Thompson's algorithm that we saw
earlier can be simulated using a Mealy machine.
\end{definition}



\glsreset{NFA}
\section{PSL and SERE}
Correctness is expressed as formal specifications in formal
verification.
Such specifications are written in a formally defined language with unambiguous
semantics.
\gls{PSL} is a language commonly used in the hardware
industry to express correctness properties of circuits.
Syntax of \gls{PSL} properties are shown in Fig.~\ref{fig:sere-syntax}.
There are two styles in which \gls{PSL} properties can be written:
\gls{LTL}-style and \gls{SERE}-style.
The \gls{SERE}-style is more popular among hardware designers as they
are based on a generalized form of \glspl{rgx}.
The term \gls{SERE} refers to these \glspl{rgx}. The atoms of
\glspl{SERE} are over boolean functions from the input alphabet.
\glspl{SERE} with boolean equality on input alphabet correspond to the
classic \glspl{rgx}.
Syntax and semantics of \gls{SERE} are shown in
Fig.~\ref{fig:sere-syntax} and Fig.~\ref{fig:re-sem} respectively.

  \begin{figure}
    \begin{mathpar}
      \begin{array}{rcll}
        r & :=    & \varepsilon & (empty)             \\
          & \Big| & b           & (boolean\ function) \\
          & \Big| & \{r\}       & (grouping)          \\
          & \Big| & r\ ;r       & (concat)            \\
          & \Big| & r | r       & (choice)            \\
          & \Big| & r\ \&\&\ r  & (concurrent\ and)   \\
          & \Big| & r : r       & (fuse)              \\
          & \Big| & r^*         & (star)              \\
      \end{array}
    \end{mathpar}
    \caption{SERE syntax}
    \label{fig:sere-syntax}
  \end{figure}

\begin{figure}
    \begin{mathpar}
    \begin{array}{rclcl}
w & \modelsere & \varepsilon & \iff & w = nil                               \\
w & \modelsere & b           & \iff & |v| = 1 \ \land \ v[0] \modelbool b   \\
w & \modelsere & r_1;r_2       & \iff & \exists w_1 \ w_2, \
                                      w=w_1 \cdot w_2 \ \land \
                                      w_1 \modelsere r_1 \ \land \
                                      w_2 \modelsere r_2                      \\
w & \modelsere & r_1|r_2       & \iff & w \modelsere r_1 \ \lor \
                                      w \modelsere r_2                       \\
w & \modelsere & r_1:r_2       & \iff & \exists w_1 \ a \ w_2, \
                                      w=w_1 \cdot [a] \cdot w_2 \land
                                      w_1 \cdot [a] \modelsere r_1 \land
                                      [a] \cdot w_2 \modelsere r_2            \\
w & \modelsere & r_1\ \&\&\ r_2  & \iff & w \modelsere r_1 \land w \modelsere r_2 \\
w & \modelsere & r^*         & \iff & w \modelsere \varepsilon \  \lor
                                      \exists w_1 w_2, w = w_1 \cdot w_2 \land
                                      w_1 \neq nil \land
                                      w_1 \modelsere r \land
                                      w_2 \modelsere r^*                      \\

    \end{array}
    \end{mathpar}
  \caption{SERE semantics}
  \label{fig:re-sem}
\end{figure}



  % \begin{figure}
  %   % {\scriptsize
  %     \begin{mathpar}
  %       \begin{array}{lcl}
  %         w \models [*0]  & \to & 
  %           w = nil                                            \\
  %         w \models f            & \to & 
  %           (w = [c]) \land (f c = true)                       \\
  %         w \models (r1;r2)      & \to & 
  %           \exists\ w1\ w2, \ (w = w1 \cdot w2) \land (w1 \models r1)
  %                                        \land (w2 \models r2) \\
  %         w \models (r1 \lor r2) & \to & 
  %           (w \models r1) \lor (w \models r2)                 \\
  %         w \models (r1\ \&\& r2) & \to & 
  %           (w \models r1) \land (w \models r2)                 \\

  %         w \models (r1 : r2) & \to & 
  %           \exists\ w1\ w2\ c, \ 
  %              (w = w1 \cdot c \cdot w2) \land
  %              (w1 \cdot c \models r1) \land
  %              (c \cdot w2 \models r2)               \\
  %         w \models r[*]      & \to & 
  %           (w \models [*0]) \lor (w \models (r;r*)) \\
  %       \end{array}
  %     \end{mathpar}
  %     \caption{SERE semantics}
  %     \label{fig:re-sem}
  %   % }
  % \end{figure}

\section{Coq}
Coq is an interactive theorem prover. It allows us to create formal
definitions and reason about them by building formal proofs.
By means of a feature called \emph{extraction}, Coq offers a way to
write programs that are certified correct.
Extraction is the process of converting terms within Coq, which are
written in a functional-style programming language named
\emph{Gallina}, into equivalent terms in a different programming
language.
% specifications and reason about them.
%The notation system allows us to make eDSLs using which.
%
% \subsection{Extraction}
% An attractive feature of Coq is \emph{extraction}.
This is possible due to a deep underlying connection between proofs
and programs~\cite{howard1980formulae}.
The extraction procedure maps parts of a given Coq term which are
relevant for computation to corresponding constructs in the target
programming language while also removing parts that are useful only
for proofs.
It is worth noting that the extraction procedure itself is not
verified correct.
Therefore, any development that involves extraction considers the
extraction procedure as part of a trusted codebase.
% The extraction mechanism of Coq is more developed when comparand
% hence easeirWhen compared to many other theorem provers, the extract
In addition, the parser of Coq is highly customizable, thereby
allowing the creation of \glspl{DSL} within Coq.

% Though the default extraction procedure of Coq allows for the
% conversion of Coq code to general purpose programming languages like
% Haskell and OCaml, there is no built-in support for conversion to an
% \gls{HDL}.

\section{Related work} \label{sec:literature}
Due to the wide range of applications where fast string-based pattern
matching is useful, many approaches of constructing hardware-based
\gls{rgx} matchers have been explored in the past.
Such implementations are primarily made using \glspl{FPGA}.
% \gls{GPU}~\cite{vasiliadis2009regular,yu2013gpu} and their
% combinations~\cite{sidler2017accelerating}.
%% \gls{CPU}, \gls{GPU}, \gls{GPGPU} and \gls{FPGA}.
%
Most of the \gls{FPGA}-based regex matchers use Thompson's
construction~\cite{thompson1968nfa} to build the hardware automata
model corresponding to the input \gls{rgx}.
Usually, a simulation of \gls{NFA} is used for running the matcher as
opposed to directly building the corresponding \gls{DFA}.
This is done to minimize the space needed to store the automata
information, thereby enhancing scalability.

One of the earliest works on \gls{FPGA} implementations of \gls{rgx}
matchers is by \citet{floyd1982compilation}
which maps an \gls{NFA} corresponding to input \gls{rgx} to \gls{PLA}.
%
\citet{sidhu2001fast} gives a method to build matchers on \glspl{FPGA}
using \glspl{NFA}.
\citet{yang2011high} uses a modified form of McNaughton-Yamada
algorithm \cite{mcnaughton1960regular} to map \glspl{NFA} to circuits.
%% capable of matching multiple input symbols at once.
%
%% \citet{yang2011space} uses a form of automata that is midway between
%% \gls{NFA} and \gls{DFA} in terms of complexity to perform
% DBT: O(√n) how?
%% Driven by the need to match multiple \glspl{rgx} at once in \gls{NIDS}
%% applications, multiple approaches have been adopted to increase the
%% number of \glspl{rgx} that can be matched with maximum throughput.
%
%% Works like example is \citet{van2012hardware}.
%
The need for matching multiple \glspl{rgx} at once in Network
Intrusion Detection System
applications led to the development of multiple approaches to increase
the number of \glspl{rgx} that can be matched simultaneously in
addition to maximizing throughput.
%
An example of this is the work by \citet{van2012hardware}, which
allows \gls{rgx} constructs like character classes to be mapped to
hardware more effectively while also enhancing state sharing.



%% A lot of research has gone into processing multiple \glspl{rgx} at
%% once for use in \glspl{NIDS}.
%% In

Dedicated hardware architectures for \gls{rgx} matching also exist.
Automata processor~\cite{micronAP} uses a domain specific architecture
for \gls{rgx} matching.
It is a reconfigurable design onto which transition information of the
automata to be run can be loaded.
% DBT: How is the re to nfa conversion done though?
%
%% Changes in the input automata does not necessitate any change to the
%% hardware design as only the instructions stored in memory need to be
%% changed.
%
CICERO~\cite{cicero} is another architecture that divides string
matching into multiple stages which allows for a certain level of
pipelining.
It uses Thompson's algorithm to convert \gls{rgx} into a set of
instructions which are then processed by hardware.
%
Similar to the case in automata processor, changes in the input
\gls{NFA} do not necessitate any change to CICERO's matching engine,
as only the instructions stored in memory need to be changed.


% - NFA over a smaller DFAs
% - Multiple chars per clock
% - Hueristics
% - ⨯ Use in GPU
% - ✗ Use in CPU
% - Use in CPU/GPU/FPGA hybrid

Multiple formally verified software implementations of \gls{rgx}
matchers are already available.
%% Work has been done on producing certifiably correct \gls{rgx}
%% matchers.
However, none of them produce designs that can directly be
converted to synthesizable hardware designs as they are either
optimized for proofs or for extraction to software.
Reglang~\cite{reglang} uses Coq to formalize and prove a wide range of
properties about \gls{rgx}, finite automata and regular languages
using the mathcomp library~\cite{mathcompbook}.
However, reglang is geared towards convenience with which correctness
proofs may be constructed rather than extractability.
Because of this, \gls{rgx} matchers that can be derived from reglang do not
extract in a manner which can be used to build synthesizable hardware designs.
\citet{firsov2013} uses Agda to create \glspl{NFA} proven to be
capable of matching given \glspl{rgx}.
This work uses linear algebraic operations to construct \gls{NFA}
using Thompson's algorithm.
Though the authors did not attempt to produce equivalent hardware, this
approach is suitable for it.
\citet{noeRegex} introduces a proven correct Coq formalization of
JavaScript \glspl{rgx} which can be extracted to OCaml.
Similarly, \citet{rice2025} provides a linear-time \gls{rgx} matching
algorithm formalized in Coq.
The \gls{rgx} used by all these works is the classic \glspl{rgx} and
assumes an equality check as the operation associated with atoms.
