% ****************************************************************************************************
% classicthesis-config.tex
% formerly known as loadpackages.sty, classicthesis-ldpkg.sty, and classicthesis-preamble.sty
% Use it at the beginning of your ClassicThesis.tex, or as a LaTeX Preamble
% in your ClassicThesis.{tex,lyx} with % ****************************************************************************************************
% classicthesis-config.tex
% formerly known as loadpackages.sty, classicthesis-ldpkg.sty, and classicthesis-preamble.sty
% Use it at the beginning of your ClassicThesis.tex, or as a LaTeX Preamble
% in your ClassicThesis.{tex,lyx} with % ****************************************************************************************************
% classicthesis-config.tex
% formerly known as loadpackages.sty, classicthesis-ldpkg.sty, and classicthesis-preamble.sty
% Use it at the beginning of your ClassicThesis.tex, or as a LaTeX Preamble
% in your ClassicThesis.{tex,lyx} with % ****************************************************************************************************
% classicthesis-config.tex
% formerly known as loadpackages.sty, classicthesis-ldpkg.sty, and classicthesis-preamble.sty
% Use it at the beginning of your ClassicThesis.tex, or as a LaTeX Preamble
% in your ClassicThesis.{tex,lyx} with \input{classicthesis-config}
% ****************************************************************************************************
% If you like the classicthesis, then I would appreciate a postcard.
% My address can be found in the file ClassicThesis.pdf. A collection
% of the postcards I received so far is available online at
% http://postcards.miede.de
% ****************************************************************************************************

\RequirePackage{silence} % :-\suppress an unnecessary warning in compilation
    \WarningFilter{scrreprt}{Usage of package `titlesec'}
    \WarningFilter{titlesec}{Non standard sectioning command detected}
    \WarningFilter{hyperref}{Token not allowed in a PDF string (PDFDocEncoding)}
   
% ****************************************************************************************************
% 0. Set the encoding of your files. UTF-8 is the only sensible encoding nowadays. If you can't read
% äöüßáéçèê∂åëæƒÏ€ then change the encoding setting in your editor, not the line below. If your editor
% does not support utf8 use another editor!
% ****************************************************************************************************
\PassOptionsToPackage{utf8}{inputenc}
  \usepackage{inputenc}

\PassOptionsToPackage{T1}{fontenc} % T2A for cyrillics
  \usepackage{fontenc}

\PassOptionsToPackage{final}{microtype}  %iitpkd
\usepackage[final]{microtype} %iitpkd

\usepackage{etoolbox}

\emergencystretch=1em

% *******************************************************************************************************************
% Note : Do not edit above this line
% Note : Do not edit options to classicthesis given below except the options eulermath, drafting, printready and numsupervisors
% % 
% *******************************************************************************************************************
\PassOptionsToPackage{
   tocaligned=false, 
  dottedtoc=true,
  eulerchapternumbers=true, 
  linedheaders=false,  
  eulermath=true, %true, %false %If false is set, cmmodern font will be used for math. % See chapter 3 of the template thesis to see the difference. %If text font is changed, be careful about this option.  
  beramono=true,   
  palatino=true,    
  style=classicthesis, 
  floatperchapter=true,     % numbering per chapter for all floats (i.e., Figure 1.1)
  numsupervisors=one, %one %two %three %set according to your number of supervisors
  %drafting=true,    % use this for draft submitted with synopsis. comment this line in the thesis for review
  drafting=false, % for final version this line is to be used
  printready=false, %Enable this for making title page and hyperreflinks in colour for screen reading
  %printready=true,  %Enable this for making title page and hyperreflinks in black for printing, margin adjustments
 }{classicthesis}


% ****************************************************************************************************
% 2. Personal data and user ad-hoc commands (insert your own data here)
% ****************************************************************************************************

\newcommand{\myTitle}{Towards verified regular expression matchers in hardware\xspace}
%\newcommand{\mySubtitle}{Subtitle of Your Thesis\xspace}
%\newcommand{\myDegree}{Doctor of Philosophy\xspace}
\newcommand{\myDegree}{Master of Science \xspace}
%\newcommand{\myDegree}{Master of Technology \xspace}
%\newcommand{\myDegree}{Master of Science in Engineering \xspace}
\newcommand{\myName}{Julin Shaji\xspace}
\newcommand{\myRollNo}{112103001\xspace}
%\newcommand{\myGender}{her\xspace}
\newcommand{\myGender}{him\xspace}
\newcommand{\myDepartment}{Department of Computer Science and Engineering\xspace}

\newcommand{\mySupervisorOne}{Dr. Piyush P. Kurur}
\newcommand{\mySupervisorOneDesig}{Associate Professor}%Example Assistant Professor
\newcommand{\mySupervisorOneDept}{Dept.~of Computer Science and Engg} %Example Dept.~of Computer Science and Engg.
\newcommand{\mySupervisorOneInstitute}{IIT Palakkad}% Example IIT Palakkad

% \newcommand{\mySupervisorTwo}{<Name of Supervisor2>}%used only if necessary
% \newcommand{\mySupervisorTwoDesig}{<Supervisor2 Designation>}%Example Assistant Professor
% \newcommand{\mySupervisorTwoDept}{<Supervisor2 Department>} %Example Dept.~of Computer Science and Engg.
% \newcommand{\mySupervisorTwoInstitute}{<Supervisor2 Institute>}% Example IIT Palakkad

\newcommand{\myUni}{Indian Institute of Technology Palakkad\xspace}
\newcommand{\myLocation}{Palakkad\xspace}
\newcommand{\myTime}{May 2025\xspace}
\newcommand{\myVersion}{\classicthesis}
\newcommand{\myDCMemberOne}{Dr. Unnikrishnan C.\xspace}
\newcommand{\myDCMemberOneDesigAndDept}{Assistant Professor, .Dept.~of Computer Science and Engineering\xspace}%Example: Assistant Professor, .Dept.~of Computer Science and Engineering
\newcommand{\myDCMemberOneInstitute}{IIT Palakkad}
%
\newcommand{\myDCMemberTwo}{Dr. Sandeep Chandran}
\newcommand{\myDCMemberTwoDesigAndDept}{Assistant Professor, .Dept.~of Computer Science and Engineering}
\newcommand{\myDCMemberTwoInstitute}{IIT Palakkad}
%
\newcommand{\myDCMemberThree}{Dr. Subrahmanyam Mula}
\newcommand{\myDCMemberThreeDesigAndDept}{Assistant Professor, .Dept.~of Electrical Engineering}
\newcommand{\myDCMemberThreeInstitute}{IIT Palakkad}

% ********************************************************************
% Setup, finetuning, and useful commands
% ********************************************************************
\providecommand{\mLyX}{L\kern-.1667em\lower.25em\hbox{Y}\kern-.125emX\@}
\newcommand{\ie}{i.\,e.}
\newcommand{\Ie}{I.\,e.}
\newcommand{\eg}{e.\,g.}
\newcommand{\Eg}{E.\,g.}
% ****************************************************************************************************


% ****************************************************************************************************
% 3. Loading some handy packages
% ****************************************************************************************************
% ********************************************************************
% Packages with options that might require adjustments
% ********************************************************************
\PassOptionsToPackage{ngerman,american}{babel} % change this to your language(s), main language last
\usepackage{babel}

\usepackage{csquotes}
%%%%%You may change the reference format to any of the options given agaist the parameters. 
%%%%%It is suggested that you maintain compatibility with natbib.
\PassOptionsToPackage{%
  %backend=biber,bibencoding=utf8, %instead of bibtex
  %backend=bibtex8,bibencoding=ascii,%
  language=auto,%
  style=numeric-comp,%
  %style=authoryear-comp, % Author 1999, 2010
  %bibstyle=authoryear,dashed=false, % dashed: substitute rep. author with ---
  sorting=nyt, % name, year, title
  maxbibnames=10, % default: 3, et al.
  backend=biber,
  %backref=true,%
  natbib=true % natbib compatibility mode (\citep and \citet still work)
}{biblatex}
\usepackage{biblatex}

\PassOptionsToPackage{fleqn}{amsmath}       % math environments and more by the AMS
   \usepackage{amsmath}

\usepackage{amssymb}
\usepackage{wasysym}
\usepackage{amsthm} 
\usepackage{mathrsfs}
\usepackage{mathtools}
\usepackage{epsfig}
%%%%Theorem like environments follow sectionwise numbering, 
%%%%only exceptions being 'Remark' and 'Note'.
%%%%Do not edit this basic setting, if you are adding more such environments.
\theoremstyle{plain}% default
\newtheorem{theorem}{Theorem}[section]
\newtheorem{lemma}[theorem]{Lemma}
\newtheorem{proposition}[theorem]{Proposition}
\newtheorem{corollary}[theorem]{Corollary}
\newtheorem{property}[theorem]{Property}

\theoremstyle{definition}
\newtheorem{definition}[theorem]{Definition}
\newtheorem{conjecture}[theorem]{Conjecture}
\newtheorem{example}[theorem]{Example}
\newtheorem{observation}[theorem]{Observation}

\theoremstyle{remark}
\newtheorem*{remark}{Remark}
\newtheorem*{note}{Note}
\newtheorem{claim}{Claim}[chapter]

\usepackage{thm-restate}
\usepackage[ruled,vlined,algochapter]{algorithm2e}

% ********************************************************************
% General useful packages
% ********************************************************************
\usepackage{graphicx} %

% \usepackage[newfloat]{minted}
% \newmintinline{haskell}{breaklines}

\usepackage{scrhack} % fix warnings when using KOMA with listings package
\usepackage{xspace} % to get the spacing after macros right
\PassOptionsToPackage{printonlyused,smaller}{acronym}
  \usepackage{acronym} % nice macros for handling all acronyms in the thesis
  \def\bflabel#1{{\acsfont{#1}\hfill}}
  \def\aclabelfont#1{\acsfont{#1}}
% ****************************************************************************************************
%\usepackage{pgfplots} % External TikZ/PGF support (thanks to Andreas Nautsch)
%\usetikzlibrary{external}
%\tikzexternalize[mode=list and make, prefix=ext-tikz/]
% ****************************************************************************************************




% ****************************************************************************************************
% 4. Setup floats: tables, (sub)figures, and captions
%%%Do not edit this setting
% ****************************************************************************************************
\usepackage{tabularx} % better tables
  \setlength{\extrarowheight}{3pt} % increase table row height
\newcommand{\tableheadline}[1]{\multicolumn{1}{l}{\spacedlowsmallcaps{#1}}}
\newcommand{\myfloatalign}{\centering} % to be used with each float for alignment
%\usepackage{subfig}
\usepackage{caption}
\usepackage{subcaption}



% ****************************************************************************************************


% ****************************************************************************************************
% 5. Setup code listings
%%Do not edit this setting
% ****************************************************************************************************
\usepackage{listings}
%\lstset{emph={trueIndex,root},emphstyle=\color{BlueViolet}}%\underbar} % for special keywords
\lstset{language=[LaTeX]Tex,%C++,
  morekeywords={PassOptionsToPackage,selectlanguage},
  keywordstyle=\color{CTkeyword},%\bfseries,
  basicstyle=\small\ttfamily,
  %identifierstyle=\color{NavyBlue},
  commentstyle=\color{CTcomment}\ttfamily,
  stringstyle=\rmfamily,
  numbers=none,%left,%
  numberstyle=\scriptsize,%\tiny
  stepnumber=5,
  numbersep=8pt,
  showstringspaces=false,
  breaklines=true,
  %frameround=ftff,
  %frame=single,
  belowcaptionskip=.75\baselineskip
  %frame=L
}
% ****************************************************************************************************

%%%%Note that for list of symbols to be inluded the following commands needs to be executed
%%%%pdflatex ClassicThesis.tex
%%%%makeindex ClassicThesis.nlo -s nomencl.ist -o ClassicThesis.nls
%%%%pdflatex ClassicThesis.tex
\usepackage{nomencl}
\makenomenclature
\renewcommand{\nomname}{List of Symbols}
\setlength{\nomlabelwidth}{2.5cm}
% ****************************************************************************************************
% 6. Last calls before the bar closes
% ****************************************************************************************************
% ********************************************************************
% Her Majesty herself
% ********************************************************************
\usepackage{classicthesis}


% ********************************************************************
% Fine-tune hyperreferences (hyperref should be called last)
%%%Do not edit this setting
% ********************************************************************
\hypersetup{%
  colorlinks=true, linktocpage=true, pdfstartpage=3, pdfstartview=FitV,%
  breaklinks=true, pageanchor=true,%
  pdfpagemode=UseNone, %
  plainpages=false, bookmarksnumbered, bookmarksopen=true, bookmarksopenlevel=1,%
  hypertexnames=true, pdfhighlight=/O,%
  urlcolor=CTurl, linkcolor=CTlink, citecolor=CTcitation, %
  pdftitle={\myTitle},%
  pdfauthor={\textcopyright\ \myName, \myUni},%
  pdfsubject={},%
  pdfkeywords={},%
  pdfcreator={xelatex},%
  %pdfcreator={pdfLaTeX},%
  %pdfproducer={LaTeX with hyperref and classicthesis}%
}

%%%%%%%%%Do not edit below this line except the very last line of this file %%%%%%%%%%%%%%%%%%%%%%%%%%%%%%%%%

% ********************************************************************
% Setup autoreferences (hyperref and babel)
% ********************************************************************
% There are some issues regarding autorefnames
% http://www.tex.ac.uk/cgi-bin/texfaq2html?label=latexwords
% you have to redefine the macros for the
% language you use, e.g., american, ngerman
% (as chosen when loading babel/AtBeginDocument)
% ********************************************************************
\makeatletter
\@ifpackageloaded{babel}%
  {%
    \addto\extrasamerican{%
      \renewcommand*{\figureautorefname}{Figure}%
      \renewcommand*{\tableautorefname}{Table}%
      \renewcommand*{\partautorefname}{Part}%
      \renewcommand*{\chapterautorefname}{Chapter}%
      \renewcommand*{\sectionautorefname}{Section}%
      \renewcommand*{\subsectionautorefname}{Section}%
      \renewcommand*{\subsubsectionautorefname}{Section}%
    }%
    \addto\extrasngerman{%
      \renewcommand*{\paragraphautorefname}{Absatz}%
      \renewcommand*{\subparagraphautorefname}{Unterabsatz}%
      \renewcommand*{\footnoteautorefname}{Fu\"snote}%
      \renewcommand*{\FancyVerbLineautorefname}{Zeile}%
      \renewcommand*{\theoremautorefname}{Theorem}%
      \renewcommand*{\appendixautorefname}{Anhang}%
      \renewcommand*{\equationautorefname}{Gleichung}%
      \renewcommand*{\itemautorefname}{Punkt}%
    }%
      % Fix to getting autorefs for subfigures right (thanks to Belinda Vogt for changing the definition)
      \providecommand{\subfigureautorefname}{\figureautorefname}%
    }{\relax}
\makeatother

\listfiles
\ifthenelse{\boolean{ct@printready}}%% line spread is kept larger for screen view option, because margin space is less
{
  \linespread{1.1} % this is for print ready
}
{
   \linespread{1.3} % increased from 1.1 to 1.3 in iitpkd v2.1 for screeview version
}   
%%%%%
%%% \renewcommand{\rmdefault}{pplx} %%uncomment this line if you want to change number typesetting style

%%%%%%%%%%%%%%%%%%%%%%%%%%%%%%%%%%%%%%%%%%%%%%%%%%%%%%%%%%%%%%%%%%%%%%%%%%%%%
% CUSTOM
%%%%%%%%%%%%%%%%%%%%%%%%%%%%%%%%%%%%%%%%%%%%%%%%%%%%%%%%%%%%%%%%%%%%%%%%%%%%%

\usepackage[acronym,shortcuts]{glossaries}
%
% https://tex.stackexchange.com/questions/98494/glossaries-dont-print-single-occurences
\glsenableentrycount % enable \cgls, \cglspl, \cGls, \cGlspl
\let\ac\cgls
\let\acpl\cglspl
\let\Ac\cGls
\let\Acpl\cGlspl
%
%\newglossary{symbols}{sym}{sbl}{List of Symbols}
\makeglossaries

\usepackage{nomencl}
\makenomenclature

\input{config.tex}

% ****************************************************************************************************
% If you like the classicthesis, then I would appreciate a postcard.
% My address can be found in the file ClassicThesis.pdf. A collection
% of the postcards I received so far is available online at
% http://postcards.miede.de
% ****************************************************************************************************

\RequirePackage{silence} % :-\suppress an unnecessary warning in compilation
    \WarningFilter{scrreprt}{Usage of package `titlesec'}
    \WarningFilter{titlesec}{Non standard sectioning command detected}
    \WarningFilter{hyperref}{Token not allowed in a PDF string (PDFDocEncoding)}
   
% ****************************************************************************************************
% 0. Set the encoding of your files. UTF-8 is the only sensible encoding nowadays. If you can't read
% äöüßáéçèê∂åëæƒÏ€ then change the encoding setting in your editor, not the line below. If your editor
% does not support utf8 use another editor!
% ****************************************************************************************************
\PassOptionsToPackage{utf8}{inputenc}
  \usepackage{inputenc}

\PassOptionsToPackage{T1}{fontenc} % T2A for cyrillics
  \usepackage{fontenc}

\PassOptionsToPackage{final}{microtype}  %iitpkd
\usepackage[final]{microtype} %iitpkd

\usepackage{etoolbox}

\emergencystretch=1em

% *******************************************************************************************************************
% Note : Do not edit above this line
% Note : Do not edit options to classicthesis given below except the options eulermath, drafting, printready and numsupervisors
% % 
% *******************************************************************************************************************
\PassOptionsToPackage{
   tocaligned=false, 
  dottedtoc=true,
  eulerchapternumbers=true, 
  linedheaders=false,  
  eulermath=true, %true, %false %If false is set, cmmodern font will be used for math. % See chapter 3 of the template thesis to see the difference. %If text font is changed, be careful about this option.  
  beramono=true,   
  palatino=true,    
  style=classicthesis, 
  floatperchapter=true,     % numbering per chapter for all floats (i.e., Figure 1.1)
  numsupervisors=one, %one %two %three %set according to your number of supervisors
  %drafting=true,    % use this for draft submitted with synopsis. comment this line in the thesis for review
  drafting=false, % for final version this line is to be used
  printready=false, %Enable this for making title page and hyperreflinks in colour for screen reading
  %printready=true,  %Enable this for making title page and hyperreflinks in black for printing, margin adjustments
 }{classicthesis}


% ****************************************************************************************************
% 2. Personal data and user ad-hoc commands (insert your own data here)
% ****************************************************************************************************

\newcommand{\myTitle}{Towards verified regular expression matchers in hardware\xspace}
%\newcommand{\mySubtitle}{Subtitle of Your Thesis\xspace}
%\newcommand{\myDegree}{Doctor of Philosophy\xspace}
\newcommand{\myDegree}{Master of Science \xspace}
%\newcommand{\myDegree}{Master of Technology \xspace}
%\newcommand{\myDegree}{Master of Science in Engineering \xspace}
\newcommand{\myName}{Julin Shaji\xspace}
\newcommand{\myRollNo}{112103001\xspace}
%\newcommand{\myGender}{her\xspace}
\newcommand{\myGender}{him\xspace}
\newcommand{\myDepartment}{Department of Computer Science and Engineering\xspace}

\newcommand{\mySupervisorOne}{Dr. Piyush P. Kurur}
\newcommand{\mySupervisorOneDesig}{Associate Professor}%Example Assistant Professor
\newcommand{\mySupervisorOneDept}{Dept.~of Computer Science and Engg} %Example Dept.~of Computer Science and Engg.
\newcommand{\mySupervisorOneInstitute}{IIT Palakkad}% Example IIT Palakkad

% \newcommand{\mySupervisorTwo}{<Name of Supervisor2>}%used only if necessary
% \newcommand{\mySupervisorTwoDesig}{<Supervisor2 Designation>}%Example Assistant Professor
% \newcommand{\mySupervisorTwoDept}{<Supervisor2 Department>} %Example Dept.~of Computer Science and Engg.
% \newcommand{\mySupervisorTwoInstitute}{<Supervisor2 Institute>}% Example IIT Palakkad

\newcommand{\myUni}{Indian Institute of Technology Palakkad\xspace}
\newcommand{\myLocation}{Palakkad\xspace}
\newcommand{\myTime}{May 2025\xspace}
\newcommand{\myVersion}{\classicthesis}
\newcommand{\myDCMemberOne}{Dr. Unnikrishnan C.\xspace}
\newcommand{\myDCMemberOneDesigAndDept}{Assistant Professor, .Dept.~of Computer Science and Engineering\xspace}%Example: Assistant Professor, .Dept.~of Computer Science and Engineering
\newcommand{\myDCMemberOneInstitute}{IIT Palakkad}
%
\newcommand{\myDCMemberTwo}{Dr. Sandeep Chandran}
\newcommand{\myDCMemberTwoDesigAndDept}{Assistant Professor, .Dept.~of Computer Science and Engineering}
\newcommand{\myDCMemberTwoInstitute}{IIT Palakkad}
%
\newcommand{\myDCMemberThree}{Dr. Subrahmanyam Mula}
\newcommand{\myDCMemberThreeDesigAndDept}{Assistant Professor, .Dept.~of Electrical Engineering}
\newcommand{\myDCMemberThreeInstitute}{IIT Palakkad}

% ********************************************************************
% Setup, finetuning, and useful commands
% ********************************************************************
\providecommand{\mLyX}{L\kern-.1667em\lower.25em\hbox{Y}\kern-.125emX\@}
\newcommand{\ie}{i.\,e.}
\newcommand{\Ie}{I.\,e.}
\newcommand{\eg}{e.\,g.}
\newcommand{\Eg}{E.\,g.}
% ****************************************************************************************************


% ****************************************************************************************************
% 3. Loading some handy packages
% ****************************************************************************************************
% ********************************************************************
% Packages with options that might require adjustments
% ********************************************************************
\PassOptionsToPackage{ngerman,american}{babel} % change this to your language(s), main language last
\usepackage{babel}

\usepackage{csquotes}
%%%%%You may change the reference format to any of the options given agaist the parameters. 
%%%%%It is suggested that you maintain compatibility with natbib.
\PassOptionsToPackage{%
  %backend=biber,bibencoding=utf8, %instead of bibtex
  %backend=bibtex8,bibencoding=ascii,%
  language=auto,%
  style=numeric-comp,%
  %style=authoryear-comp, % Author 1999, 2010
  %bibstyle=authoryear,dashed=false, % dashed: substitute rep. author with ---
  sorting=nyt, % name, year, title
  maxbibnames=10, % default: 3, et al.
  backend=biber,
  %backref=true,%
  natbib=true % natbib compatibility mode (\citep and \citet still work)
}{biblatex}
\usepackage{biblatex}

\PassOptionsToPackage{fleqn}{amsmath}       % math environments and more by the AMS
   \usepackage{amsmath}

\usepackage{amssymb}
\usepackage{wasysym}
\usepackage{amsthm} 
\usepackage{mathrsfs}
\usepackage{mathtools}
\usepackage{epsfig}
%%%%Theorem like environments follow sectionwise numbering, 
%%%%only exceptions being 'Remark' and 'Note'.
%%%%Do not edit this basic setting, if you are adding more such environments.
\theoremstyle{plain}% default
\newtheorem{theorem}{Theorem}[section]
\newtheorem{lemma}[theorem]{Lemma}
\newtheorem{proposition}[theorem]{Proposition}
\newtheorem{corollary}[theorem]{Corollary}
\newtheorem{property}[theorem]{Property}

\theoremstyle{definition}
\newtheorem{definition}[theorem]{Definition}
\newtheorem{conjecture}[theorem]{Conjecture}
\newtheorem{example}[theorem]{Example}
\newtheorem{observation}[theorem]{Observation}

\theoremstyle{remark}
\newtheorem*{remark}{Remark}
\newtheorem*{note}{Note}
\newtheorem{claim}{Claim}[chapter]

\usepackage{thm-restate}
\usepackage[ruled,vlined,algochapter]{algorithm2e}

% ********************************************************************
% General useful packages
% ********************************************************************
\usepackage{graphicx} %

% \usepackage[newfloat]{minted}
% \newmintinline{haskell}{breaklines}

\usepackage{scrhack} % fix warnings when using KOMA with listings package
\usepackage{xspace} % to get the spacing after macros right
\PassOptionsToPackage{printonlyused,smaller}{acronym}
  \usepackage{acronym} % nice macros for handling all acronyms in the thesis
  \def\bflabel#1{{\acsfont{#1}\hfill}}
  \def\aclabelfont#1{\acsfont{#1}}
% ****************************************************************************************************
%\usepackage{pgfplots} % External TikZ/PGF support (thanks to Andreas Nautsch)
%\usetikzlibrary{external}
%\tikzexternalize[mode=list and make, prefix=ext-tikz/]
% ****************************************************************************************************




% ****************************************************************************************************
% 4. Setup floats: tables, (sub)figures, and captions
%%%Do not edit this setting
% ****************************************************************************************************
\usepackage{tabularx} % better tables
  \setlength{\extrarowheight}{3pt} % increase table row height
\newcommand{\tableheadline}[1]{\multicolumn{1}{l}{\spacedlowsmallcaps{#1}}}
\newcommand{\myfloatalign}{\centering} % to be used with each float for alignment
%\usepackage{subfig}
\usepackage{caption}
\usepackage{subcaption}



% ****************************************************************************************************


% ****************************************************************************************************
% 5. Setup code listings
%%Do not edit this setting
% ****************************************************************************************************
\usepackage{listings}
%\lstset{emph={trueIndex,root},emphstyle=\color{BlueViolet}}%\underbar} % for special keywords
\lstset{language=[LaTeX]Tex,%C++,
  morekeywords={PassOptionsToPackage,selectlanguage},
  keywordstyle=\color{CTkeyword},%\bfseries,
  basicstyle=\small\ttfamily,
  %identifierstyle=\color{NavyBlue},
  commentstyle=\color{CTcomment}\ttfamily,
  stringstyle=\rmfamily,
  numbers=none,%left,%
  numberstyle=\scriptsize,%\tiny
  stepnumber=5,
  numbersep=8pt,
  showstringspaces=false,
  breaklines=true,
  %frameround=ftff,
  %frame=single,
  belowcaptionskip=.75\baselineskip
  %frame=L
}
% ****************************************************************************************************

%%%%Note that for list of symbols to be inluded the following commands needs to be executed
%%%%pdflatex ClassicThesis.tex
%%%%makeindex ClassicThesis.nlo -s nomencl.ist -o ClassicThesis.nls
%%%%pdflatex ClassicThesis.tex
\usepackage{nomencl}
\makenomenclature
\renewcommand{\nomname}{List of Symbols}
\setlength{\nomlabelwidth}{2.5cm}
% ****************************************************************************************************
% 6. Last calls before the bar closes
% ****************************************************************************************************
% ********************************************************************
% Her Majesty herself
% ********************************************************************
\usepackage{classicthesis}


% ********************************************************************
% Fine-tune hyperreferences (hyperref should be called last)
%%%Do not edit this setting
% ********************************************************************
\hypersetup{%
  colorlinks=true, linktocpage=true, pdfstartpage=3, pdfstartview=FitV,%
  breaklinks=true, pageanchor=true,%
  pdfpagemode=UseNone, %
  plainpages=false, bookmarksnumbered, bookmarksopen=true, bookmarksopenlevel=1,%
  hypertexnames=true, pdfhighlight=/O,%
  urlcolor=CTurl, linkcolor=CTlink, citecolor=CTcitation, %
  pdftitle={\myTitle},%
  pdfauthor={\textcopyright\ \myName, \myUni},%
  pdfsubject={},%
  pdfkeywords={},%
  pdfcreator={xelatex},%
  %pdfcreator={pdfLaTeX},%
  %pdfproducer={LaTeX with hyperref and classicthesis}%
}

%%%%%%%%%Do not edit below this line except the very last line of this file %%%%%%%%%%%%%%%%%%%%%%%%%%%%%%%%%

% ********************************************************************
% Setup autoreferences (hyperref and babel)
% ********************************************************************
% There are some issues regarding autorefnames
% http://www.tex.ac.uk/cgi-bin/texfaq2html?label=latexwords
% you have to redefine the macros for the
% language you use, e.g., american, ngerman
% (as chosen when loading babel/AtBeginDocument)
% ********************************************************************
\makeatletter
\@ifpackageloaded{babel}%
  {%
    \addto\extrasamerican{%
      \renewcommand*{\figureautorefname}{Figure}%
      \renewcommand*{\tableautorefname}{Table}%
      \renewcommand*{\partautorefname}{Part}%
      \renewcommand*{\chapterautorefname}{Chapter}%
      \renewcommand*{\sectionautorefname}{Section}%
      \renewcommand*{\subsectionautorefname}{Section}%
      \renewcommand*{\subsubsectionautorefname}{Section}%
    }%
    \addto\extrasngerman{%
      \renewcommand*{\paragraphautorefname}{Absatz}%
      \renewcommand*{\subparagraphautorefname}{Unterabsatz}%
      \renewcommand*{\footnoteautorefname}{Fu\"snote}%
      \renewcommand*{\FancyVerbLineautorefname}{Zeile}%
      \renewcommand*{\theoremautorefname}{Theorem}%
      \renewcommand*{\appendixautorefname}{Anhang}%
      \renewcommand*{\equationautorefname}{Gleichung}%
      \renewcommand*{\itemautorefname}{Punkt}%
    }%
      % Fix to getting autorefs for subfigures right (thanks to Belinda Vogt for changing the definition)
      \providecommand{\subfigureautorefname}{\figureautorefname}%
    }{\relax}
\makeatother

\listfiles
\ifthenelse{\boolean{ct@printready}}%% line spread is kept larger for screen view option, because margin space is less
{
  \linespread{1.1} % this is for print ready
}
{
   \linespread{1.3} % increased from 1.1 to 1.3 in iitpkd v2.1 for screeview version
}   
%%%%%
%%% \renewcommand{\rmdefault}{pplx} %%uncomment this line if you want to change number typesetting style

%%%%%%%%%%%%%%%%%%%%%%%%%%%%%%%%%%%%%%%%%%%%%%%%%%%%%%%%%%%%%%%%%%%%%%%%%%%%%
% CUSTOM
%%%%%%%%%%%%%%%%%%%%%%%%%%%%%%%%%%%%%%%%%%%%%%%%%%%%%%%%%%%%%%%%%%%%%%%%%%%%%

\usepackage[acronym,shortcuts]{glossaries}
%
% https://tex.stackexchange.com/questions/98494/glossaries-dont-print-single-occurences
\glsenableentrycount % enable \cgls, \cglspl, \cGls, \cGlspl
\let\ac\cgls
\let\acpl\cglspl
\let\Ac\cGls
\let\Acpl\cGlspl
%
%\newglossary{symbols}{sym}{sbl}{List of Symbols}
\makeglossaries

\usepackage{nomencl}
\makenomenclature

% https://github.com/nickgian/thesis/blob/master/thesis.tex

\usepackage{mathpartir}
\usepackage{fontspec}
\usepackage{amssymb}
\usepackage{amsthm}
\usepackage{amsmath}
\usepackage{forest}
\usepackage{xcolor}
\usepackage{listingsutf8}
% \usepackage{lmodern}
\usepackage{pgfplots}
%\pgfplotsset{compat=1.18}
\usepackage{alectryon}
\usepackage{pygments}
\usepackage{fancyvrb}
\usepackage{turnstile}
%\usepackage{minted}
%% \usepackage{refcheck}

\newcommand\bbool{{\fontspec{Symbola} 𝔹}}
\newcommand\cmodels{{\fontspec{Symbola} ⊨}}
\newcommand\blackcircle{{\fontspec{Symbola} ●}}
\newcommand\alecmodels{{\fontspec{DejaVu Sans} ⊨}}
\newcommand\aleczero{{\fontspec{DejaVu Sans} 𝟘}}
\newcommand\alecone{{\fontspec{DejaVu Sans} 𝟙}}
\newcommand{\modelsere}{\sttstile{}{}} % three horizontal stripe models symbol
\newcommand{\modelbool}{\ddtstile{}{}} % two horizontal stripe models symbol
\newcommand{\modelppty}{\sdtstile{}{}} % three horizontal stripe models symbol

% Disjoint union
\newcommand\utimes{\mathbin{\ooalign{$\cup$\cr%
   \hfil\raise0.42ex\hbox{$\scriptscriptstyle\times$}\hfil\cr}}}
\newcommand\bigutimes{\mathop{\ooalign{$\bigcup$\cr%
   \hfil\raise0.36ex\hbox{$\scriptscriptstyle\boldsymbol{\times}$}\hfil\cr}}}



% For tblr env
\usepackage{tabularray}
\UseTblrLibrary{booktabs}

%\usepackage{subfig}
%\usepackage[acronym]{glossaries}

\usepackage{tikz}
\usetikzlibrary{arrows} % ...customizing arrows
\usetikzlibrary{arrows.meta}
\usetikzlibrary{automata} % Import library for drawing automata
\usetikzlibrary{calc}
\usetikzlibrary{decorations.markings}
\usetikzlibrary{decorations.pathreplacing}
\usetikzlibrary{positioning} % ...positioning nodes
\usetikzlibrary{shapes}
\usetikzlibrary{shapes.misc}
%
\usetikzlibrary{external}
\tikzexternalize
\usepackage{circuitikz}

\lstset{%
    basicstyle=\ttfamily\footnotesize,
    numbers=none,
    escapechar=\#,
}

% \lstset{%
%     numbers=none,
%     numberstyle=\tiny\color[rgb]{0.5,0.5,0.5},
%     basicstyle=\ttfamily\footnotesize,
%     basewidth=0.59em,
%     keywordstyle=[3]{},
%     commentstyle=\itshape\footnotesize,
%     tabsize=4,
%     frame=single,
%     frameround=tttt,
%     showstringspaces=false,
%     breaklines=true,
%     %breaklines=false,
%     captionpos=b,
%     aboveskip=\bigskipamount,
%     belowskip=\bigskipamount,
%     escapechar=#,
%     keywordstyle=\color[rgb]{0,0,1},
%     commentstyle=\color[rgb]{0.133,0.545,0.133},
%     stringstyle=\color[rgb]{0.627,0.126,0.941},
%     extendedchars=true,
% }


% \usepackage{lstcoq}
% \usepackage{lsthsk}
% \usepackage{lstverilog}
% 
% % Style options:
% % numberstyle,basicstyle,identifierstyle,commentstyle,stringstyle
% % keywordstyle=[1]{},keywordstyle=[2]{},directivestyle
% % \small\tiny\footnotesize\itshape\ttfamily\bf
% \lstdefinestyle{coq_style}{%
%   language=Coq, float=htb!
% }
% \lstnewenvironment{Coq}[2]{%
%   \nopagebreak
%   \lstset{style=coq_style,label={#1},caption={#2}}
% }{}
% 
% \newcommand{\includecode}[4][Coq]{%
%   \nopagebreak
%   \lstinputlisting[label={#2},caption={#3},style={#1_style}]{#4}
% }
% \newcommand{\lstheader}[2]{%
%   \begin{lstlisting}[label={#1},caption={#2},style=coq_style]
% }


% \def\lstlanguagefiles{defManSSR.tex}
% \lstset{language=SSR}

% \newcommand\doubleplus{+ \kern-1.3ex + \kern0.8ex}

% Twiggly arrow
\newcommand{\vto}{\mathrel{\leadsto}}
\newcommand{\cvto}{{\fontspec{DejaVu Sans} ↝}}

\newcommand\hsk[1]{\texttt{#1}}
\newcommand\cq[1]{\texttt{#1}}
\newcommand\code[1]{\texttt{#1}}

% Celsius
\usepackage{siunitx}

% \theoremstyle{plain}
\theoremstyle{definition}
\newtheorem*{objective}{Objective}

\input{acronyms.tex}

% % \setlength{\abovecaptionskip}{2pt}
% \setlength{\belowcaptionskip}{3em}
% % \addtolength{\subfigcapskip}{-1pt}

%% \usepackage[pagewise]{lineno}
%% \linenumbers

\newcommand*{\rplus}{\textit{r+}}
\newcommand*{\bool}{{\fontspec{DejaVu Sans} 𝔹}}
\newcommand*{\sem}{{\fontspec{DejaVu Sans} ⊨}}
\newcommand*{\eqdouble}{==}
% \newcommand*{\eqeq}{{\fontspec{Noto Sans Math Regular} ⩵}}

\tikzset{
%     node distance=2.5cm, % Minimum distance between two nodes. Change if necessary.
    every state/.style={ % Sets the properties for each state
    semithick,
    fill=gray!10},
    initial text={}, % No label on start arrow
%     double distance=2pt, % Adjust appearance of accept states
%     every edge/.style={ % Sets the properties for each transition
%     draw,
%     ->,>=stealth', % Makes edges directed with bold arrowheads
%     auto,
%     semithick}%
}


%%% New TikZ shape

% taken from manual
\makeatletter
\pgfdeclareshape{document}{
\inheritsavedanchors[from=rectangle] % this is nearly a rectangle
\inheritanchorborder[from=rectangle]
\inheritanchor[from=rectangle]{center}
\inheritanchor[from=rectangle]{north}
\inheritanchor[from=rectangle]{south}
\inheritanchor[from=rectangle]{west}
\inheritanchor[from=rectangle]{east}
% ... and possibly more
\backgroundpath{% this is new
% store lower right in xa/ya and upper right in xb/yb
\southwest \pgf@xa=\pgf@x \pgf@ya=\pgf@y
\northeast \pgf@xb=\pgf@x \pgf@yb=\pgf@y
% compute corner of ‘‘flipped page’’
\pgf@xc=\pgf@xb \advance\pgf@xc by-10pt % this should be a parameter
\pgf@yc=\pgf@yb \advance\pgf@yc by-10pt
% construct main path
\pgfpathmoveto{\pgfpoint{\pgf@xa}{\pgf@ya}}
\pgfpathlineto{\pgfpoint{\pgf@xa}{\pgf@yb}}
\pgfpathlineto{\pgfpoint{\pgf@xc}{\pgf@yb}}
\pgfpathlineto{\pgfpoint{\pgf@xb}{\pgf@yc}}
\pgfpathlineto{\pgfpoint{\pgf@xb}{\pgf@ya}}
\pgfpathclose
% add little corner
\pgfpathmoveto{\pgfpoint{\pgf@xc}{\pgf@yb}}
\pgfpathlineto{\pgfpoint{\pgf@xc}{\pgf@yc}}
\pgfpathlineto{\pgfpoint{\pgf@xb}{\pgf@yc}}
\pgfpathlineto{\pgfpoint{\pgf@xc}{\pgf@yc}}
}
}
\makeatother

\tikzstyle{doc}=[%
  draw,
  thick,
  align=center,
  color=black,
  shape=document,
  % minimum width=20mm,
  % minimum height=28.2mm,
  shape=document,
  inner sep=2ex,
]


% ****************************************************************************************************
% If you like the classicthesis, then I would appreciate a postcard.
% My address can be found in the file ClassicThesis.pdf. A collection
% of the postcards I received so far is available online at
% http://postcards.miede.de
% ****************************************************************************************************

\RequirePackage{silence} % :-\suppress an unnecessary warning in compilation
    \WarningFilter{scrreprt}{Usage of package `titlesec'}
    \WarningFilter{titlesec}{Non standard sectioning command detected}
    \WarningFilter{hyperref}{Token not allowed in a PDF string (PDFDocEncoding)}
   
% ****************************************************************************************************
% 0. Set the encoding of your files. UTF-8 is the only sensible encoding nowadays. If you can't read
% äöüßáéçèê∂åëæƒÏ€ then change the encoding setting in your editor, not the line below. If your editor
% does not support utf8 use another editor!
% ****************************************************************************************************
\PassOptionsToPackage{utf8}{inputenc}
  \usepackage{inputenc}

\PassOptionsToPackage{T1}{fontenc} % T2A for cyrillics
  \usepackage{fontenc}

\PassOptionsToPackage{final}{microtype}  %iitpkd
\usepackage[final]{microtype} %iitpkd

\usepackage{etoolbox}

\emergencystretch=1em

% *******************************************************************************************************************
% Note : Do not edit above this line
% Note : Do not edit options to classicthesis given below except the options eulermath, drafting, printready and numsupervisors
% % 
% *******************************************************************************************************************
\PassOptionsToPackage{
   tocaligned=false, 
  dottedtoc=true,
  eulerchapternumbers=true, 
  linedheaders=false,  
  eulermath=true, %true, %false %If false is set, cmmodern font will be used for math. % See chapter 3 of the template thesis to see the difference. %If text font is changed, be careful about this option.  
  beramono=true,   
  palatino=true,    
  style=classicthesis, 
  floatperchapter=true,     % numbering per chapter for all floats (i.e., Figure 1.1)
  numsupervisors=one, %one %two %three %set according to your number of supervisors
  %drafting=true,    % use this for draft submitted with synopsis. comment this line in the thesis for review
  drafting=false, % for final version this line is to be used
  printready=false, %Enable this for making title page and hyperreflinks in colour for screen reading
  %printready=true,  %Enable this for making title page and hyperreflinks in black for printing, margin adjustments
 }{classicthesis}


% ****************************************************************************************************
% 2. Personal data and user ad-hoc commands (insert your own data here)
% ****************************************************************************************************

\newcommand{\myTitle}{Towards verified regular expression matchers in hardware\xspace}
%\newcommand{\mySubtitle}{Subtitle of Your Thesis\xspace}
%\newcommand{\myDegree}{Doctor of Philosophy\xspace}
\newcommand{\myDegree}{Master of Science \xspace}
%\newcommand{\myDegree}{Master of Technology \xspace}
%\newcommand{\myDegree}{Master of Science in Engineering \xspace}
\newcommand{\myName}{Julin Shaji\xspace}
\newcommand{\myRollNo}{112103001\xspace}
%\newcommand{\myGender}{her\xspace}
\newcommand{\myGender}{him\xspace}
\newcommand{\myDepartment}{Department of Computer Science and Engineering\xspace}

\newcommand{\mySupervisorOne}{Dr. Piyush P. Kurur}
\newcommand{\mySupervisorOneDesig}{Associate Professor}%Example Assistant Professor
\newcommand{\mySupervisorOneDept}{Dept.~of Computer Science and Engg} %Example Dept.~of Computer Science and Engg.
\newcommand{\mySupervisorOneInstitute}{IIT Palakkad}% Example IIT Palakkad

% \newcommand{\mySupervisorTwo}{<Name of Supervisor2>}%used only if necessary
% \newcommand{\mySupervisorTwoDesig}{<Supervisor2 Designation>}%Example Assistant Professor
% \newcommand{\mySupervisorTwoDept}{<Supervisor2 Department>} %Example Dept.~of Computer Science and Engg.
% \newcommand{\mySupervisorTwoInstitute}{<Supervisor2 Institute>}% Example IIT Palakkad

\newcommand{\myUni}{Indian Institute of Technology Palakkad\xspace}
\newcommand{\myLocation}{Palakkad\xspace}
\newcommand{\myTime}{May 2025\xspace}
\newcommand{\myVersion}{\classicthesis}
\newcommand{\myDCMemberOne}{Dr. Unnikrishnan C.\xspace}
\newcommand{\myDCMemberOneDesigAndDept}{Assistant Professor, .Dept.~of Computer Science and Engineering\xspace}%Example: Assistant Professor, .Dept.~of Computer Science and Engineering
\newcommand{\myDCMemberOneInstitute}{IIT Palakkad}
%
\newcommand{\myDCMemberTwo}{Dr. Sandeep Chandran}
\newcommand{\myDCMemberTwoDesigAndDept}{Assistant Professor, .Dept.~of Computer Science and Engineering}
\newcommand{\myDCMemberTwoInstitute}{IIT Palakkad}
%
\newcommand{\myDCMemberThree}{Dr. Subrahmanyam Mula}
\newcommand{\myDCMemberThreeDesigAndDept}{Assistant Professor, .Dept.~of Electrical Engineering}
\newcommand{\myDCMemberThreeInstitute}{IIT Palakkad}

% ********************************************************************
% Setup, finetuning, and useful commands
% ********************************************************************
\providecommand{\mLyX}{L\kern-.1667em\lower.25em\hbox{Y}\kern-.125emX\@}
\newcommand{\ie}{i.\,e.}
\newcommand{\Ie}{I.\,e.}
\newcommand{\eg}{e.\,g.}
\newcommand{\Eg}{E.\,g.}
% ****************************************************************************************************


% ****************************************************************************************************
% 3. Loading some handy packages
% ****************************************************************************************************
% ********************************************************************
% Packages with options that might require adjustments
% ********************************************************************
\PassOptionsToPackage{ngerman,american}{babel} % change this to your language(s), main language last
\usepackage{babel}

\usepackage{csquotes}
%%%%%You may change the reference format to any of the options given agaist the parameters. 
%%%%%It is suggested that you maintain compatibility with natbib.
\PassOptionsToPackage{%
  %backend=biber,bibencoding=utf8, %instead of bibtex
  %backend=bibtex8,bibencoding=ascii,%
  language=auto,%
  style=numeric-comp,%
  %style=authoryear-comp, % Author 1999, 2010
  %bibstyle=authoryear,dashed=false, % dashed: substitute rep. author with ---
  sorting=nyt, % name, year, title
  maxbibnames=10, % default: 3, et al.
  backend=biber,
  %backref=true,%
  natbib=true % natbib compatibility mode (\citep and \citet still work)
}{biblatex}
\usepackage{biblatex}

\PassOptionsToPackage{fleqn}{amsmath}       % math environments and more by the AMS
   \usepackage{amsmath}

\usepackage{amssymb}
\usepackage{wasysym}
\usepackage{amsthm} 
\usepackage{mathrsfs}
\usepackage{mathtools}
\usepackage{epsfig}
%%%%Theorem like environments follow sectionwise numbering, 
%%%%only exceptions being 'Remark' and 'Note'.
%%%%Do not edit this basic setting, if you are adding more such environments.
\theoremstyle{plain}% default
\newtheorem{theorem}{Theorem}[section]
\newtheorem{lemma}[theorem]{Lemma}
\newtheorem{proposition}[theorem]{Proposition}
\newtheorem{corollary}[theorem]{Corollary}
\newtheorem{property}[theorem]{Property}

\theoremstyle{definition}
\newtheorem{definition}[theorem]{Definition}
\newtheorem{conjecture}[theorem]{Conjecture}
\newtheorem{example}[theorem]{Example}
\newtheorem{observation}[theorem]{Observation}

\theoremstyle{remark}
\newtheorem*{remark}{Remark}
\newtheorem*{note}{Note}
\newtheorem{claim}{Claim}[chapter]

\usepackage{thm-restate}
\usepackage[ruled,vlined,algochapter]{algorithm2e}

% ********************************************************************
% General useful packages
% ********************************************************************
\usepackage{graphicx} %

% \usepackage[newfloat]{minted}
% \newmintinline{haskell}{breaklines}

\usepackage{scrhack} % fix warnings when using KOMA with listings package
\usepackage{xspace} % to get the spacing after macros right
\PassOptionsToPackage{printonlyused,smaller}{acronym}
  \usepackage{acronym} % nice macros for handling all acronyms in the thesis
  \def\bflabel#1{{\acsfont{#1}\hfill}}
  \def\aclabelfont#1{\acsfont{#1}}
% ****************************************************************************************************
%\usepackage{pgfplots} % External TikZ/PGF support (thanks to Andreas Nautsch)
%\usetikzlibrary{external}
%\tikzexternalize[mode=list and make, prefix=ext-tikz/]
% ****************************************************************************************************




% ****************************************************************************************************
% 4. Setup floats: tables, (sub)figures, and captions
%%%Do not edit this setting
% ****************************************************************************************************
\usepackage{tabularx} % better tables
  \setlength{\extrarowheight}{3pt} % increase table row height
\newcommand{\tableheadline}[1]{\multicolumn{1}{l}{\spacedlowsmallcaps{#1}}}
\newcommand{\myfloatalign}{\centering} % to be used with each float for alignment
%\usepackage{subfig}
\usepackage{caption}
\usepackage{subcaption}



% ****************************************************************************************************


% ****************************************************************************************************
% 5. Setup code listings
%%Do not edit this setting
% ****************************************************************************************************
\usepackage{listings}
%\lstset{emph={trueIndex,root},emphstyle=\color{BlueViolet}}%\underbar} % for special keywords
\lstset{language=[LaTeX]Tex,%C++,
  morekeywords={PassOptionsToPackage,selectlanguage},
  keywordstyle=\color{CTkeyword},%\bfseries,
  basicstyle=\small\ttfamily,
  %identifierstyle=\color{NavyBlue},
  commentstyle=\color{CTcomment}\ttfamily,
  stringstyle=\rmfamily,
  numbers=none,%left,%
  numberstyle=\scriptsize,%\tiny
  stepnumber=5,
  numbersep=8pt,
  showstringspaces=false,
  breaklines=true,
  %frameround=ftff,
  %frame=single,
  belowcaptionskip=.75\baselineskip
  %frame=L
}
% ****************************************************************************************************

%%%%Note that for list of symbols to be inluded the following commands needs to be executed
%%%%pdflatex ClassicThesis.tex
%%%%makeindex ClassicThesis.nlo -s nomencl.ist -o ClassicThesis.nls
%%%%pdflatex ClassicThesis.tex
\usepackage{nomencl}
\makenomenclature
\renewcommand{\nomname}{List of Symbols}
\setlength{\nomlabelwidth}{2.5cm}
% ****************************************************************************************************
% 6. Last calls before the bar closes
% ****************************************************************************************************
% ********************************************************************
% Her Majesty herself
% ********************************************************************
\usepackage{classicthesis}


% ********************************************************************
% Fine-tune hyperreferences (hyperref should be called last)
%%%Do not edit this setting
% ********************************************************************
\hypersetup{%
  colorlinks=true, linktocpage=true, pdfstartpage=3, pdfstartview=FitV,%
  breaklinks=true, pageanchor=true,%
  pdfpagemode=UseNone, %
  plainpages=false, bookmarksnumbered, bookmarksopen=true, bookmarksopenlevel=1,%
  hypertexnames=true, pdfhighlight=/O,%
  urlcolor=CTurl, linkcolor=CTlink, citecolor=CTcitation, %
  pdftitle={\myTitle},%
  pdfauthor={\textcopyright\ \myName, \myUni},%
  pdfsubject={},%
  pdfkeywords={},%
  pdfcreator={xelatex},%
  %pdfcreator={pdfLaTeX},%
  %pdfproducer={LaTeX with hyperref and classicthesis}%
}

%%%%%%%%%Do not edit below this line except the very last line of this file %%%%%%%%%%%%%%%%%%%%%%%%%%%%%%%%%

% ********************************************************************
% Setup autoreferences (hyperref and babel)
% ********************************************************************
% There are some issues regarding autorefnames
% http://www.tex.ac.uk/cgi-bin/texfaq2html?label=latexwords
% you have to redefine the macros for the
% language you use, e.g., american, ngerman
% (as chosen when loading babel/AtBeginDocument)
% ********************************************************************
\makeatletter
\@ifpackageloaded{babel}%
  {%
    \addto\extrasamerican{%
      \renewcommand*{\figureautorefname}{Figure}%
      \renewcommand*{\tableautorefname}{Table}%
      \renewcommand*{\partautorefname}{Part}%
      \renewcommand*{\chapterautorefname}{Chapter}%
      \renewcommand*{\sectionautorefname}{Section}%
      \renewcommand*{\subsectionautorefname}{Section}%
      \renewcommand*{\subsubsectionautorefname}{Section}%
    }%
    \addto\extrasngerman{%
      \renewcommand*{\paragraphautorefname}{Absatz}%
      \renewcommand*{\subparagraphautorefname}{Unterabsatz}%
      \renewcommand*{\footnoteautorefname}{Fu\"snote}%
      \renewcommand*{\FancyVerbLineautorefname}{Zeile}%
      \renewcommand*{\theoremautorefname}{Theorem}%
      \renewcommand*{\appendixautorefname}{Anhang}%
      \renewcommand*{\equationautorefname}{Gleichung}%
      \renewcommand*{\itemautorefname}{Punkt}%
    }%
      % Fix to getting autorefs for subfigures right (thanks to Belinda Vogt for changing the definition)
      \providecommand{\subfigureautorefname}{\figureautorefname}%
    }{\relax}
\makeatother

\listfiles
\ifthenelse{\boolean{ct@printready}}%% line spread is kept larger for screen view option, because margin space is less
{
  \linespread{1.1} % this is for print ready
}
{
   \linespread{1.3} % increased from 1.1 to 1.3 in iitpkd v2.1 for screeview version
}   
%%%%%
%%% \renewcommand{\rmdefault}{pplx} %%uncomment this line if you want to change number typesetting style

%%%%%%%%%%%%%%%%%%%%%%%%%%%%%%%%%%%%%%%%%%%%%%%%%%%%%%%%%%%%%%%%%%%%%%%%%%%%%
% CUSTOM
%%%%%%%%%%%%%%%%%%%%%%%%%%%%%%%%%%%%%%%%%%%%%%%%%%%%%%%%%%%%%%%%%%%%%%%%%%%%%

\usepackage[acronym,shortcuts]{glossaries}
%
% https://tex.stackexchange.com/questions/98494/glossaries-dont-print-single-occurences
\glsenableentrycount % enable \cgls, \cglspl, \cGls, \cGlspl
\let\ac\cgls
\let\acpl\cglspl
\let\Ac\cGls
\let\Acpl\cGlspl
%
%\newglossary{symbols}{sym}{sbl}{List of Symbols}
\makeglossaries

\usepackage{nomencl}
\makenomenclature

% https://github.com/nickgian/thesis/blob/master/thesis.tex

\usepackage{mathpartir}
\usepackage{fontspec}
\usepackage{amssymb}
\usepackage{amsthm}
\usepackage{amsmath}
\usepackage{forest}
\usepackage{xcolor}
\usepackage{listingsutf8}
% \usepackage{lmodern}
\usepackage{pgfplots}
%\pgfplotsset{compat=1.18}
\usepackage{alectryon}
\usepackage{pygments}
\usepackage{fancyvrb}
\usepackage{turnstile}
%\usepackage{minted}
%% \usepackage{refcheck}

\newcommand\bbool{{\fontspec{Symbola} 𝔹}}
\newcommand\cmodels{{\fontspec{Symbola} ⊨}}
\newcommand\blackcircle{{\fontspec{Symbola} ●}}
\newcommand\alecmodels{{\fontspec{DejaVu Sans} ⊨}}
\newcommand\aleczero{{\fontspec{DejaVu Sans} 𝟘}}
\newcommand\alecone{{\fontspec{DejaVu Sans} 𝟙}}
\newcommand{\modelsere}{\sttstile{}{}} % three horizontal stripe models symbol
\newcommand{\modelbool}{\ddtstile{}{}} % two horizontal stripe models symbol
\newcommand{\modelppty}{\sdtstile{}{}} % three horizontal stripe models symbol

% Disjoint union
\newcommand\utimes{\mathbin{\ooalign{$\cup$\cr%
   \hfil\raise0.42ex\hbox{$\scriptscriptstyle\times$}\hfil\cr}}}
\newcommand\bigutimes{\mathop{\ooalign{$\bigcup$\cr%
   \hfil\raise0.36ex\hbox{$\scriptscriptstyle\boldsymbol{\times}$}\hfil\cr}}}



% For tblr env
\usepackage{tabularray}
\UseTblrLibrary{booktabs}

%\usepackage{subfig}
%\usepackage[acronym]{glossaries}

\usepackage{tikz}
\usetikzlibrary{arrows} % ...customizing arrows
\usetikzlibrary{arrows.meta}
\usetikzlibrary{automata} % Import library for drawing automata
\usetikzlibrary{calc}
\usetikzlibrary{decorations.markings}
\usetikzlibrary{decorations.pathreplacing}
\usetikzlibrary{positioning} % ...positioning nodes
\usetikzlibrary{shapes}
\usetikzlibrary{shapes.misc}
%
\usetikzlibrary{external}
\tikzexternalize
\usepackage{circuitikz}

\lstset{%
    basicstyle=\ttfamily\footnotesize,
    numbers=none,
    escapechar=\#,
}

% \lstset{%
%     numbers=none,
%     numberstyle=\tiny\color[rgb]{0.5,0.5,0.5},
%     basicstyle=\ttfamily\footnotesize,
%     basewidth=0.59em,
%     keywordstyle=[3]{},
%     commentstyle=\itshape\footnotesize,
%     tabsize=4,
%     frame=single,
%     frameround=tttt,
%     showstringspaces=false,
%     breaklines=true,
%     %breaklines=false,
%     captionpos=b,
%     aboveskip=\bigskipamount,
%     belowskip=\bigskipamount,
%     escapechar=#,
%     keywordstyle=\color[rgb]{0,0,1},
%     commentstyle=\color[rgb]{0.133,0.545,0.133},
%     stringstyle=\color[rgb]{0.627,0.126,0.941},
%     extendedchars=true,
% }


% \usepackage{lstcoq}
% \usepackage{lsthsk}
% \usepackage{lstverilog}
% 
% % Style options:
% % numberstyle,basicstyle,identifierstyle,commentstyle,stringstyle
% % keywordstyle=[1]{},keywordstyle=[2]{},directivestyle
% % \small\tiny\footnotesize\itshape\ttfamily\bf
% \lstdefinestyle{coq_style}{%
%   language=Coq, float=htb!
% }
% \lstnewenvironment{Coq}[2]{%
%   \nopagebreak
%   \lstset{style=coq_style,label={#1},caption={#2}}
% }{}
% 
% \newcommand{\includecode}[4][Coq]{%
%   \nopagebreak
%   \lstinputlisting[label={#2},caption={#3},style={#1_style}]{#4}
% }
% \newcommand{\lstheader}[2]{%
%   \begin{lstlisting}[label={#1},caption={#2},style=coq_style]
% }


% \def\lstlanguagefiles{defManSSR.tex}
% \lstset{language=SSR}

% \newcommand\doubleplus{+ \kern-1.3ex + \kern0.8ex}

% Twiggly arrow
\newcommand{\vto}{\mathrel{\leadsto}}
\newcommand{\cvto}{{\fontspec{DejaVu Sans} ↝}}

\newcommand\hsk[1]{\texttt{#1}}
\newcommand\cq[1]{\texttt{#1}}
\newcommand\code[1]{\texttt{#1}}

% Celsius
\usepackage{siunitx}

% \theoremstyle{plain}
\theoremstyle{definition}
\newtheorem*{objective}{Objective}

\newacronym{SuV}{SuV}{System under Verification}
\newacronym{SuO}{SuO}{System under Observation}
\newacronym{RV}{RV}{Runtime verification}
\newacronym{PSL}{PSL}{Property Specification Language}
\newacronym{SERE}{SERE}{Sequential Regular Expression}
\newacronym{LTL}{LTL}{Linear Temporal Logic}

\newacronym{DFA}{DFA}{Deterministic finite automata}
\newacronym{NFA}{NFA}{Non-deterministic finite automata}
\newacronym{TCB}{TCB}{Trusted Code Base}

\newacronym{HDL}{HDL}{Hardware Description Language}
\newacronym{FM}{FM}{Formal methods}
\newacronym{FV}{FV}{Formal verification}

\newacronym{BSV}{BSV}{Bluespec SystemVerilog}

\newacronym{DSL}{DSL}{Domain Specific Language}
\newacronym{eDSL}{eDSL}{Embedded Domain Specific Language}
\newacronym{LOC}{LOC}{Lines of Code}
\newacronym{STA}{STA}{Static Timing Analysis}

\newacronym{GADT}{GADT}{Generalized Algebraic Data Type}
\newacronym{ADT}{ADT}{Algebraic Data Type}

\newacronym{CPU}{CPU}{Central Processing Unit}
\newacronym{GPU}{GPU}{Graphics Processing Unit}
\newacronym{GPGPU}{GPGPU}{General Purpose Graphics Processing Unit}

% \newacronym{NIDS}{NIDS}{Network Intrusion Detection System}
\newacronym{PLA}{PLA}{Programmable Logic Array}
\newacronym{FOSS}{FOSS}{Free and Open Source Software}
\newacronym{REPL}{REPL}{Read-Eval-Print-Loop}
\newacronym{IP}{IP}{Intellectual Property}

%% \newacronym[
%%     \glslongpluralkey={Regular expressions},
%%     \glsshortpluralkey={regexes}
%%   ]{TaA}{regex}{regular expression}

\newacronym[
    \glslongpluralkey={Regular expressions},
    \glsshortpluralkey={regexes}
  ]{rgx}{regex}{regular expression}

%% % \newacronym[
%% %     \glslongpluralkey={Tests-as-atoms regexes},
%% %     \glsshortpluralkey={T-regexes}
%% %   ]{TaA}{T-regex}{Tests-as-atoms regex}

%% \newacronym[
%%     \glslongpluralkey={Regular expressions},
%%     \glsshortpluralkey={regexes}
%%   ]{regex}{regex}{Regular expression}

\newacronym{EDA}{EDA}{Electronic Design Automation}
\newacronym{FPGA}{FPGA}{Field Programmable Gate Array}
\newacronym{ASIC}{ASIC}{Application Specific Integrated Circuit}
\newacronym{SSD}{SSD}{Solid State Drive}
\newacronym{GHC}{GHC}{Glasgow Haskell Compiler}
\newacronym{VHDL}{VHDL}{VHSIC Hardware Description Language}
\newacronym{LUT}{LUT}{Look-up table}
% \newacronym{AC}{AC}{Air conditioning system}
\newacronym{RAM}{RAM}{Random Access Memory}

\newcommand{\bvec}{Boolean vector}
\newcommand{\bvecs}{Boolean vectors}

% \newglossaryentry{TaA}{
%   name=T-regex,
%   description={Tests-as-atoms regex},
%   plural=T-regexes
% }

% \newglossaryentry{regex}{
%   name=regex,
%   description={Regular expression},
%   plural=regexes
% }

% \newglossaryentry{EDA}{
%   name=EDA,
%   description={Electronic Design Automation}
% }

% \newglossaryentry{FPGA}{
%   name=FPGA,
%   description={Field Programmable Gate Array}
% }

% \newglossaryentry{ASIC}{
%   name=FPGA,
%   description={Application Specific Integrated Circuit}
% }

% \newglossaryentry{SSD}{
%   name=SSD,
%   description={Solid State Drive}
% }

% \newglossaryentry{GHC}{
%   name=GHC,
%   description={Glasgow Haskell Compiler}
% }

% \newglossaryentry{VHDL}{
%   name=VHDL,
%   description={VHSIC Hardware Description Language}
% }

% \newglossaryentry{LUT}{
%   name=LUT,
%   description={Look-up table}
% }

% \newglossaryentry{AC}{
%   name=AC,
%   description={Air conditioning system}
% }


% % \setlength{\abovecaptionskip}{2pt}
% \setlength{\belowcaptionskip}{3em}
% % \addtolength{\subfigcapskip}{-1pt}

%% \usepackage[pagewise]{lineno}
%% \linenumbers

\newcommand*{\rplus}{\textit{r+}}
\newcommand*{\bool}{{\fontspec{DejaVu Sans} 𝔹}}
\newcommand*{\sem}{{\fontspec{DejaVu Sans} ⊨}}
\newcommand*{\eqdouble}{==}
% \newcommand*{\eqeq}{{\fontspec{Noto Sans Math Regular} ⩵}}

\tikzset{
%     node distance=2.5cm, % Minimum distance between two nodes. Change if necessary.
    every state/.style={ % Sets the properties for each state
    semithick,
    fill=gray!10},
    initial text={}, % No label on start arrow
%     double distance=2pt, % Adjust appearance of accept states
%     every edge/.style={ % Sets the properties for each transition
%     draw,
%     ->,>=stealth', % Makes edges directed with bold arrowheads
%     auto,
%     semithick}%
}


%%% New TikZ shape

% taken from manual
\makeatletter
\pgfdeclareshape{document}{
\inheritsavedanchors[from=rectangle] % this is nearly a rectangle
\inheritanchorborder[from=rectangle]
\inheritanchor[from=rectangle]{center}
\inheritanchor[from=rectangle]{north}
\inheritanchor[from=rectangle]{south}
\inheritanchor[from=rectangle]{west}
\inheritanchor[from=rectangle]{east}
% ... and possibly more
\backgroundpath{% this is new
% store lower right in xa/ya and upper right in xb/yb
\southwest \pgf@xa=\pgf@x \pgf@ya=\pgf@y
\northeast \pgf@xb=\pgf@x \pgf@yb=\pgf@y
% compute corner of ‘‘flipped page’’
\pgf@xc=\pgf@xb \advance\pgf@xc by-10pt % this should be a parameter
\pgf@yc=\pgf@yb \advance\pgf@yc by-10pt
% construct main path
\pgfpathmoveto{\pgfpoint{\pgf@xa}{\pgf@ya}}
\pgfpathlineto{\pgfpoint{\pgf@xa}{\pgf@yb}}
\pgfpathlineto{\pgfpoint{\pgf@xc}{\pgf@yb}}
\pgfpathlineto{\pgfpoint{\pgf@xb}{\pgf@yc}}
\pgfpathlineto{\pgfpoint{\pgf@xb}{\pgf@ya}}
\pgfpathclose
% add little corner
\pgfpathmoveto{\pgfpoint{\pgf@xc}{\pgf@yb}}
\pgfpathlineto{\pgfpoint{\pgf@xc}{\pgf@yc}}
\pgfpathlineto{\pgfpoint{\pgf@xb}{\pgf@yc}}
\pgfpathlineto{\pgfpoint{\pgf@xc}{\pgf@yc}}
}
}
\makeatother

\tikzstyle{doc}=[%
  draw,
  thick,
  align=center,
  color=black,
  shape=document,
  % minimum width=20mm,
  % minimum height=28.2mm,
  shape=document,
  inner sep=2ex,
]


% ****************************************************************************************************
% If you like the classicthesis, then I would appreciate a postcard.
% My address can be found in the file ClassicThesis.pdf. A collection
% of the postcards I received so far is available online at
% http://postcards.miede.de
% ****************************************************************************************************

\RequirePackage{silence} % :-\suppress an unnecessary warning in compilation
    \WarningFilter{scrreprt}{Usage of package `titlesec'}
    \WarningFilter{titlesec}{Non standard sectioning command detected}
    \WarningFilter{hyperref}{Token not allowed in a PDF string (PDFDocEncoding)}
   
% ****************************************************************************************************
% 0. Set the encoding of your files. UTF-8 is the only sensible encoding nowadays. If you can't read
% äöüßáéçèê∂åëæƒÏ€ then change the encoding setting in your editor, not the line below. If your editor
% does not support utf8 use another editor!
% ****************************************************************************************************
\PassOptionsToPackage{utf8}{inputenc}
  \usepackage{inputenc}

\PassOptionsToPackage{T1}{fontenc} % T2A for cyrillics
  \usepackage{fontenc}

\PassOptionsToPackage{final}{microtype}  %iitpkd
\usepackage[final]{microtype} %iitpkd

\usepackage{etoolbox}

\emergencystretch=1em

% *******************************************************************************************************************
% Note : Do not edit above this line
% Note : Do not edit options to classicthesis given below except the options eulermath, drafting, printready and numsupervisors
% % 
% *******************************************************************************************************************
\PassOptionsToPackage{
   tocaligned=false, 
  dottedtoc=true,
  eulerchapternumbers=true, 
  linedheaders=false,  
  eulermath=true, %true, %false %If false is set, cmmodern font will be used for math. % See chapter 3 of the template thesis to see the difference. %If text font is changed, be careful about this option.  
  beramono=true,   
  palatino=true,    
  style=classicthesis, 
  floatperchapter=true,     % numbering per chapter for all floats (i.e., Figure 1.1)
  numsupervisors=one, %one %two %three %set according to your number of supervisors
  %drafting=true,    % use this for draft submitted with synopsis. comment this line in the thesis for review
  drafting=false, % for final version this line is to be used
  printready=false, %Enable this for making title page and hyperreflinks in colour for screen reading
  %printready=true,  %Enable this for making title page and hyperreflinks in black for printing, margin adjustments
 }{classicthesis}


% ****************************************************************************************************
% 2. Personal data and user ad-hoc commands (insert your own data here)
% ****************************************************************************************************

\newcommand{\myTitle}{Towards verified regular expression matchers in hardware\xspace}
%\newcommand{\mySubtitle}{Subtitle of Your Thesis\xspace}
%\newcommand{\myDegree}{Doctor of Philosophy\xspace}
\newcommand{\myDegree}{Master of Science \xspace}
%\newcommand{\myDegree}{Master of Technology \xspace}
%\newcommand{\myDegree}{Master of Science in Engineering \xspace}
\newcommand{\myName}{Julin Shaji\xspace}
\newcommand{\myRollNo}{112103001\xspace}
%\newcommand{\myGender}{her\xspace}
\newcommand{\myGender}{him\xspace}
\newcommand{\myDepartment}{Department of Computer Science and Engineering\xspace}

\newcommand{\mySupervisorOne}{Dr. Piyush P. Kurur}
\newcommand{\mySupervisorOneDesig}{Associate Professor}%Example Assistant Professor
\newcommand{\mySupervisorOneDept}{Dept.~of Computer Science and Engg} %Example Dept.~of Computer Science and Engg.
\newcommand{\mySupervisorOneInstitute}{IIT Palakkad}% Example IIT Palakkad

% \newcommand{\mySupervisorTwo}{<Name of Supervisor2>}%used only if necessary
% \newcommand{\mySupervisorTwoDesig}{<Supervisor2 Designation>}%Example Assistant Professor
% \newcommand{\mySupervisorTwoDept}{<Supervisor2 Department>} %Example Dept.~of Computer Science and Engg.
% \newcommand{\mySupervisorTwoInstitute}{<Supervisor2 Institute>}% Example IIT Palakkad

\newcommand{\myUni}{Indian Institute of Technology Palakkad\xspace}
\newcommand{\myLocation}{Palakkad\xspace}
\newcommand{\myTime}{May 2025\xspace}
\newcommand{\myVersion}{\classicthesis}
\newcommand{\myDCMemberOne}{Dr. Unnikrishnan C.\xspace}
\newcommand{\myDCMemberOneDesigAndDept}{Assistant Professor, .Dept.~of Computer Science and Engineering\xspace}%Example: Assistant Professor, .Dept.~of Computer Science and Engineering
\newcommand{\myDCMemberOneInstitute}{IIT Palakkad}
%
\newcommand{\myDCMemberTwo}{Dr. Sandeep Chandran}
\newcommand{\myDCMemberTwoDesigAndDept}{Assistant Professor, .Dept.~of Computer Science and Engineering}
\newcommand{\myDCMemberTwoInstitute}{IIT Palakkad}
%
\newcommand{\myDCMemberThree}{Dr. Subrahmanyam Mula}
\newcommand{\myDCMemberThreeDesigAndDept}{Assistant Professor, .Dept.~of Electrical Engineering}
\newcommand{\myDCMemberThreeInstitute}{IIT Palakkad}

% ********************************************************************
% Setup, finetuning, and useful commands
% ********************************************************************
\providecommand{\mLyX}{L\kern-.1667em\lower.25em\hbox{Y}\kern-.125emX\@}
\newcommand{\ie}{i.\,e.}
\newcommand{\Ie}{I.\,e.}
\newcommand{\eg}{e.\,g.}
\newcommand{\Eg}{E.\,g.}
% ****************************************************************************************************


% ****************************************************************************************************
% 3. Loading some handy packages
% ****************************************************************************************************
% ********************************************************************
% Packages with options that might require adjustments
% ********************************************************************
\PassOptionsToPackage{ngerman,american}{babel} % change this to your language(s), main language last
\usepackage{babel}

\usepackage{csquotes}
%%%%%You may change the reference format to any of the options given agaist the parameters. 
%%%%%It is suggested that you maintain compatibility with natbib.
\PassOptionsToPackage{%
  %backend=biber,bibencoding=utf8, %instead of bibtex
  %backend=bibtex8,bibencoding=ascii,%
  language=auto,%
  style=numeric-comp,%
  %style=authoryear-comp, % Author 1999, 2010
  %bibstyle=authoryear,dashed=false, % dashed: substitute rep. author with ---
  sorting=nyt, % name, year, title
  maxbibnames=10, % default: 3, et al.
  backend=biber,
  %backref=true,%
  natbib=true % natbib compatibility mode (\citep and \citet still work)
}{biblatex}
\usepackage{biblatex}

\PassOptionsToPackage{fleqn}{amsmath}       % math environments and more by the AMS
   \usepackage{amsmath}

\usepackage{amssymb}
\usepackage{wasysym}
\usepackage{amsthm} 
\usepackage{mathrsfs}
\usepackage{mathtools}
\usepackage{epsfig}
%%%%Theorem like environments follow sectionwise numbering, 
%%%%only exceptions being 'Remark' and 'Note'.
%%%%Do not edit this basic setting, if you are adding more such environments.
\theoremstyle{plain}% default
\newtheorem{theorem}{Theorem}[section]
\newtheorem{lemma}[theorem]{Lemma}
\newtheorem{proposition}[theorem]{Proposition}
\newtheorem{corollary}[theorem]{Corollary}
\newtheorem{property}[theorem]{Property}

\theoremstyle{definition}
\newtheorem{definition}[theorem]{Definition}
\newtheorem{conjecture}[theorem]{Conjecture}
\newtheorem{example}[theorem]{Example}
\newtheorem{observation}[theorem]{Observation}

\theoremstyle{remark}
\newtheorem*{remark}{Remark}
\newtheorem*{note}{Note}
\newtheorem{claim}{Claim}[chapter]

\usepackage{thm-restate}
\usepackage[ruled,vlined,algochapter]{algorithm2e}

% ********************************************************************
% General useful packages
% ********************************************************************
\usepackage{graphicx} %

% \usepackage[newfloat]{minted}
% \newmintinline{haskell}{breaklines}

\usepackage{scrhack} % fix warnings when using KOMA with listings package
\usepackage{xspace} % to get the spacing after macros right
\PassOptionsToPackage{printonlyused,smaller}{acronym}
  \usepackage{acronym} % nice macros for handling all acronyms in the thesis
  \def\bflabel#1{{\acsfont{#1}\hfill}}
  \def\aclabelfont#1{\acsfont{#1}}
% ****************************************************************************************************
%\usepackage{pgfplots} % External TikZ/PGF support (thanks to Andreas Nautsch)
%\usetikzlibrary{external}
%\tikzexternalize[mode=list and make, prefix=ext-tikz/]
% ****************************************************************************************************




% ****************************************************************************************************
% 4. Setup floats: tables, (sub)figures, and captions
%%%Do not edit this setting
% ****************************************************************************************************
\usepackage{tabularx} % better tables
  \setlength{\extrarowheight}{3pt} % increase table row height
\newcommand{\tableheadline}[1]{\multicolumn{1}{l}{\spacedlowsmallcaps{#1}}}
\newcommand{\myfloatalign}{\centering} % to be used with each float for alignment
%\usepackage{subfig}
\usepackage{caption}
\usepackage{subcaption}



% ****************************************************************************************************


% ****************************************************************************************************
% 5. Setup code listings
%%Do not edit this setting
% ****************************************************************************************************
\usepackage{listings}
%\lstset{emph={trueIndex,root},emphstyle=\color{BlueViolet}}%\underbar} % for special keywords
\lstset{language=[LaTeX]Tex,%C++,
  morekeywords={PassOptionsToPackage,selectlanguage},
  keywordstyle=\color{CTkeyword},%\bfseries,
  basicstyle=\small\ttfamily,
  %identifierstyle=\color{NavyBlue},
  commentstyle=\color{CTcomment}\ttfamily,
  stringstyle=\rmfamily,
  numbers=none,%left,%
  numberstyle=\scriptsize,%\tiny
  stepnumber=5,
  numbersep=8pt,
  showstringspaces=false,
  breaklines=true,
  %frameround=ftff,
  %frame=single,
  belowcaptionskip=.75\baselineskip
  %frame=L
}
% ****************************************************************************************************

%%%%Note that for list of symbols to be inluded the following commands needs to be executed
%%%%pdflatex ClassicThesis.tex
%%%%makeindex ClassicThesis.nlo -s nomencl.ist -o ClassicThesis.nls
%%%%pdflatex ClassicThesis.tex
\usepackage{nomencl}
\makenomenclature
\renewcommand{\nomname}{List of Symbols}
\setlength{\nomlabelwidth}{2.5cm}
% ****************************************************************************************************
% 6. Last calls before the bar closes
% ****************************************************************************************************
% ********************************************************************
% Her Majesty herself
% ********************************************************************
\usepackage{classicthesis}


% ********************************************************************
% Fine-tune hyperreferences (hyperref should be called last)
%%%Do not edit this setting
% ********************************************************************
\hypersetup{%
  colorlinks=true, linktocpage=true, pdfstartpage=3, pdfstartview=FitV,%
  breaklinks=true, pageanchor=true,%
  pdfpagemode=UseNone, %
  plainpages=false, bookmarksnumbered, bookmarksopen=true, bookmarksopenlevel=1,%
  hypertexnames=true, pdfhighlight=/O,%
  urlcolor=CTurl, linkcolor=CTlink, citecolor=CTcitation, %
  pdftitle={\myTitle},%
  pdfauthor={\textcopyright\ \myName, \myUni},%
  pdfsubject={},%
  pdfkeywords={},%
  pdfcreator={xelatex},%
  %pdfcreator={pdfLaTeX},%
  %pdfproducer={LaTeX with hyperref and classicthesis}%
}

%%%%%%%%%Do not edit below this line except the very last line of this file %%%%%%%%%%%%%%%%%%%%%%%%%%%%%%%%%

% ********************************************************************
% Setup autoreferences (hyperref and babel)
% ********************************************************************
% There are some issues regarding autorefnames
% http://www.tex.ac.uk/cgi-bin/texfaq2html?label=latexwords
% you have to redefine the macros for the
% language you use, e.g., american, ngerman
% (as chosen when loading babel/AtBeginDocument)
% ********************************************************************
\makeatletter
\@ifpackageloaded{babel}%
  {%
    \addto\extrasamerican{%
      \renewcommand*{\figureautorefname}{Figure}%
      \renewcommand*{\tableautorefname}{Table}%
      \renewcommand*{\partautorefname}{Part}%
      \renewcommand*{\chapterautorefname}{Chapter}%
      \renewcommand*{\sectionautorefname}{Section}%
      \renewcommand*{\subsectionautorefname}{Section}%
      \renewcommand*{\subsubsectionautorefname}{Section}%
    }%
    \addto\extrasngerman{%
      \renewcommand*{\paragraphautorefname}{Absatz}%
      \renewcommand*{\subparagraphautorefname}{Unterabsatz}%
      \renewcommand*{\footnoteautorefname}{Fu\"snote}%
      \renewcommand*{\FancyVerbLineautorefname}{Zeile}%
      \renewcommand*{\theoremautorefname}{Theorem}%
      \renewcommand*{\appendixautorefname}{Anhang}%
      \renewcommand*{\equationautorefname}{Gleichung}%
      \renewcommand*{\itemautorefname}{Punkt}%
    }%
      % Fix to getting autorefs for subfigures right (thanks to Belinda Vogt for changing the definition)
      \providecommand{\subfigureautorefname}{\figureautorefname}%
    }{\relax}
\makeatother

\listfiles
\ifthenelse{\boolean{ct@printready}}%% line spread is kept larger for screen view option, because margin space is less
{
  \linespread{1.1} % this is for print ready
}
{
   \linespread{1.3} % increased from 1.1 to 1.3 in iitpkd v2.1 for screeview version
}   
%%%%%
%%% \renewcommand{\rmdefault}{pplx} %%uncomment this line if you want to change number typesetting style

%%%%%%%%%%%%%%%%%%%%%%%%%%%%%%%%%%%%%%%%%%%%%%%%%%%%%%%%%%%%%%%%%%%%%%%%%%%%%
% CUSTOM
%%%%%%%%%%%%%%%%%%%%%%%%%%%%%%%%%%%%%%%%%%%%%%%%%%%%%%%%%%%%%%%%%%%%%%%%%%%%%

\usepackage[acronym,shortcuts]{glossaries}
%
% https://tex.stackexchange.com/questions/98494/glossaries-dont-print-single-occurences
\glsenableentrycount % enable \cgls, \cglspl, \cGls, \cGlspl
\let\ac\cgls
\let\acpl\cglspl
\let\Ac\cGls
\let\Acpl\cGlspl
%
%\newglossary{symbols}{sym}{sbl}{List of Symbols}
\makeglossaries

\usepackage{nomencl}
\makenomenclature

% https://github.com/nickgian/thesis/blob/master/thesis.tex

\usepackage{mathpartir}
\usepackage{fontspec}
\usepackage{amssymb}
\usepackage{amsthm}
\usepackage{amsmath}
\usepackage{forest}
\usepackage{xcolor}
\usepackage{listingsutf8}
% \usepackage{lmodern}
\usepackage{pgfplots}
%\pgfplotsset{compat=1.18}
\usepackage{alectryon}
\usepackage{pygments}
\usepackage{fancyvrb}
\usepackage{turnstile}
%\usepackage{minted}
%% \usepackage{refcheck}

\newcommand\bbool{{\fontspec{Symbola} 𝔹}}
\newcommand\cmodels{{\fontspec{Symbola} ⊨}}
\newcommand\blackcircle{{\fontspec{Symbola} ●}}
\newcommand\alecmodels{{\fontspec{DejaVu Sans} ⊨}}
\newcommand\aleczero{{\fontspec{DejaVu Sans} 𝟘}}
\newcommand\alecone{{\fontspec{DejaVu Sans} 𝟙}}
\newcommand{\modelsere}{\sttstile{}{}} % three horizontal stripe models symbol
\newcommand{\modelbool}{\ddtstile{}{}} % two horizontal stripe models symbol
\newcommand{\modelppty}{\sdtstile{}{}} % three horizontal stripe models symbol

% Disjoint union
\newcommand\utimes{\mathbin{\ooalign{$\cup$\cr%
   \hfil\raise0.42ex\hbox{$\scriptscriptstyle\times$}\hfil\cr}}}
\newcommand\bigutimes{\mathop{\ooalign{$\bigcup$\cr%
   \hfil\raise0.36ex\hbox{$\scriptscriptstyle\boldsymbol{\times}$}\hfil\cr}}}



% For tblr env
\usepackage{tabularray}
\UseTblrLibrary{booktabs}

%\usepackage{subfig}
%\usepackage[acronym]{glossaries}

\usepackage{tikz}
\usetikzlibrary{arrows} % ...customizing arrows
\usetikzlibrary{arrows.meta}
\usetikzlibrary{automata} % Import library for drawing automata
\usetikzlibrary{calc}
\usetikzlibrary{decorations.markings}
\usetikzlibrary{decorations.pathreplacing}
\usetikzlibrary{positioning} % ...positioning nodes
\usetikzlibrary{shapes}
\usetikzlibrary{shapes.misc}
%
\usetikzlibrary{external}
\tikzexternalize
\usepackage{circuitikz}

\lstset{%
    basicstyle=\ttfamily\footnotesize,
    numbers=none,
    escapechar=\#,
}

% \lstset{%
%     numbers=none,
%     numberstyle=\tiny\color[rgb]{0.5,0.5,0.5},
%     basicstyle=\ttfamily\footnotesize,
%     basewidth=0.59em,
%     keywordstyle=[3]{},
%     commentstyle=\itshape\footnotesize,
%     tabsize=4,
%     frame=single,
%     frameround=tttt,
%     showstringspaces=false,
%     breaklines=true,
%     %breaklines=false,
%     captionpos=b,
%     aboveskip=\bigskipamount,
%     belowskip=\bigskipamount,
%     escapechar=#,
%     keywordstyle=\color[rgb]{0,0,1},
%     commentstyle=\color[rgb]{0.133,0.545,0.133},
%     stringstyle=\color[rgb]{0.627,0.126,0.941},
%     extendedchars=true,
% }


% \usepackage{lstcoq}
% \usepackage{lsthsk}
% \usepackage{lstverilog}
% 
% % Style options:
% % numberstyle,basicstyle,identifierstyle,commentstyle,stringstyle
% % keywordstyle=[1]{},keywordstyle=[2]{},directivestyle
% % \small\tiny\footnotesize\itshape\ttfamily\bf
% \lstdefinestyle{coq_style}{%
%   language=Coq, float=htb!
% }
% \lstnewenvironment{Coq}[2]{%
%   \nopagebreak
%   \lstset{style=coq_style,label={#1},caption={#2}}
% }{}
% 
% \newcommand{\includecode}[4][Coq]{%
%   \nopagebreak
%   \lstinputlisting[label={#2},caption={#3},style={#1_style}]{#4}
% }
% \newcommand{\lstheader}[2]{%
%   \begin{lstlisting}[label={#1},caption={#2},style=coq_style]
% }


% \def\lstlanguagefiles{defManSSR.tex}
% \lstset{language=SSR}

% \newcommand\doubleplus{+ \kern-1.3ex + \kern0.8ex}

% Twiggly arrow
\newcommand{\vto}{\mathrel{\leadsto}}
\newcommand{\cvto}{{\fontspec{DejaVu Sans} ↝}}

\newcommand\hsk[1]{\texttt{#1}}
\newcommand\cq[1]{\texttt{#1}}
\newcommand\code[1]{\texttt{#1}}

% Celsius
\usepackage{siunitx}

% \theoremstyle{plain}
\theoremstyle{definition}
\newtheorem*{objective}{Objective}

\newacronym{SuV}{SuV}{System under Verification}
\newacronym{SuO}{SuO}{System under Observation}
\newacronym{RV}{RV}{Runtime verification}
\newacronym{PSL}{PSL}{Property Specification Language}
\newacronym{SERE}{SERE}{Sequential Regular Expression}
\newacronym{LTL}{LTL}{Linear Temporal Logic}

\newacronym{DFA}{DFA}{Deterministic finite automata}
\newacronym{NFA}{NFA}{Non-deterministic finite automata}
\newacronym{TCB}{TCB}{Trusted Code Base}

\newacronym{HDL}{HDL}{Hardware Description Language}
\newacronym{FM}{FM}{Formal methods}
\newacronym{FV}{FV}{Formal verification}

\newacronym{BSV}{BSV}{Bluespec SystemVerilog}

\newacronym{DSL}{DSL}{Domain Specific Language}
\newacronym{eDSL}{eDSL}{Embedded Domain Specific Language}
\newacronym{LOC}{LOC}{Lines of Code}
\newacronym{STA}{STA}{Static Timing Analysis}

\newacronym{GADT}{GADT}{Generalized Algebraic Data Type}
\newacronym{ADT}{ADT}{Algebraic Data Type}

\newacronym{CPU}{CPU}{Central Processing Unit}
\newacronym{GPU}{GPU}{Graphics Processing Unit}
\newacronym{GPGPU}{GPGPU}{General Purpose Graphics Processing Unit}

% \newacronym{NIDS}{NIDS}{Network Intrusion Detection System}
\newacronym{PLA}{PLA}{Programmable Logic Array}
\newacronym{FOSS}{FOSS}{Free and Open Source Software}
\newacronym{REPL}{REPL}{Read-Eval-Print-Loop}
\newacronym{IP}{IP}{Intellectual Property}

%% \newacronym[
%%     \glslongpluralkey={Regular expressions},
%%     \glsshortpluralkey={regexes}
%%   ]{TaA}{regex}{regular expression}

\newacronym[
    \glslongpluralkey={Regular expressions},
    \glsshortpluralkey={regexes}
  ]{rgx}{regex}{regular expression}

%% % \newacronym[
%% %     \glslongpluralkey={Tests-as-atoms regexes},
%% %     \glsshortpluralkey={T-regexes}
%% %   ]{TaA}{T-regex}{Tests-as-atoms regex}

%% \newacronym[
%%     \glslongpluralkey={Regular expressions},
%%     \glsshortpluralkey={regexes}
%%   ]{regex}{regex}{Regular expression}

\newacronym{EDA}{EDA}{Electronic Design Automation}
\newacronym{FPGA}{FPGA}{Field Programmable Gate Array}
\newacronym{ASIC}{ASIC}{Application Specific Integrated Circuit}
\newacronym{SSD}{SSD}{Solid State Drive}
\newacronym{GHC}{GHC}{Glasgow Haskell Compiler}
\newacronym{VHDL}{VHDL}{VHSIC Hardware Description Language}
\newacronym{LUT}{LUT}{Look-up table}
% \newacronym{AC}{AC}{Air conditioning system}
\newacronym{RAM}{RAM}{Random Access Memory}

\newcommand{\bvec}{Boolean vector}
\newcommand{\bvecs}{Boolean vectors}

% \newglossaryentry{TaA}{
%   name=T-regex,
%   description={Tests-as-atoms regex},
%   plural=T-regexes
% }

% \newglossaryentry{regex}{
%   name=regex,
%   description={Regular expression},
%   plural=regexes
% }

% \newglossaryentry{EDA}{
%   name=EDA,
%   description={Electronic Design Automation}
% }

% \newglossaryentry{FPGA}{
%   name=FPGA,
%   description={Field Programmable Gate Array}
% }

% \newglossaryentry{ASIC}{
%   name=FPGA,
%   description={Application Specific Integrated Circuit}
% }

% \newglossaryentry{SSD}{
%   name=SSD,
%   description={Solid State Drive}
% }

% \newglossaryentry{GHC}{
%   name=GHC,
%   description={Glasgow Haskell Compiler}
% }

% \newglossaryentry{VHDL}{
%   name=VHDL,
%   description={VHSIC Hardware Description Language}
% }

% \newglossaryentry{LUT}{
%   name=LUT,
%   description={Look-up table}
% }

% \newglossaryentry{AC}{
%   name=AC,
%   description={Air conditioning system}
% }


% % \setlength{\abovecaptionskip}{2pt}
% \setlength{\belowcaptionskip}{3em}
% % \addtolength{\subfigcapskip}{-1pt}

%% \usepackage[pagewise]{lineno}
%% \linenumbers

\newcommand*{\rplus}{\textit{r+}}
\newcommand*{\bool}{{\fontspec{DejaVu Sans} 𝔹}}
\newcommand*{\sem}{{\fontspec{DejaVu Sans} ⊨}}
\newcommand*{\eqdouble}{==}
% \newcommand*{\eqeq}{{\fontspec{Noto Sans Math Regular} ⩵}}

\tikzset{
%     node distance=2.5cm, % Minimum distance between two nodes. Change if necessary.
    every state/.style={ % Sets the properties for each state
    semithick,
    fill=gray!10},
    initial text={}, % No label on start arrow
%     double distance=2pt, % Adjust appearance of accept states
%     every edge/.style={ % Sets the properties for each transition
%     draw,
%     ->,>=stealth', % Makes edges directed with bold arrowheads
%     auto,
%     semithick}%
}


%%% New TikZ shape

% taken from manual
\makeatletter
\pgfdeclareshape{document}{
\inheritsavedanchors[from=rectangle] % this is nearly a rectangle
\inheritanchorborder[from=rectangle]
\inheritanchor[from=rectangle]{center}
\inheritanchor[from=rectangle]{north}
\inheritanchor[from=rectangle]{south}
\inheritanchor[from=rectangle]{west}
\inheritanchor[from=rectangle]{east}
% ... and possibly more
\backgroundpath{% this is new
% store lower right in xa/ya and upper right in xb/yb
\southwest \pgf@xa=\pgf@x \pgf@ya=\pgf@y
\northeast \pgf@xb=\pgf@x \pgf@yb=\pgf@y
% compute corner of ‘‘flipped page’’
\pgf@xc=\pgf@xb \advance\pgf@xc by-10pt % this should be a parameter
\pgf@yc=\pgf@yb \advance\pgf@yc by-10pt
% construct main path
\pgfpathmoveto{\pgfpoint{\pgf@xa}{\pgf@ya}}
\pgfpathlineto{\pgfpoint{\pgf@xa}{\pgf@yb}}
\pgfpathlineto{\pgfpoint{\pgf@xc}{\pgf@yb}}
\pgfpathlineto{\pgfpoint{\pgf@xb}{\pgf@yc}}
\pgfpathlineto{\pgfpoint{\pgf@xb}{\pgf@ya}}
\pgfpathclose
% add little corner
\pgfpathmoveto{\pgfpoint{\pgf@xc}{\pgf@yb}}
\pgfpathlineto{\pgfpoint{\pgf@xc}{\pgf@yc}}
\pgfpathlineto{\pgfpoint{\pgf@xb}{\pgf@yc}}
\pgfpathlineto{\pgfpoint{\pgf@xc}{\pgf@yc}}
}
}
\makeatother

\tikzstyle{doc}=[%
  draw,
  thick,
  align=center,
  color=black,
  shape=document,
  % minimum width=20mm,
  % minimum height=28.2mm,
  shape=document,
  inner sep=2ex,
]

