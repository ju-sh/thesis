% https://github.com/nickgian/thesis/blob/master/thesis.tex

\usepackage{mathpartir}
\usepackage{fontspec}
\usepackage{amssymb}
\usepackage{amsthm}
\usepackage{amsmath}
\usepackage{forest}
\usepackage{xcolor}
\usepackage{listingsutf8}
% \usepackage{lmodern}
\usepackage{pgfplots}
%\pgfplotsset{compat=1.18}
\usepackage{alectryon}
\usepackage{pygments}
\usepackage{fancyvrb}
\usepackage{turnstile}
%\usepackage{minted}
%% \usepackage{refcheck}

\newcommand\bbool{{\fontspec{Symbola} 𝔹}}
\newcommand\cmodels{{\fontspec{Symbola} ⊨}}
\newcommand\blackcircle{{\fontspec{Symbola} ●}}
\newcommand\alecmodels{{\fontspec{DejaVu Sans} ⊨}}
\newcommand\aleczero{{\fontspec{DejaVu Sans} 𝟘}}
\newcommand\alecone{{\fontspec{DejaVu Sans} 𝟙}}
\newcommand{\modelsere}{\sttstile{}{}} % three horizontal stripe models symbol
\newcommand{\modelbool}{\ddtstile{}{}} % two horizontal stripe models symbol
\newcommand{\modelppty}{\sdtstile{}{}} % three horizontal stripe models symbol

% Disjoint union
\newcommand\utimes{\mathbin{\ooalign{$\cup$\cr%
   \hfil\raise0.42ex\hbox{$\scriptscriptstyle\times$}\hfil\cr}}}
\newcommand\bigutimes{\mathop{\ooalign{$\bigcup$\cr%
   \hfil\raise0.36ex\hbox{$\scriptscriptstyle\boldsymbol{\times}$}\hfil\cr}}}



% For tblr env
\usepackage{tabularray}
\UseTblrLibrary{booktabs}

%\usepackage{subfig}
%\usepackage[acronym]{glossaries}

\usepackage{tikz}
\usetikzlibrary{arrows} % ...customizing arrows
\usetikzlibrary{arrows.meta}
\usetikzlibrary{automata} % Import library for drawing automata
\usetikzlibrary{calc}
\usetikzlibrary{decorations.markings}
\usetikzlibrary{decorations.pathreplacing}
\usetikzlibrary{positioning} % ...positioning nodes
\usetikzlibrary{shapes}
\usetikzlibrary{shapes.misc}
%
\usetikzlibrary{external}
\tikzexternalize
\usepackage{circuitikz}

\lstset{%
    basicstyle=\ttfamily\footnotesize,
    numbers=none,
    escapechar=\#,
}

% \lstset{%
%     numbers=none,
%     numberstyle=\tiny\color[rgb]{0.5,0.5,0.5},
%     basicstyle=\ttfamily\footnotesize,
%     basewidth=0.59em,
%     keywordstyle=[3]{},
%     commentstyle=\itshape\footnotesize,
%     tabsize=4,
%     frame=single,
%     frameround=tttt,
%     showstringspaces=false,
%     breaklines=true,
%     %breaklines=false,
%     captionpos=b,
%     aboveskip=\bigskipamount,
%     belowskip=\bigskipamount,
%     escapechar=#,
%     keywordstyle=\color[rgb]{0,0,1},
%     commentstyle=\color[rgb]{0.133,0.545,0.133},
%     stringstyle=\color[rgb]{0.627,0.126,0.941},
%     extendedchars=true,
% }


% \usepackage{lstcoq}
% \usepackage{lsthsk}
% \usepackage{lstverilog}
% 
% % Style options:
% % numberstyle,basicstyle,identifierstyle,commentstyle,stringstyle
% % keywordstyle=[1]{},keywordstyle=[2]{},directivestyle
% % \small\tiny\footnotesize\itshape\ttfamily\bf
% \lstdefinestyle{coq_style}{%
%   language=Coq, float=htb!
% }
% \lstnewenvironment{Coq}[2]{%
%   \nopagebreak
%   \lstset{style=coq_style,label={#1},caption={#2}}
% }{}
% 
% \newcommand{\includecode}[4][Coq]{%
%   \nopagebreak
%   \lstinputlisting[label={#2},caption={#3},style={#1_style}]{#4}
% }
% \newcommand{\lstheader}[2]{%
%   \begin{lstlisting}[label={#1},caption={#2},style=coq_style]
% }


% \def\lstlanguagefiles{defManSSR.tex}
% \lstset{language=SSR}

% \newcommand\doubleplus{+ \kern-1.3ex + \kern0.8ex}

% Twiggly arrow
\newcommand{\vto}{\mathrel{\leadsto}}
\newcommand{\cvto}{{\fontspec{DejaVu Sans} ↝}}

\newcommand\hsk[1]{\texttt{#1}}
\newcommand\cq[1]{\texttt{#1}}
\newcommand\code[1]{\texttt{#1}}

% Celsius
\usepackage{siunitx}

% \theoremstyle{plain}
\theoremstyle{definition}
\newtheorem*{objective}{Objective}

\newacronym{SuV}{SuV}{System under Verification}
\newacronym{SuO}{SuO}{System under Observation}
\newacronym{RV}{RV}{Runtime verification}
\newacronym{PSL}{PSL}{Property Specification Language}
\newacronym{SERE}{SERE}{Sequential Regular Expression}
\newacronym{LTL}{LTL}{Linear Temporal Logic}

\newacronym{DFA}{DFA}{Deterministic finite automata}
\newacronym{NFA}{NFA}{Non-deterministic finite automata}
\newacronym{TCB}{TCB}{Trusted Code Base}

\newacronym{HDL}{HDL}{Hardware Description Language}
\newacronym{FM}{FM}{Formal methods}
\newacronym{FV}{FV}{Formal verification}

\newacronym{BSV}{BSV}{Bluespec SystemVerilog}

\newacronym{DSL}{DSL}{Domain Specific Language}
\newacronym{eDSL}{eDSL}{Embedded Domain Specific Language}
\newacronym{LOC}{LOC}{Lines of Code}
\newacronym{STA}{STA}{Static Timing Analysis}

\newacronym{GADT}{GADT}{Generalized Algebraic Data Type}
\newacronym{ADT}{ADT}{Algebraic Data Type}

\newacronym{CPU}{CPU}{Central Processing Unit}
\newacronym{GPU}{GPU}{Graphics Processing Unit}
\newacronym{GPGPU}{GPGPU}{General Purpose Graphics Processing Unit}

% \newacronym{NIDS}{NIDS}{Network Intrusion Detection System}
\newacronym{PLA}{PLA}{Programmable Logic Array}
\newacronym{FOSS}{FOSS}{Free and Open Source Software}
\newacronym{REPL}{REPL}{Read-Eval-Print-Loop}
\newacronym{IP}{IP}{Intellectual Property}

%% \newacronym[
%%     \glslongpluralkey={Regular expressions},
%%     \glsshortpluralkey={regexes}
%%   ]{TaA}{regex}{regular expression}

\newacronym[
    \glslongpluralkey={Regular expressions},
    \glsshortpluralkey={regexes}
  ]{rgx}{regex}{regular expression}

%% % \newacronym[
%% %     \glslongpluralkey={Tests-as-atoms regexes},
%% %     \glsshortpluralkey={T-regexes}
%% %   ]{TaA}{T-regex}{Tests-as-atoms regex}

%% \newacronym[
%%     \glslongpluralkey={Regular expressions},
%%     \glsshortpluralkey={regexes}
%%   ]{regex}{regex}{Regular expression}

\newacronym{EDA}{EDA}{Electronic Design Automation}
\newacronym{FPGA}{FPGA}{Field Programmable Gate Array}
\newacronym{ASIC}{ASIC}{Application Specific Integrated Circuit}
\newacronym{SSD}{SSD}{Solid State Drive}
\newacronym{GHC}{GHC}{Glasgow Haskell Compiler}
\newacronym{VHDL}{VHDL}{VHSIC Hardware Description Language}
\newacronym{LUT}{LUT}{Look-up table}
% \newacronym{AC}{AC}{Air conditioning system}
\newacronym{RAM}{RAM}{Random Access Memory}

\newcommand{\bvec}{Boolean vector}
\newcommand{\bvecs}{Boolean vectors}

% \newglossaryentry{TaA}{
%   name=T-regex,
%   description={Tests-as-atoms regex},
%   plural=T-regexes
% }

% \newglossaryentry{regex}{
%   name=regex,
%   description={Regular expression},
%   plural=regexes
% }

% \newglossaryentry{EDA}{
%   name=EDA,
%   description={Electronic Design Automation}
% }

% \newglossaryentry{FPGA}{
%   name=FPGA,
%   description={Field Programmable Gate Array}
% }

% \newglossaryentry{ASIC}{
%   name=FPGA,
%   description={Application Specific Integrated Circuit}
% }

% \newglossaryentry{SSD}{
%   name=SSD,
%   description={Solid State Drive}
% }

% \newglossaryentry{GHC}{
%   name=GHC,
%   description={Glasgow Haskell Compiler}
% }

% \newglossaryentry{VHDL}{
%   name=VHDL,
%   description={VHSIC Hardware Description Language}
% }

% \newglossaryentry{LUT}{
%   name=LUT,
%   description={Look-up table}
% }

% \newglossaryentry{AC}{
%   name=AC,
%   description={Air conditioning system}
% }


% % \setlength{\abovecaptionskip}{2pt}
% \setlength{\belowcaptionskip}{3em}
% % \addtolength{\subfigcapskip}{-1pt}

%% \usepackage[pagewise]{lineno}
%% \linenumbers

\newcommand*{\rplus}{\textit{r+}}
\newcommand*{\bool}{{\fontspec{DejaVu Sans} 𝔹}}
\newcommand*{\sem}{{\fontspec{DejaVu Sans} ⊨}}
\newcommand*{\eqdouble}{==}
% \newcommand*{\eqeq}{{\fontspec{Noto Sans Math Regular} ⩵}}

\tikzset{
%     node distance=2.5cm, % Minimum distance between two nodes. Change if necessary.
    every state/.style={ % Sets the properties for each state
    semithick,
    fill=gray!10},
    initial text={}, % No label on start arrow
%     double distance=2pt, % Adjust appearance of accept states
%     every edge/.style={ % Sets the properties for each transition
%     draw,
%     ->,>=stealth', % Makes edges directed with bold arrowheads
%     auto,
%     semithick}%
}


%%% New TikZ shape

% taken from manual
\makeatletter
\pgfdeclareshape{document}{
\inheritsavedanchors[from=rectangle] % this is nearly a rectangle
\inheritanchorborder[from=rectangle]
\inheritanchor[from=rectangle]{center}
\inheritanchor[from=rectangle]{north}
\inheritanchor[from=rectangle]{south}
\inheritanchor[from=rectangle]{west}
\inheritanchor[from=rectangle]{east}
% ... and possibly more
\backgroundpath{% this is new
% store lower right in xa/ya and upper right in xb/yb
\southwest \pgf@xa=\pgf@x \pgf@ya=\pgf@y
\northeast \pgf@xb=\pgf@x \pgf@yb=\pgf@y
% compute corner of ‘‘flipped page’’
\pgf@xc=\pgf@xb \advance\pgf@xc by-10pt % this should be a parameter
\pgf@yc=\pgf@yb \advance\pgf@yc by-10pt
% construct main path
\pgfpathmoveto{\pgfpoint{\pgf@xa}{\pgf@ya}}
\pgfpathlineto{\pgfpoint{\pgf@xa}{\pgf@yb}}
\pgfpathlineto{\pgfpoint{\pgf@xc}{\pgf@yb}}
\pgfpathlineto{\pgfpoint{\pgf@xb}{\pgf@yc}}
\pgfpathlineto{\pgfpoint{\pgf@xb}{\pgf@ya}}
\pgfpathclose
% add little corner
\pgfpathmoveto{\pgfpoint{\pgf@xc}{\pgf@yb}}
\pgfpathlineto{\pgfpoint{\pgf@xc}{\pgf@yc}}
\pgfpathlineto{\pgfpoint{\pgf@xb}{\pgf@yc}}
\pgfpathlineto{\pgfpoint{\pgf@xc}{\pgf@yc}}
}
}
\makeatother

\tikzstyle{doc}=[%
  draw,
  thick,
  align=center,
  color=black,
  shape=document,
  % minimum width=20mm,
  % minimum height=28.2mm,
  shape=document,
  inner sep=2ex,
]
