%*******************************************************
% Abstract
%*******************************************************
%\renewcommand{\abstractname}{Abstract}
\pdfbookmark[1]{Abstract}{Abstract}
% \addcontentsline{toc}{chapter}{\tocEntry{Abstract}}
\begingroup
\let\clearpage\relax
\let\cleardoublepage\relax
\let\cleardoublepage\relax
%%%%**********************************************************************************
%%% Note: Do not edit above this line
%%%%**********************************************************************************

\chapter*{Abstract}
% Safety-critical systems cannot be allowed to go wrong.
% Hence, some means of attaining a level of guarantee about the
% correctness of their working is highly desirable.
% Formal verification is useful for proving systems correct which,
% unlike testing, can actually prove absence of errors with respect to
% correctness specification.
% There are cases where system's internal data cannot be used for
% reasoning about its correctness and only its observable behaviour can
% be reasoned about.
% \gls{RV} is a dynamic form of formal verification that can
% be used to ensure system correctness while it is working.
% It involves having a monitor constantly check the system for
% deviations from correct behaviour. 
% %
% Since \gls{RV} entrusts monitor to perform the verification,
% correctness of monitors themselves is important. 
% Correctness specifications of such monitors need to be written in a formal
% language.
% \gls{PSL} is a formal language commonly used to write hardware correctness
% specifications.
% Properties in this language can be written using a generalized form of
% regular expressions called \glspl{SERE}.
% This means that a checker for \gls{PSL} properties needs a regular
% expression matcher that is more expressive than classic regex.
% %
% In this thesis, we produce a workflow by which a subset of
% \glspl{SERE} can be converted to hardware implementations of their
% matchers from within the Coq proof assistant.
% %
% The resultant matchers can readily be loaded onto FPGAs and can operate at
% reasonable speed.

Regular expression matching is used in a wide variety of
applications ranging from compilers to computational biology.
We are interested in this task because of their usefulness in
\emph{runtime verification}, which is used to ensure the correctness
of reactive systems.
%
Runtime verification involves the design and use of a \emph{monitor}
which observes the behaviour of a running system and flags deviations
from a formal specification describing its intended behaviour.
%
The effectiveness of runtime verification depends on the correctness
of the monitor.
Hence, the confidence in overall system correctness can be enhanced if
the monitor is proven to be correct.

% Specifications used in runtime verification are usually written in a
% formal language based on a temporal logic.
% We are interested in regular expression matching because of its use in
% one such language named \emph{Property Specification Language} (PSL).
% This language is commonly used to write correctness specifications in
% hardware verification.
% %
% Properties in PSL can be written using a generalized form of regular
% expressions.
% Hence, a monitor for specifications written in PSL needs a regular
% expression matcher.

In this work, we provide a mechanism that accepts a generalized form of
regular expressions within the theorem prover Coq and produces the
hardware implementation for a corresponding matcher.
The resultant matchers are reasonably small and can be used in
practical applications like monitoring liveness properties.
Our experimental results show that our designs have reasonably low
resource utilization with the growth in their size limited to linear
bounds.
Hence, designs are useful for practical use cases.

\bigskip
%\vspace*{\fill}
\vfill
\begingroup
\noindent
{\textbf{Keywords:} Formal methods, Runtime verification, Hardware}\\
\smallskip
\noindent
% {\textbf{AMS Subject Classification:} Give Class here . Remove if irrelevant}\\
% \smallskip
\noindent
{\textbf{ACM Subject Classification:} Logic and verification}
\endgroup
\endgroup

\vfill
